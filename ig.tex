\subsection{Interaction Grammars}

We will briefly discuss one more grammar formalism and that is the
formalism of Interaction Grammars \cite{guillaume2009interaction}.
Interaction grammars are a grammatical formalism centered around the
concept of \emph{polarity}. An interaction grammar is a set of
(underspecified) tree descriptions which define a set of trees that can
be constructed by superposing some of these tree descriptions while
respecting the polarities described in the nodes of the individual tree
descriptions.

We will illustrate the objects and mechanisms of interaction grammars on
a simple example sentence.

\begin{exe}
  \ex \label{ex:ig} Jean la voit.
\end{exe}

\begin{figure}
  \centering
  \includegraphics[scale=0.25]{ig-parse.pdf}
  \caption{\label{fig:ig-parse} Syntactic tree of sentence
    (\ref{ex:ig}). The numbers at the top of the nodes are labels of
    nodes from the elementary polarized tree descriptions that generated
    the syntactic tree (see Figure \ref{fig:ig-eptds}).}
\end{figure}

Figure \ref{fig:ig-parse} shows the syntactic tree assigned by Frigram
to the sentence in (\ref{ex:ig}). It is a rooted ordered tree with the
topmost node being the root. Solid lines indicate immediate dominance
with the higher node being the parent and the lower the child. The
ordering of children is given by the arrows, which signify the immediate
precedence relation between sister nodes.

The nodes in the tree are of three different kinds with respect to the
phonological form of their subtree. We recognize the \emph{anchor
  nodes}, which are displayed in vivid yellow and which are leaf nodes
containing some non-empty string as their phonological content (written
in a gray rectangle at the top of the node). Then we have the
\emph{empty} nodes, which are drawn in white and whose phonological form
is empty. Finally, there are the pale yellow \emph{non-empty} nodes,
internal nodes whose phonological content (the yield of the subtree) is
not empty.

The nodes of the syntactic tree are also decorated with
features. Interaction grammars are a formalism enabling us to define
sets of the structures described above, the \emph{syntactic
  trees}. Similar to Tree Adjoining Grammars, interaction grammars are
formulated in terms of a set of elementary structures which can combine
to produce the final output structures. However, unlike in tree
adjoining grammars, the output structure is not constructed from the
elementary structures via some set of algebraic operations (substitution
and adjunction in case of TAG). Instead, \emph{polarized tree
  descriptions} (PTDs), the elementary structures of interaction
grammars, impose constraints on the final structure and a structure is
said to be generated by some PTDs if it is a minimal structure
satisfying those constraints (it is a \emph{minimal model}).

\begin{figure}
  \centering
  \includegraphics[scale=0.25]{ig-eptds.pdf}
  \caption{\label{fig:ig-eptds} The elementary polarized tree
    descriptions used to generate the sentence (\ref{ex:ig}).}
\end{figure}

In Figure~\ref{fig:ig-eptds}, we can see the elementary polarized tree
descriptions (EPTDs) that generated the parse tree in
Figure~\ref{fig:ig-parse}.

As we said before, a PTD is a set of constraints on some syntactic
tree. Let us expound on what constraints the structures of
Figure~\ref{fig:ig-eptds} impose.

A node in a PTD can be read as a statement that there must exist a node
in the final syntactic tree that has compatible values for all the
features and that carries the same phonological string, if any. Such a
node of the syntactic tree is then called the \emph{interpretation} of
the PTD node (in Figure~\ref{fig:ig-parse}, every node of the syntactic
tree bears a list of the PTD nodes that it interprets, so you can see
exactly how the interpretation function works in our example).

A solid line between two nodes in a PTD means that the interpretations
of these two nodes must be in a parent-child relationship
(\emph{immediate dominance}). A dashed green arrow tells us that the
interpretations have to be sister nodes with the former preceding the
latter in the ordered tree (\emph{precedence}). Note that not all sister
nodes in a PTD have to be linked with this precedence relation. See for
example the nodes (1,1) and (1,5) in the EPTD of the clitic pronoun
$la$.

The formalism also allows us to specify \emph{immediate precedence} and
general \emph{dominance}. The latter is useful for modelling unbounded
dependencies such as those between a relative pronoun and its trace in
some embedded clause of the relative clause, but it is not used in our
present example. Finally, one kind of constraint that is used in our
example is the orange rectangle in node (3,0), which requires that the
interpretation of (3,0) is the right-most daughter amongst its siblings.

Now that we have covered the structural constraints imposed by the tree
descriptions, we will turn our attention to the defining characteristic
of interaction grammars, the polarities. As you have noticed on
Figure~\ref{fig:ig-eptds}, some of the features in our PTDs are
annotated with special symbols and colors. These denote the different
polarities and are used by the formalism for two distinct purposes: the
positive ($\textcolor{red}{\rightarrow}$) and negative
($\textcolor{blue}{\leftarrow}$) polarities are used to model the
resource sensitivity of language, while the virtual polarities
($\textcolor{purple}{\sim}$) are used to match against the
surrounding context.

The way the polarities are handled is that every model (output syntactic
tree) is required to have only saturated polarities on its features
(i.e. no positive, negative or virtual polarities). Whenever more than
one node of the PTD is mapped onto a single node of the syntactic tree,
the polarities of each feature are combined. The combination mechanism
allows to combine a positively polarized feature with a negatively
polarized one to yield a saturated one. Virtual features can be factored
out only by combining with a saturated polarity.

If we look at the PTDs of Figure~\ref{fig:ig-eptds}, we can see this
polarity mechanism in action. The EPTD of $Jean$ has a positive $cat$
feature in its root node saying that it provides one $np$, and a
negative $funct$ feature saying that it expects some function. In the
EPTD of $voit$, we have two nodes for the subject and object, both of
them expecting an $np$ and providing them the $subj$ and $obj$
functions, respectively. The root of the EPTD than provides a complete
sentence of category $s$ and expects some function that this sentence
will play. The full stop EPTD finalizes the sentence by accepting a node
with category $s$ and giving it a $void$ function.

The clitic pronoun $la$ participates in the positive/negative resource
management system as well, since using the clitic fills up the object
slot in the valency of a verb. The EPTD of $la$ also relies heavily on
virtual features and uses them to specify in more detail the context in
which this rule is applicable, so as to avoid over-generation.

Now that we understand the constraints imposed by PTDs, we can start to
see that the syntactic tree in Figure~\ref{fig:ig-parse} is truly a
model of the PTDS given in Figure~\ref{fig:ig-eptds} and more than that,
it is the only minimal model. To illustrate more clearly how the
polarities and constraints of the individual PTDs end up generating the
syntactic tree of Figure~\ref{fig:ig-parse}, we can look at the result
of superposing the EPTDs of $la$ and $voit$ by merging the nodes (1,2)
and (1,5) with (2,2) and (2,4) respectively on
Figure~\ref{fig:ig-partial}.

\begin{figure}
  \centering
  \includegraphics[scale=0.25]{ig-partial.pdf}
  \caption{\label{fig:ig-partial} The result of merging the EPTDs of
    $la$ and $voit$ by merging the nodes (1,5) and (1,5) with (2,2) and
    (2,4), respectively.}
\end{figure}
