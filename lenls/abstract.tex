\documentclass[twocolumn]{article}

\usepackage[utf8]{inputenc}
\usepackage{amsmath}
\usepackage{amsthm}
\usepackage{xcolor}
\usepackage{graphicx}
\usepackage{gb4e}
\usepackage{bussproofs}
\usepackage{hyperref}
\usepackage[labelformat=simple]{subcaption}
\renewcommand\thesubfigure{(\alph{subfigure})}
\usepackage{cancel}
\usepackage[nottoc,numbib]{tocbibind}
\usepackage{pdfpages}
\usepackage{enumitem}
\usepackage{titling}

\setlength{\droptitle}{-14em}
\title{Integration of Multiple Constraints in ACG}
\date{\vspace{-5em}}

%% Linear implication from Alessio Guglielmi
%% https://groups.google.com/forum/?fromgroups=#!topic/comp.text.tex/0B4C3F_BsVI
\def\limp {\mathbin{{-}\mkern-3.5mu{\circ}}}

\newtheorem*{observation}{Observation}
\newtheorem*{theorem}{Theorem}
\newtheorem*{corollary}{Corollary}

\begin{document}

\maketitle

This proposal is a first step towards a wide-coverage Abstract Categorial
Grammar (ACG) that could be used to automatically build discourse-level
representations. We focus on the challenge of integrating the treatment of
disparate linguistic constraints in a single ACG and propose a generalization
of the formalism: Graphical Abstract Categorial Grammars.


\section{Motivation}

Abstract Categorial Grammars (ACGs) have shown to be a viable formalism for
elegantly encoding the dynamic nature of discourse. Proposals based on
continuation semantics \cite{de2006towards} have tackled topics such as event
anaphora \cite{qian2011event}, SDRT discourse structure \cite{asher2011sdrt}
and modal accessibility constraints \cite{asher2011montagovian}. However, all
of these treatments only consider tiny fragments of languages. We are
interested in building a wide-coverage grammar which integrates and reconciles
the existing formal treatments of discourse and allows us to study their
interactions and to build discourse representations automatically.

In this paper, we focus on the problem of constructing a wide-coverage ACG. We
identify a problem that comes up when one tries to enforce multiple
constraints/dependencies in a single ACG and we propose a generalization of
the formalism which resolves the issue.


\begin{figure*}[t]
  \centering
  \begin{subfigure}[b]{0.25\textwidth}
    \centering
    \includegraphics[height=0.2\textheight]{../diagrams/double-acg.pdf}
    \caption{\label{fig:acg-comp-basic} Connecting form with meaning.}
  \end{subfigure}
  \qquad
  \begin{subfigure}[b]{0.25\textwidth}
    \centering
    \includegraphics[height=0.2\textheight]{../diagrams/serial-over-parallel.pdf}
    \caption{\label{fig:acg-comp-constr} Adding a constraint.}
  \end{subfigure}
  \qquad
  \begin{subfigure}[b]{0.25\textwidth}
    \centering
    \includegraphics[height=0.2\textheight]{../diagrams/parallel-over-serial.pdf}
    \caption{\label{fig:acg-comp-sem} Distinguishing syntactic and
      semantic ambiguities.}
  \end{subfigure}
  \caption{\label{fig:acg-comp} Diagrams of systems of ACGs.}
\end{figure*}


\section{Systems of ACGs}
\label{sec:acg-comp}

An ACG is defined in terms of two higher-order signatures, sets of typed
constants on which we can form well-typed lambda terms, and a
translating function from terms of one of the signatures to terms of the
other (the former signature being called the \emph{abstract signature},
the latter being called the \emph{object signature} and the function
being called a \emph{lexicon}). We are then interested in the terms of
the abstract signature having some distinguished type and their images
through the lexicon (these two sets of terms are called the
\emph{abstract language} and the \emph{object language}, respectively).

When describing languages, we are usually interested in systems of more than
one ACG, and diagrams such as the ones on Figure~\ref{fig:acg-comp} have been
used to explain them.

In Figure~\ref{fig:acg-comp-basic}, abstract syntactic terms from the
signature $\Sigma_{Synt}$ yield phonological strings and semantic
representations using two different lexicons. In
Figure~\ref{fig:acg-comp-constr}, this pattern is extended by adding a
new abstract signature which narrows the language of syntactic terms and
implements some linguistic constraint
\cite{pogodalla2012controlling}. Finally, in
Figure~\ref{fig:acg-comp-sem}, the abstract signature $\Sigma_{Synt}$ of
Figure~\ref{fig:acg-comp-basic} is split into two signatures,
$\Sigma_{Syntax}$ and $\Sigma_{SyntSem}$, in order to model syntactic
and semantic ambiguities in different grammars
\cite{pogodalla2007generalizing}.


\section{The Problem of Multiple Constraints}
\label{sec:constraints}

We will consider several linguistic constraints that have been given
formal treatments in grammatical formalisms.

In French, negation is signalled by prepending the particle \emph{ne} to the
negated verb in conjunction with using a properly placed accompanying word,
such as a negative determiner, in one of the verb's arguments. This phenomenon
has been elegantly formalized in the Frigram interaction grammar
\cite{perrier2007french}.

\begin{exe}
  \ex \label{ex:aucun-shallow} Jean \textbf{ne} parle à \textbf{aucun} collègue. \\
      (Jean speaks to no colleague.)
  \ex \label{ex:aucun-deep-obj} Jean \textbf{ne} parle à la femme d'\textbf{aucun} collègue. \\
      (Jean speaks to the wife of no colleague.)
  \ex \label{ex:aucun-deep-subj} \textbf{Aucun} collègue de Jean \textbf{ne} parle à sa femme. \\
      (No colleague of Jean's speaks to his wife.)
\end{exe}

We see here that the negative determiner \emph{aucun} can be present in
the subject or the object of the negated verb and it can modify the
argument directly or be attached to one of its complements. Furthermore,
we note that omitting either the word \emph{ne} or the word \emph{aucun}
while keeping the other produces a sentence which is considered
ungrammatical.

This difference in syntactic behavior between noun phrases that contain a
negative determiner and those that do not has implications for our
grammar. Since two terms that have an identical type in an ACG signature can
be freely interchanged in any sentence, we are forced to assign two different
types to these two different kinds of noun phrases.

% NOTE: We could use a single constant for the two paired words.
% NOTE: We could use a higher-order type for the negative determiner
% which demands the negative particle itself.

This leads us to a grammar in which we subdivide the atomic types $N$
and $NP$ into subtypes that reflect whether or not they contain a
negative determiner inside. Types of the other constants, such as the
preposition \emph{de} seen in (\ref{ex:aucun-deep-obj}) and
(\ref{ex:aucun-deep-subj}), will constrain their arguments to have
compatible features on their types and will propagate the information
carried in the features to its result type, e.g.:

\begin{align*}
N_{de_1} &: NP\_NEG{=}F \limp N\_NEG{=}F \limp N\_NEG{=}F \\
N_{de_2} &: NP\_NEG{=}F \limp N\_NEG{=}T \limp N\_NEG{=}T \\
N_{de_3} &: NP\_NEG{=}T \limp N\_NEG{=}F \limp N\_NEG{=}T
\end{align*}

Enforcing other constraints leads us to subdivide our ``atomic'' types
even further (e.g. the authors of \cite{pogodalla2012controlling} add
features to the $S$ and $NP$ types to implement constraints about
extraction). Other phenomena, such as agreement on morphological
properties like number, gender, person or case, intuitively lead us to
make our types even more specific.

If we were to use this approach to write a grammar that enforces multiple
constraints at the same time, we would end up with complicated types, like the
one below, which provide complete specifications of the various possible
situations.

\begin{align*}
C_{de_{11}} :\ &(NP\_\textcolor{red}{NEG{=}T}\_\textcolor{green}{VAR{=}F}\_\textcolor{blue}{NUM{=}PL}) \\
&\limp (N\_\textcolor{red}{NEG{=}F}\_\textcolor{blue}{NUM{=}SG}) \\
&\limp (N\_\textcolor{red}{NEG{=}T}\_\textcolor{blue}{NUM{=}SG})
\end{align*}

This creates two problems. Firstly, the number of such types grows
exponentially with the number of features added. This can be fixed by
introducing dependent types into the type system as in
\cite{de2007type}. However, while this allows us to abstract over the
combinations of feature values and write our grammar down concisely, it does
not take away the complexity. The treatments of the various linguistic
phenomena are all expressed in the same types making it hard to see whether
they are independent or not. Since the treatments cannot be considered in
isolation, reasoning about the entire grammar becomes difficult and so does
enhancing it with more constraints. This is a fatal problem for a grammar
which strives to cover a wide range of linguistic facts. We firmly believe
that simplicity is a fundamental requirement for constructing a large and
robust grammar and our proposal aims to reclaim that simplicity.

In our grammar, we would like to combine several constraints
(Figure~\ref{fig:acg-comp-constr}) and possibly to also separate the syntactic
ambiguities from the purely semantic ones (Figure~\ref{fig:acg-comp-sem}).
However, trying to mix these strategies in the ACG framework forces us to
solve all the constraints in a single type signature or contaminate the
syntax-semantics interface with the implementation details of the syntactic
layer, both of which introduce incidental complexity we want to avoid.

We would like to have a system which would be characterized by a diagram like
the one on Figure~\ref{fig:gacg} where the constraint signatures delimit the
legal syntactic structures independently of each other and without interfering
with the syntax-semantics interface. However, ACG diagrams are limited to
arborescences and we are obliged to generalize them in order to get
the expected interpretation of Figure~\ref{fig:gacg}.


\section{Graphical ACGs}

\begin{figure}
  \centering
  \includegraphics[width=0.25\textwidth,height=0.2\textheight]{../diagrams/final.pdf}
  \caption{\label{fig:gacg} A graphical ACG that implements two
      independent syntactic constraints and distinguishes syntactic and
      semantic ambiguities.}
\end{figure}

We define a \emph{graphical abstract categorial grammar} as a directed acyclic
graph (DAG) whose nodes are labeled with signatures (and distinguished types)
and whose edges are labeled with lexicons, in other words a mathematical
reification of an ACG diagram that has been generalized from
arborescences to DAGs. We then search for an appropriate semantics for these
structures, i.e. how to determine what languages are defined by a graphical
ACG.

\subsection{Nodes as Languages}

We first follow a paradigm in which nodes of the diagrams are
interpreted as languages with the edges telling us how these languages
are defined in terms of each other. A single arrow leading to a language
means that the target language is produced from the source by mapping it
through a lexicon. We then argue that the suitable meaning of two or
more edges arriving at the same node is intersection of languages based
both on the simplicity of the resulting definitions and on our
expectations about the desired semantics. This leads us to the following
definitions of \emph{intrinsic} and \emph{extrinsic} languages
associated with some node $v$ in a graphical ACG $\mathcal{G}$:

$$
\mathcal{I}_{\mathcal{G}}(v) = \{t \in \Lambda(\Sigma_v)
\mid\ \vdash_{\Sigma_v} t : S_v\}
$$
$$
\mathcal{E}_{\mathcal{G}}(v) = \mathcal{I}_{\mathcal{G}}(v) \cap
\bigcap_{(u,v) \in E} \mathcal{L}_{(u,v)}(\mathcal{E}_{\mathcal{G}}(u))
$$

The intrinsic language is just the set of terms built on the node's
signature and having the node's distinguished type. The extrinsic
language is established by taking the extrinsic languages of its
predecessors, mapping them through lexicons and taking their
intersection, or just taking the node's intrinsic language if it has no
predecessors.

We then examine the relationship between the languages defined by ACGs
and graphical ACGs (G-ACGs). Intrinsic languages correspond exactly to
abstract languages and therefore the sets of languages definable by both
are equal.

$$
\mathcal{I} = \mathcal{A}
$$

G-ACG extrinsic languages correspond to ACG object languages with
intersection. More formally, while ACG object languages are ACG abstract
languages closed on transformation by a lexicon, G-ACG extrinsic languages are
ACG abstract languages closed on transformation by a lexicon and intersection.

$$
\mathcal{O} = \mathcal{A}^{\mathcal{L}}
$$
$$
\mathcal{E} = \mathcal{A}^{\mathcal{L}{\cap}}
$$

We then argue that closure on intersection is an intuitive property of a
grammatical formalism that allows independent combination of linguistic
constraints. Furthermore, from the above we can see that object languages are
as expressive as extrinsic languages iff object languages are closed on
intersection, which is conjectured to be false.

However, the extrinsic languages we have defined above turn out to give
counter-intuitive interpretations to some graphical ACGs and we revisit the
question in an another paradigm.

\subsection{Nodes as Terms}

In the new paradigm, we interpret the nodes of the graph as terms and the
edges as statements that one term is mapped into another using a lexicon. This
leads us to the definition of the \emph{pangraphical}\footnote{As opposed to
  extrinsic languages, which are constrained only by their predecessors in the
  graph, pangraphical languages are constrained by the entire graph.} language
of a node $u$ in a G-ACG $\mathcal{G}$.

A term $t$ belongs to $\mathcal{P}_{\mathcal{G}}(u)$ whenever there
exists a labeling $T$ of the nodes of the graph such that:

\begin{itemize}
  \item $T_u = t$.
  \item For all $v \in V(G)$, $\vdash_{\Sigma_v} T_v : S_v$.
  \item For all $(v,w) \in E(G)$, $\mathcal{L}_{(v,w)}(T_v) = T_w$.
\end{itemize}

The two paradigms under which we can interpret ACGs are equivalent in the
context of plain ACGs whose diagrams are arborescent and hence the two
metaphors (nodes as languages and nodes as terms) can be used
interchangeably. We back this up formally by showing that for every node $u$
in every arborescent G-ACG $\mathcal{G}$, we have
$\mathcal{E}_{\mathcal{G}}(u) = \mathcal{P}_{\mathcal{G}}(u)$. However, as we
start to consider non-arborescent graphs, we find, interestingly, that the two
paradigms diverge (i.e. $\exists \mathcal{G}, u.\ \mathcal{E}_{\mathcal{G}}(u)
\neq \mathcal{P}_{\mathcal{G}}(u)$).

The newly defined pangraphical languages solve the problem of extrinsic
languages giving us counter-intuitive interpretations for some specific
G-ACGs. We show by construction that pangraphical languages are as
expressive as extrinsic languages and so the kinds of languages we have
defined here are increasingly more expressive, $\mathcal{I} \subseteq
\mathcal{E} \subseteq \mathcal{P}$. We also show that
$\mathcal{I}_{\mathcal{G}}(u) \supseteq \mathcal{E}_{\mathcal{G}}(u)
\supseteq \mathcal{P}_{\mathcal{G}}(u)$, meaning that the language
definitions are more and more constraining and that the extrinsic
languages went in the same direction as pangraphical languages when
narrowing the elements of $\mathcal{I}_{\mathcal{G}}(u)$, but not as
far.

Finally, we construct a G-ACG which integrates the French negation constraint
discussed in Section~\ref{sec:constraints}, the constraints on extraction
introduced in \cite{pogodalla2012controlling} and a constraint handling
agreement in a single grammar specification. The constraints are implemented
exactly as they were originally described. The underlying syntactic signature
is defined without any concern for the additional constraints and the
syntax-semantics interface is kept simple as well.


\section{Conclusion}

We have considered the problem of building a wide-coverage ACG, specifically
the problem of expressing a multitude of linguistic constraints. We have
examined previous techniques and found no satisfying solution. We have thus
provided an extension of the ACG formalism to solve the problem and justified
the need for the increased expressivity. This embedding of syntactic
constraints will be used to define a syntax-semantics interface and then to
build discourse structures.

In the end, this lets us define the syntax in a clean way using the idiomatic
style of categorial grammars (simple atomic types like $N$, $NP$ and $S$) and
then define the constraints themselves the same way as they are defined in
papers which try to formalize them individually (such as is the case with
\cite{pogodalla2012controlling}).

\bibliographystyle{plain}
\bibliography{../biblio}

\end{document}
