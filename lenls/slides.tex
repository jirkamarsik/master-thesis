\documentclass{beamer}

\usepackage[utf8]{inputenc}
\usepackage{xcolor}

\definecolor{darkgreen}{rgb}{0., 0.6, 0.}
\definecolor{darkgold}{rgb}{8., 0.6, 0.}
\definecolor{limegreen}{RGB}{175, 255, 175}
\definecolor{limeblue}{RGB}{200, 200, 235}

\hypersetup{pdfstartview={Fit}}
\def\limp {\mathbin{{-}\mkern-3.5mu{\circ}}}

\AtBeginSection[]
{
  \begin{frame}
    \frametitle{Outline}
    \tableofcontents[currentsection]
    \addtocounter{framenumber}{-1}
  \end{frame}
}

\setbeamertemplate{navigation symbols}{}
\setbeamertemplate{footline}
  {\hfill {\normalsize \insertframenumber{}/\inserttotalframenumber{}}}


\begin{document}

\title[G-ACGs]{Integration of Multiple Constraints in ACG}
\subtitle{Graphical Abstract Categorial Grammars}
%
\author{Jiří Maršík \and Maxime Amblard}
%
\institute{Université de Lorraine, Laboratoire lorrain de recherche en informatique et ses applications, UMR 7503, Vandoeuvre-lès-Nancy, 54500, France\\
INRIA, Villers-lès-Nancy, 54600, France\\
CNRS, Loria, Vandoeuvre-lès-Nancy, UMR 7503, 54500, France\\
\texttt{\{jiri.marsik, maxime.amblard\}@loria.fr}}

\date[October 2013]{October 28, 2013}

\frame{\titlepage \setcounter{framenumber}{1}}

\begin{frame}
\frametitle{Outline}
\tableofcontents
\end{frame}

\section{Introduction}

\begin{frame}
  \frametitle{Introduction}

  \begin{itemize}
    \item Abstract Categorial Grammars provide interesting tools for managing
      discourse and dynamic semantics. \vfill
    \item Refining grammars to enforce numerous constraints breeds complexity.
      \vfill
    \item We formalize an extension of ACGs which lets us use the ACG
      apparatus and maintain independent constraints. \vfill
    \item Our goal is a large-scale grammar usable for discussing discourse.
  \end{itemize}
\end{frame}


\section{ACG Preliminaries}

\newcommand{\synt}[1]{C_{\textrm{#1}}}

\begin{frame}
  \begin{block}{Abstract languages}
    Lambda terms constrained by the types of constants. \\ Defined by a
    \textcolor{red}{signature} and a \textcolor{darkgreen}{distinguished
      type}.

    \begin{columns}[t]
      \begin{column}{0.5\textwidth}
        \begin{align*}
          \synt{tatous} &: \textcolor{red}{N} \\
          \synt{Laura} &: \textcolor{red}{NP}
        \end{align*}
      \end{column}
      \begin{column}{0.5\textwidth}
        \begin{align*}
          \synt{les} &: \textcolor{red}{N \limp NP} \\
          \synt{aime} &: \textcolor{red}{NP \limp NP \limp S}
        \end{align*}
      \end{column}
    \end{columns}

    $$\synt{aime}\ \synt{Laura}\ (\synt{les}\ \synt{tatous}) :
    \textcolor{darkgreen}{S}$$
  \end{block}

  \begin{block}{Object languages}
    Languages of (object) terms which are obtained by replacing the
    constants in an existing (abstract) term using a \textcolor{darkgold}{lexicon}.

    \begin{columns}[t]
      \begin{column}{0.5\textwidth}
        \begin{align*}
          \mathcal{L}(\synt{tatous}) &= \textcolor{darkgold}{tatous} \\
          \mathcal{L}(\synt{Laura}) &= \textcolor{darkgold}{Laura}
        \end{align*}
      \end{column}
      \begin{column}{0.5\textwidth}
        \begin{align*}
          \mathcal{L}(\synt{les}) &= \textcolor{darkgold}{\lambda^{\circ} x.\ les + x} \\
          \mathcal{L}(\synt{aime}) &= \textcolor{darkgold}{\lambda^{\circ} x y.\ x + aime + y}
        \end{align*}
      \end{column}
    \end{columns}

    $$\mathcal{L}(\synt{aime}\ \synt{Laura}\ (\synt{les}\ \synt{tatous}))
    =_{\beta} Laura + aime + les + tatous$$
  \end{block}
\end{frame}


\begin{frame}
  \frametitle{ACG Diagrams}

  \begin{columns}[c]
    \begin{column}{0.5\textwidth}
      \begin{itemize}
      \item Two different lexicons interpret the same abstract terms.
      \item Instead of referring to the two ACGs $\mathopen{<}
        \Sigma_{Synt}, \Sigma_{String}, \mathcal{L}_{syntax}, S
        \mathclose{>}$ and $\mathopen{<} \Sigma_{Synt}, \Sigma_{Sem},
        \mathcal{L}_{sem}, S \mathclose{>}$, refer to this single
        diagram.
      \end{itemize}
    \end{column}
    \begin{column}{0.5\textwidth}
      \includegraphics[width=\textwidth]{../diagrams/double-acg.pdf}
    \end{column}
  \end{columns}
\end{frame}


\begin{frame}
  \frametitle{ACG Patterns}

  \begin{columns}[c]
    \begin{column}{0.5\textwidth}
      \begin{block}{Semantic ambiguities}
        \vspace{2 mm}
        \begin{center}
          \includegraphics[height=0.7\textheight]{../diagrams/parallel-over-serial.pdf}
        \end{center}
      \end{block}
    \end{column}
    \begin{column}{0.5\textwidth}
      \begin{block}{Syntactic constraint}
        \vspace{2 mm}
        \begin{center}
          \includegraphics[height=0.7\textheight]{../diagrams/serial-over-parallel.pdf}
        \end{center}
      \end{block}
    \end{column}
  \end{columns}
\end{frame}


\section{Why Graphical ACGs?}

\begin{frame}
  \frametitle{Composing ACG Patterns I}

  ACG diagrams are arborescences.

  \begin{columns}[c]
    \begin{column}{0.25\textwidth}
      {\small Syntax-semantics interface deals with syntactic
        constraints.}

      \onslide<2->{$\implies$ Complex.}
    \end{column}
    \begin{column}{0.25\textwidth}
      \begin{center}
        \includegraphics[height=0.7\textheight]{../diagrams/fail1.pdf}
      \end{center}
    \end{column}
    \begin{column}{0.25\textwidth}
      {\small Syntactic constraints deal with the syntax-semantics
        interface.}

      \onslide<3->{$\implies$ Complex.}
    \end{column}
    \begin{column}{0.25\textwidth}
      \begin{center}
        \includegraphics[height=0.7\textheight]{../diagrams/fail2.pdf}
      \end{center}
    \end{column}
  \end{columns}
\end{frame}


\begin{frame}
  \frametitle{Composing ACG Patterns II}

  Writing a simple wide-coverage grammar implies specifying constraints
  independently.

  \begin{columns}[c]
    \begin{column}{0.25\textwidth}
      Second constraint reimplements first constraint.

      \onslide<2->{$\implies$ Ineffective.}
    \end{column}
    \begin{column}{0.25\textwidth}
      \begin{center}
        \includegraphics[height=0.7\textheight]{../diagrams/constr-fail-1.pdf}
      \end{center}
    \end{column}
    \begin{column}{0.25\textwidth}
      Third constraint reimplements first and second constraint.

      \onslide<3->{$\implies$ Ineffective.}
    \end{column}
    \begin{column}{0.25\textwidth}
      \begin{center}
        \includegraphics[height=0.7\textheight]{../diagrams/constr-fail-2.pdf}
      \end{center}
    \end{column}
  \end{columns}
\end{frame}


\begin{frame}
  \frametitle{Multiple Constraints}

  Implementing multiple constraints in a single signature is
  \textbf{difficult}.

  Agreement:
    \begin{align*}
      A_{aime_1} &: NP_{\textcolor{darkgold}{NUM{=}SG}} \limp NP_{\textcolor{darkgold}{NUM{=}SG}} \limp S \\
      A_{aime_2} &: NP_{\textcolor{darkgold}{NUM{=}SG}} \limp NP_{\textcolor{darkgold}{NUM{=}PL}} \limp S
    \end{align*}

  Extraction:
    \begin{align*}
      E_{aime_1} &: NP_{\textcolor{darkgreen}{VAR{=}F}} \limp NP_{\textcolor{darkgreen}{VAR{=}F}} \limp S_{\textcolor{darkgreen}{EXT{=}NO}} \\
      E_{aime_2} &: NP_{\textcolor{darkgreen}{VAR{=}T}} \limp NP_{\textcolor{darkgreen}{VAR{=}F}} \limp S_{\textcolor{darkgreen}{EXT{=}ROOT}} \\
      E_{aime_3} &: NP_{\textcolor{darkgreen}{VAR{=}F}} \limp NP_{\textcolor{darkgreen}{VAR{=}T}} \limp S_{\textcolor{darkgreen}{EXT{=}ROOT}
}    \end{align*}

  Combined:
  \begin{small}
  \begin{align*}
      C_{aime_1} &: NP_{\textcolor{darkgold}{NUM{=}SG},\textcolor{darkgreen}{VAR{=}F}} \limp NP_{\textcolor{darkgold}{NUM{=}SG},\textcolor{darkgreen}{VAR{=}F}} \limp S_{\textcolor{darkgreen}{EXT{=}NO}} \\
      C_{aime_2} &: NP_{\textcolor{darkgold}{NUM{=}SG},\textcolor{darkgreen}{VAR{=}T}} \limp NP_{\textcolor{darkgold}{NUM{=}SG},\textcolor{darkgreen}{VAR{=}F}} \limp S_{\textcolor{darkgreen}{EXT{=}ROOT}} \\
      C_{aime_3} &: NP_{\textcolor{darkgold}{NUM{=}SG},\textcolor{darkgreen}{VAR{=}F}} \limp NP_{\textcolor{darkgold}{NUM{=}SG},\textcolor{darkgreen}{VAR{=}T}} \limp S_{\textcolor{darkgreen}{EXT{=}ROOT}} \\
      C_{aime_4} &: NP_{\textcolor{darkgold}{NUM{=}SG},\textcolor{darkgreen}{VAR{=}F}} \limp NP_{\textcolor{darkgold}{NUM{=}PL},\textcolor{darkgreen}{VAR{=}F}} \limp S_{\textcolor{darkgreen}{EXT{=}NO}} \\
      C_{aime_5} &: NP_{\textcolor{darkgold}{NUM{=}SG},\textcolor{darkgreen}{VAR{=}T}} \limp NP_{\textcolor{darkgold}{NUM{=}PL},\textcolor{darkgreen}{VAR{=}F}} \limp S_{\textcolor{darkgreen}{EXT{=}ROOT}} \\
      C_{aime_6} &: NP_{\textcolor{darkgold}{NUM{=}SG},\textcolor{darkgreen}{VAR{=}F}} \limp NP_{\textcolor{darkgold}{NUM{=}PL},\textcolor{darkgreen}{VAR{=}T}} \limp S_{\textcolor{darkgreen}{EXT{=}ROOT}
}  \end{align*}
  \end{small}
\end{frame}


\begin{frame}
  \frametitle{Graphical Abstract Categorial Grammars (G-ACGs)}

  \begin{columns}[c]
    \begin{column}{0.6\textwidth}
      \begin{center}
        \includegraphics[height=0.7\textheight]{../diagrams/final.pdf}
      \end{center}
    \end{column}

    \pause

    \begin{column}{0.4\textwidth}
        \begin{itemize}
        \item A DAG.
        \item Nodes labelled with:
          \begin{itemize}
          \item Signatures.
          \item Distinguished types.
          \end{itemize}
        \item Edges labelled with:
          \begin{itemize}
          \item Lexicons.
          \end{itemize}
        \end{itemize}
    \end{column}
  \end{columns}
\end{frame}




\section{Semantics of G-ACG Nodes}

\begin{frame}
  \frametitle{Two Intuitions for G-ACGs}

  \setbeamercolor{block title}{bg=limeblue}
  \setbeamercolor{block body}{bg=limeblue}
  \begin{block}{Algebraic Perspective}
    \begin{itemize}
    \item Nodes as \emph{languages}.
    \item Edges define \emph{languages} in terms of other
      \emph{languages} using lexicons and intersections.
    \end{itemize}
  \end{block}
  \setbeamercolor{block title}{bg=limegreen}
  \setbeamercolor{block body}{bg=limegreen}
  \begin{block}{Generative Perspective}
    \begin{itemize}
    \item Nodes as \emph{terms}.
    \item Edges describe how \emph{terms} are obtained from other
      \emph{terms} by applying lexicons.
    \end{itemize}
  \end{block}
\end{frame}


\begin{frame}
  \frametitle{Algebraic Intuition}

  \begin{columns}[c]
    \begin{column}{0.7\textwidth}
      \setbeamercolor{block title}{bg=limeblue}
      \setbeamercolor{block body}{bg=limeblue}
      \begin{block}{}
        \begin{itemize}
        \item Every node is a language.
        \item An arrow means that a language is produced from another by
          mapping through a \textcolor{darkgreen}{lexicon}.
        \item Two arrows converging on a node mean an
          \textcolor{red}{intersection} of languages.
        \end{itemize}
      \end{block}

$$
\mathcal{I}_{\mathcal{G}}(v) = \{t \in \Lambda(\Sigma_v)
\mid\ \vdash_{\Sigma_v} t : S_v\}
$$
$$
\mathcal{E}_{\mathcal{G}}(v) = \mathcal{I}_{\mathcal{G}}(v) \cap
\textcolor{red}{\bigcap_{(u,v) \in E}}
\textcolor{darkgreen}{\mathcal{L}_{(u,v)}}(\mathcal{E}_{\mathcal{G}}(u)).
$$
    \end{column}
    \begin{column}{0.3\textwidth}
      \includegraphics[width=\textwidth]{../diagrams/algebraic-perspective.pdf}
    \end{column}
  \end{columns}

\end{frame}


\begin{frame}
  \frametitle{Two Problems with the Algebraic View}

  \begin{columns}[t]
    \begin{column}{0.5\textwidth}
      \begin{center}
        \includegraphics[height=0.6\textheight]{../diagrams/both-patterns.pdf}
      \end{center}

      How do we get the language of meanings expressible only by
      syntactically correct sentences?
    \end{column}
    \begin{column}{0.5\textwidth}
      \begin{center}
        \includegraphics[height=0.6\textheight]{../diagrams/diamond-grammar.pdf}
      \end{center}

      How do we express the language of $Bottom$ terms whose $Left$ and
      $Right$ antecedent are generated by the same $Top$ antecedent?
    \end{column}
  \end{columns}
\end{frame}


\begin{frame}
  \frametitle{Generative Intuition}

  \begin{columns}[c]
    \begin{column}{0.7\textwidth}
\setbeamercolor{block title}{bg=limegreen}
      \setbeamercolor{block body}{bg=limegreen}
      \begin{block}{}
        \begin{itemize}
        \item Every node is a \textcolor{red}{term}.
        \item An arrow means that a term is produced from another by
          applying a \textcolor{darkgreen}{lexicon}.
        \item Two arrows converging on a node mean that the two (abstract)
          terms yield \textcolor{red}{the same} (object) term.
        \end{itemize}
      \end{block}

      \vspace{3 mm}

      $t$ belongs to $\mathcal{P}_{\mathcal{G}}(u)$ iff \\$\exists T :
      \textcolor{red}{V(G) \to \Lambda}$ such that:
      \begin{itemize}
      \item $T_u = t$.
      \item For all $v \in V(G)$, $\vdash_{\Sigma_v} T_v : S_v$.
      \item For all $(v,w) \in E(G)$,
        $\textcolor{darkgreen}{\mathcal{L}_{(v,w)}(T_v) = T_w}$.
      \end{itemize}
    \end{column}
    \begin{column}{0.3\textwidth}
      \includegraphics[width=\textwidth]{../diagrams/generative-perspective.pdf}
    \end{column}
  \end{columns}
\end{frame}


\begin{frame}
  \frametitle{Algebraic and Generative Relations}

  \begin{itemize}
  \item $\forall \, \mathcal{G}, u$ where $\mathcal{G}$ is arborescent:
    $$\mathcal{E}_{\mathcal{G}}(u) = \mathcal{P}_{\mathcal{G}}(u)
    = \mathcal{O}_{\mathcal{G}}(u)$$
    \pause
  \item However, $\exists \, \mathcal{G}, u$ such that
    $$\mathcal{E}_{\mathcal{G}}(u) \neq \mathcal{P}_{\mathcal{G}}(u)$$
    \vfill
    \pause
  \item[$\Rightarrow$] They are not interchangeable points of view.
  \end{itemize}
\end{frame}


\begin{frame}
  \frametitle{Properties of G-ACG languages}

  \begin{itemize}
  \item The three language families we defined above are\ldots
    \begin{itemize}
    \item {\ldots}increasingly constraining, $\forall \, \mathcal{G},
      u. \; \mathcal{I}_{\mathcal{G}}(u) \supseteq
      \mathcal{E}_{\mathcal{G}}(u) \supseteq
      \mathcal{P}_{\mathcal{G}}(u)$.
    \item {\ldots}increasingly expressive, $\mathcal{I} \subseteq
      \mathcal{E} \subseteq \mathcal{P}$.
    \end{itemize}
  \vfill
  \item Intrinsic languages are just the abstract languages of ACGs.
    $$\mathcal{I} = \mathcal{A}$$
  \item Extrinsic languages are simple extensions of ACGs that add
    intersections.
    $$\mathcal{O} = \mathcal{A}^{\mathcal{L}}$$
    $$\mathcal{E} = \mathcal{A}^{\mathcal{L}{\cap}}$$
  \vfill
  \item Pangraphical languages are faithful to the generative
    perspective and subsume the previous classes of languages.
    \begin{itemize}
    \item They solve both the example problems we saw before
      (\emph{meanings of grammatical sentences} and the \emph{diamond}).
    \end{itemize}
  \end{itemize}
\end{frame}


\section{Illustration}

\begin{frame}
  \frametitle{Basic Setup}
  \begin{center}
    \includegraphics[height=0.8\textheight]{../diagrams/illustration0.pdf}
  \end{center}
\end{frame}

\begin{frame}
  \frametitle{Single Constraint}
  \begin{center}
    \includegraphics[height=0.8\textheight]{../diagrams/illustration1.pdf}
  \end{center}
\end{frame}

\begin{frame}
  \frametitle{Multiple Constraints}
  \begin{center}
    \includegraphics[height=0.8\textheight]{../diagrams/illustration2.pdf}
  \end{center}
\end{frame}

\begin{frame}
  \frametitle{Exposing Another Signature}
  \begin{center}
    \includegraphics[height=0.8\textheight]{../diagrams/illustration3.pdf}
  \end{center}
\end{frame}

\begin{frame}
  \frametitle{Constraints on Multiple Signatures}
  \begin{center}
    \includegraphics[height=0.8\textheight]{../diagrams/illustration4.pdf}
  \end{center}
\end{frame}

\begin{frame}
  \frametitle{Reflection}
  \begin{columns}[c]
    \begin{column}{0.5\textwidth}
      \begin{center}
        \includegraphics[height=0.8\textheight]{../diagrams/illustration5.pdf}
      \end{center}
    \end{column}
    \begin{column}{0.5\textwidth}
      \begin{itemize}
        \item 9 signatures
        \item 8 lexicons
        \item Constraints decomplected from each other
        \item Constraints on different signatures
        \item High level of redundancy
      \end{itemize}
    \end{column}
  \end{columns}
\end{frame}


\section{Conclusion}

\begin{frame}
  \frametitle{Recapitulation}

  \begin{itemize}
  \item We reinterpreted ACGs by treating the diagrams as primary constructs
    instead of as artifacts.
  \vfill
  \pause
  \item This lets us handle non-arborescent diagrams in a principled manner.
  \vfill
  \pause
  \item Non-arborescence enables intersection and therefore conjunction of
    independent constraints.
  \vfill
  \pause
  \item We presented the kind of grammar we envisioned, built by collecting
    fragments described in existing research.
  \end{itemize}
\end{frame}


\begin{frame}
  \frametitle{Future Directions}

  \begin{itemize}
  \item Are extrinsic languages as expressive as pangraphical
    ones? $$\mathcal{P} \subseteq \mathcal{E}$$
  \item Are ACG object languages closed on
    intersection? $$\mathcal{O}^{\mathcal{L}{\cap}} \subseteq \mathcal{O}^{\mathcal{L}}$$
  \item Is there a metagrammatical approach that makes refining the types of a
    signature more concise?
  \vfill
  \item Is this enough to build a satisfactory grammar of a real language?
  \end{itemize}
\end{frame}

\end{document}
