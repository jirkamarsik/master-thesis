\section{Wide-coverage abstract categorial grammars}

As part of my internship, I will have to construct a wide-coverage
abstract categorial grammar of French. Abstract categorial grammars
are defined in terms of signatures, universes of typed constants, and
lexicons, which map elements of signatures to lambda terms in other
signatures. Our grammar will thus end up as an ensemble of signatures,
each containing a number of elements proportional to the size of our
dictionary, and some lexicons that will link the signatures together.

From the above exposition, it should be obvious that giving the
grammar in its literal form, i.e. listing the contents of all the
signatures and the mappings of the lexicons, would be unfeasible. The
approach that we have chosen is to employ a pre-existing database of
lexical items and describe how entries in this database generate the
elements of our signatures and the mappings in our lexicons.

The lexical database that we will be using is Frilex. Frilex was
itself constructed by collating several lexicographic sources and its
purpose was to serve as the lexical database for Frigram, a
wide-coverage interaction grammar of French, much like the way we will
use it to populate our grammar. In terms of structure, Frilex is a
relation between wordforms and hypertags, which are feature structures
giving the lemma, syntactic category, agreement, valency and other
pertinent morpho-syntactic features of the wordform.

\subsection{Defining signatures}

The task of the grammar author is then to describe for each signature
how the entries of the lexical database correspond to typed citizens
of the signature and analogously so for the lexicons. If we look at
how this correspondence should be modelled, we quickly arrive at the
conclusion that it cannot be a total function (e.g. in the case of a
signature of semantic predicates, we will most likely not want to have
predicates that correspond to determiners) nor in fact a partial
function (e.g. in the syntactic signature, we might very well desire
to have two different elements for transitive verbs depending on
whether they are to be interpreted with subject-wide or object-wide
scope). In the end, the correspondence seems to be most aptly
described as a relation. Therefore in our computational representation
of the grammar, the signature is defined via a relational (logic)
program. The relational programming system we have chosen is
miniKanren, particularly its Clojure implementation dubbed core.logic.

The next question we will have to answer is how exactly will this
signature-defining relation look like. To answer that, we first have
to be more precise about how we want the elements of our signature
yielded by this relation to be structured. As was said before, we want
all the elements in our signatures to have a type. Furthermore, we
also need a way how to distinguish elements of the same type, so we
will need to provide each element with some kind of identifier. We
could state that these identifiers are completely arbitrary or up to
the grammar author to generate, but this would make specifying the
mappings of lexicons burdensome, so we will try to reach something
better.

For starters, we might consider the hypertag as a candidate for the
identifier. However, it is possible that our lexical database will
hold two entries with the same hypertag but different wordforms (e.g.
alternative spellings) and we would like to distinguish the two to be
able to produce either of the two spellings using a lexicon. So next,
we can think of the entire lexical entry, the wordform and its
hypertag, as an identifier. This breaks down in the case of transitive
verbs having two projections which differ in subject/object scope. To
resolve cases like these, we add a synthetic part to our identifier
which we call a specifier and which is the grammar author's duty to
assign.

In the end, the identifiers for the elements of our signatures are
composed of their ``naturally-occurring'' part (the wordform and the
hypertag of their lexical entry) and their synthetic part (the
author-defined specifier). To answer our question, the relation which
defines the elements of a signature is a 4-ary relation between
wordforms, hypertags, spec(ifier)s and types. This relation posits
some constraints on the wordform and hypertag and then licenses the
possible combinations of values for the spec and the type. Since the
definition of this relation takes place within a general purpose
programming language, the grammar author is able to define any
abstractions which facilitate his statement of the relation.

It might seem we have solved the problem of defining signatures, but
there is still one thing that we have to address. In many of our
signatures, we will desire to have some elements of the signature that
don't correspond to any lexical item but instead provide some kind of
furniture for working in the domain. In the signature of semantics,
these might be constants corresponding to the boolean operators and
quantifiers, whereas in the domain of strings, we might think of the
string concatenation operator. Therefore, in our formalism, we must
acknowledge that elements of signatures are not only generated by the
above-described lexical relation, but also by contain some set of
constant elements. These elements also have a type and an identifier,
however, in their case, the identifier is not composed of some lexical
entry, but is simply the symbol that was used by the grammar author to
name the constant.

\subsection{Defining lexicons}

Now that we have a way of giving a concise computational
representation of a wide-coverage signature, we turn our attention to
the lexicons, the mappings relating the signatures. A lexicon is
defined as a mapping from elements of one signature to terms over the
elements of another signature. In our formalism, we simply follow this
definition, though for ease of implementation, we do not implement the
function as a functional program, but as a relational program. So far,
it is up to the grammar author to make sure that his lexicon assigns
only one term to a single element of the signature.
