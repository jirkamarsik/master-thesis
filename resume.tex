\newpage

\pagenumbering{gobble}

\begin{description}[style=nextline]
\item[Résumé]

Le but final de notre travail est la création d'une grammaire
catégorielle abstraite (ACG) à large échelle qu'on pourra utiliser pour
construire les représentations au niveau de discours d'une manière
automatique.

On commence par examiner les ressources linguistiques qui existent déjà,
en particulier la grammaire d'interaction Frigram et son lexique Frilex,
et on examine leur utilité pour la construction d'une ACG à large
échelle. Ensuite, on présente une implémentation des mécanismes ACG qui
nous permet d'expérimenter avec des grammaires lexicalisées par
Frilex. Enfin, on considère les problèmes posés par le traitement des
plusieurs contraintes linguistiques dans une seule ACG et on propose une
généralisation du formalisme: les grammaires catégorielles abstraites
graphiques.

\item[Mots-clés]

Grammaires catégorielles abstraites (ACG), Génie des grammaires,
Grammaires d'interaction, Metagrammaires, Programmation relationnelle,
Formalismes grammaticaux, Grammaires formelles, Linguistique
computationnelle, Lambda-calcul, Théorie des types
\end{description}

\vfill

\begin{description}[style=nextline]
\item[Summary]

We present work whose ultimate goal is the creation of a wide-coverage
abstract categorial grammar (ACG) that could be used to automatically
build discourse-level representations.

We first examine existing language resources, in particular the Frigram
interaction grammar and its lexicon Frilex, and assess their utility to
building a wide-coverage ACG. We then present our implementation of the
ACG machinery which allows us to experiment with grammars lexicalized by
Frilex. Finally, we consider the challenge of integrating the treatment
of disparate linguistic constraints in a single ACG and propose a
generalization of the formalism: graphical abstract categorial
grammars.

\item[Keywords]

Abstract Categorial Grammars (ACG), Grammar Engineering, Interaction
Grammars, Metagrammars, Relational Programming, Grammatical Formalisms,
Formal Grammars, Computational Linguistics, Lambda Calculus, Type Theory
\end{description}
