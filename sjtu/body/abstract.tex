%%==================================================
%% abstract.tex for SJTU Master Thesis
%% based on CASthesis
%% modified by wei.jianwen@gmail.com
%% version: 0.3a
%% Encoding: UTF-8
%% last update: Dec 5th, 2010
%%==================================================

\begin{abstract}

本研究的最终目的在于创建一个具有广覆盖面的抽象范畴语法(Abstract Category Grammar - ACG)。该语法可以用于自动构建自然语言段落层的表达式。为达到这一目标,我们成功地对创建广覆盖面抽象范畴语法的必要基础条件进行了实现。

在该论文中,我们首先调查了现存的语言资源,尤其是基于交互语法的Frigram以及它的语料库:Frilex,并评估了其作为构建广覆盖面抽象范畴语法的实用性。然后,我们介绍了对于范畴语法功能的具体实现,这项实现可以使用经Frilex词汇化了的语法进行测试。再后,我们考虑尝试使用单一的抽象范畴语法来整合不同语言限定条件,并提出了之前系统框架的一个推广,即图形化抽样范畴语法。最后,我们探索了一些图形化抽样范畴语法的形式特性,以此作为论文的总结。

  \keywords{\large 抽样范畴语法 \quad 语法工程 \quad 交互语法 \quad 元语法
    \quad 关系化编程 \quad 语法形式 \quad 形式语法 \quad 计算语言 \quad Lambda
    算子 \quad 类型论}
\end{abstract}

\begin{englishabstract}

We present work whose ultimate goal is the creation of a wide-coverage
abstract categorial grammar (ACG) that could be used to automatically build
discourse-level representations. In our work, we advance towards that goal by
laying down the foundations necessary for building wide-coverage ACGs.

We first examine existing language resources, in particular the Frigram
interaction grammar and its lexicon Frilex, and assess their utility to
building a wide-coverage ACG. We then present our implementation of the ACG
machinery which allows us to experiment with grammars lexicalized by
Frilex. Finally, we consider the challenge of integrating the treatment of
disparate linguistic constraints in a single ACG and propose a generalization
of the formalism: graphical abstract categorial grammars. The report concludes
with an exploration of some of the formal properties of graphical ACGs.

  \englishkeywords{\large Abstract Categorial Grammars (ACG), Grammar
    Engineering, Interaction Grammars, Metagrammars, Relational Programming,
    Grammatical Formalisms, Formal Grammars, Computational Linguistics, Lambda
    Calculus, Type Theory}
\end{englishabstract}
