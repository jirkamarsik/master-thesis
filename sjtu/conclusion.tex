\chapter{In Conclusion}
\label{chap:conclusion}

In this final chapter, we first discuss some the consequences of using
G-ACGs for writing grammars. We then summarize the contributions that we
have put forward and suggest problems we believe to be interesting.

\section{On the Practical Consequences of Graphical ACGs}
\label{sec:practical}

We will enumerate some of the practical benefits and challenges of using
graphical ACGs for implementing grammar-based systems.

\begin{itemize}
\item The lexicons that go from our constraint signatures ($Neg$, $Ext$
  and $Agr$ in subsection \ref{ssec:graphical-engineering}) are all
  trivial in the sense they map constants to constants. Finding an
  antecedent to a term is then just a question of finding an antecedent
  for every constant in the term (i.e. ``tagging'' the constants of the
  term with antecedent constants). This means that there is an effective
  parsing algorithm for verifying the constraints. Furthermore, this
  algorithm could also be made efficient by introducing filtering
  techniques \cite{guillaume2008toolchain} and statistical supertagging
  \cite{moot2012categorial}.

\item Constraints can introduce type extensions in their abstract
  signatures and the added complexity of these extensions is then
  contained within the constraint. Furthermore, the constraint signature
  can be treated as a ``black box'' that verifies a constraint and in
  the context of an implementation, it can be replaced with a
  special-purpose hand-written procedure that does the same more
  efficiently.

\item The distributed nature of graphical ACGs has its own practical
  applications. Parsing can be made more robust by being able to parse
  sentences which violate some of the constraints. Parsing an
  ungrammatical sentence and identifying the constraints it violates can
  also serve as a basis for a syntax checker.

\item If we decide to define all of our syntactic constraints in
  separate signatures, we will end up with a lot of signatures which
  look exactly like the syntactic signature but with some small
  refinements (see the signatures in sections \ref{sec:negation},
  \ref{sec:extraction} and \ref{sec:agreement}). Thus we will desire
  some concise way of formulating these refinements so that we will not
  need to replicate the entire syntactic signature for every
  constraint.

  We hope that the class of useful refinements is small and that these
  constraint signatures could be systematically constructed from the
  syntactic signature by applying a series of refining
  transformations. This is an interesting engineering challenge awaiting
  anyone willing to design a truly wide-coverage graphical ACG using
  constraints.
\end{itemize}


\section{Conclusion}
\label{sec:conclusion}

The contributions of our work can be summarized in the following three
points:

\begin{enumerate}
\item We have highlighted the challenges inherent in translating an
  existing grammar in the formalism of IGs to the formalism of
  ACGs. Based on the breadth of different constraints built in to the IG
  formalism, we have conjectured that a direct translation of IG lexical
  items to ACG lexical items based on the deep underlying similarity
  between the two formalisms is not practical.

\item We have implemented a system for experimenting with (lexicalized)
  abstract categorial grammars that can serve as the basis for
  formulating wide-coverage graphical ACGs
  (\url{https://github.com/jirkamarsik/acg-clj}).

\item We have presented the graphical abstract categorial grammars
  (G-ACGs) which generalize ACGs. We have determined the set of
  languages definable by G-ACGs in terms of ACG languages and we have
  shown how G-ACGs allow us to write categorial grammars that handle
  multiple disparate linguistic constraints without sacrificing
  simplicity.
\end{enumerate}

Here are some of the items that we believe warrant future investigation:

\begin{itemize}
\item Try to express the multiple independent polarized features of IGs
  using G-ACGs.
\item Determine whether ACG object languages are closed on intersection
  or not.
\item Make a closer examination of the relationship between extrinsic
  and pangraphical languages in G-ACGs.
\item Write grammars using the G-ACG constraint architecture to verify
  its applicability and test the claims of modularity we have made here.
\item Study the problem of concisely defining constraint signatures from
  syntactic signatures and rules of feature propagation.
\end{itemize}
