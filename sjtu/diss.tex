%%==================================================
%% diss.tex for SJTU Master Thesis
%% based on CASthesis
%% modified by wei.jianwen@gmail.com
%% version: 0.3a
%% Encoding: UTF-8
%% last update: Dec 5th, 2010
%%==================================================

% 字号选项: c5size 五号(默认) cs4size 小四
% 双面打印(注意字号设置)
\documentclass[cs4size, a4paper, twoside]{sjtuthesis} 
% 单面打印(注意字号设置)
% \documentclass[cs4size, a4paper, oneside, openany]{sjtuthesis} 


% \usepackage[sectionbib]{chapterbib}%每章都用参考文献
\usepackage[utf8]{inputenc}
\usepackage[labelformat=simple]{subcaption}
\renewcommand\thesubfigure{(\alph{subfigure})}
\usepackage{amsmath}
\usepackage{amsthm}
\usepackage{xcolor}
\usepackage{graphicx}
\usepackage{gb4e}
\usepackage{bussproofs}
\usepackage{hyperref}
\usepackage{cancel}
\usepackage[nottoc,numbib]{tocbibind}
\usepackage{pdfpages}
\usepackage{enumitem}

\def\limp {\mathbin{{-}\mkern-3.5mu{\circ}}}

\newtheorem*{observation}{Observation}
\newtheorem*{theorem}{Theorem}
\newtheorem*{corollary}{Corollary}



\newboolean{DOIT}
\setboolean{DOIT}{false}%编译某些只想自己看的内容,编译true,否则false

%% 行距缩放因子(x倍字号)
\renewcommand{\baselinestretch}{1.3}

% 设置图形文件的搜索路径
\graphicspath{{figure/}{figures/}{logo/}{logos/}{graph/}{graphs}}

%%========================================
%% 在sjtuthesis.cls中定义的有用命令
%%========================================
% \cndash 中文破折号
% 数学常量
% \me 对数常数e
% \mi 虚数单位i
% \mj 虚数单位j
% \dif 直立的微分算符d为直立体。
% 可伸长的数学箭头、等号
% \myRightarrow{}{}
% \myLeftarrow{}{}
% \myBioarrow{}{}
% \myLongEqual{}{}
% 参考文献
% \upcite{} 上标引用
%%========================================


\begin{document}

%%%%%%%%%%%%%%%%%%%%%%%%%%%%%% 
%% 封面
%%%%%%%%%%%%%%%%%%%%%%%%%%%%%% 

% 中文封面内容(关注内容而不是形式)
\title{一个广覆盖面语法的构造: 图形化抽样范畴语法}
\author{Jiri Marsik}
\advisor{赵海}
\degree{硕士}
\defenddate{2013年11月26日}
\school{上海交通大学}
\institute{计算机科学与工程系}
\studentnumber{1110339127}
\major{EM LCT}

% 英文封面内容(关注内容而不是表现形式)
\englishtitle{Towards a Wide-Coverage Grammar: Graphical Abstract Categorial Grammars}
\englishauthor{\textsc{Jiri Marsik}}
\englishadvisor{MCF \textsc{Maxime Amblard} \\ A/Prof. \textsc{Zhao Hai}}
\englishschool{Shanghai Jiao Tong University}
\englishinstitute{\textsc{Department of Computer Science \& Engineering, Graduate School} \\
  \textsc{Shanghai Jiao Tong University} \\
  \textsc{Shanghai, P.R.China}}
\englishdegree{Master}
\englishmajor{Language and Communication Technologies}
\englishdate{Nov. 26th, 2013}

% 封面
\maketitle

% 英文封面
\makeenglishtitle

% 论文原创性声明和使用授权
\makeDeclareOriginal
\makeDeclareAuthorization

%%%%%%%%%%%%%%%%%%%%%%%%%%%%%% 
%% 前言
%%%%%%%%%%%%%%%%%%%%%%%%%%%%%% 
\frontmatter

% 摘要
\documentclass[twocolumn]{article}

\usepackage[utf8]{inputenc}
\usepackage{amsmath}
\usepackage{amsthm}
\usepackage{xcolor}
\usepackage{graphicx}
\usepackage{gb4e}
\usepackage{bussproofs}
\usepackage{hyperref}
\usepackage[labelformat=simple]{subcaption}
\renewcommand\thesubfigure{(\alph{subfigure})}
\usepackage{cancel}
\usepackage[nottoc,numbib]{tocbibind}
\usepackage{pdfpages}
\usepackage{enumitem}
\usepackage{titling}

\setlength{\droptitle}{-13em}

%% Linear implication from Alessio Guglielmi
%% https://groups.google.com/forum/?fromgroups=#!topic/comp.text.tex/0B4C3F_BsVI
\def\limp {\mathbin{{-}\mkern-3.5mu{\circ}}}

\newtheorem*{observation}{Observation}
\newtheorem*{theorem}{Theorem}
\newtheorem*{corollary}{Corollary}

\begin{document}

\title{An Integration of Multiple Constraints in ACG}
\author{Jiří Maršík \and Maxime Amblard}
\maketitle

\begin{abstract}
  We present work whose ultimate goal is the creation of a wide-coverage
  abstract categorial grammar (ACG) that could be used to automatically
  build discourse-level representations. We focus on the challenge of
  integrating the treatment of disparate linguistic constraints in a
  single ACG and propose a generalization of the formalism: graphical
  abstract categorial grammars.
\end{abstract}


\section{Motivation}

Abstract categorial grammars (ACGs) have shown to be a viable formalism
for elegantly encoding the dynamic nature of discourse. Proposals based
on continuation semantics \cite{de2006towards} have tackled topics such
as event anaphora \cite{qian2011event}, SDRT discourse structure
\cite{asher2011sdrt} and modal accessibility constraints
\cite{asher2011montagovian}. However, all of these treatments only
consider tiny fragments of languages. We are interested in building a
wide-coverage grammar which integrates and reconciles the existing
formal treatments of discourse and allows us to study their interactions
and to build discourse representations automatically.

In this paper we focus on the problem of constructing a wide-coverage
ACG. We identify a problem that comes up when one tries to enforce
multiple constraints/dependencies in a single ACG and we propose a
generalization of the formalism which allows for a natural solution to
the problem.


\begin{figure*}[t]
  \centering
  \begin{subfigure}[b]{0.25\textwidth}
    \centering
    \includegraphics[height=0.2\textheight]{../diagrams/double-acg.pdf}
    \caption{\label{fig:acg-comp-basic} Connecting form with meaning.}
  \end{subfigure}
  \qquad
  \begin{subfigure}[b]{0.25\textwidth}
    \centering
    \includegraphics[height=0.2\textheight]{../diagrams/serial-over-parallel.pdf}
    \caption{\label{fig:acg-comp-constr} Adding a constraint.}
  \end{subfigure}
  \qquad
  \begin{subfigure}[b]{0.25\textwidth}
    \centering
    \includegraphics[height=0.2\textheight]{../diagrams/parallel-over-serial.pdf}
    \caption{\label{fig:acg-comp-sem} Distinguishing syntactic and
      semantic ambiguities.}
  \end{subfigure}
  \caption{\label{fig:acg-comp} Diagrams of systems of ACGs.}
\end{figure*}


\section{Systems of ACGs}
\label{sec:acg-comp}

An ACG is defined in terms of two higher-order signatures, sets of typed
constants on which we can form well-typed lambda terms, and a
translating function from terms of one of the signatures to terms of the
other (the former signature being called the \emph{abstract signature},
the latter being called the \emph{object signature} and the function
being called a \emph{lexicon}). We are then interested in the terms of
the abstract signature having some distinguished type and their images
through the lexicon (these two sets of terms are called the
\emph{abstract language} and the \emph{object language}, respectively).

When describing languages, we are usually interested in systems of more
than one ACG and diagrams such as the ones on Figure~\ref{fig:acg-comp}
have been used to explain them.

In Figure~\ref{fig:acg-comp-basic}, abstract syntactic terms from the
signature $\Sigma_{Synt}$ yield phonological strings and semantic
representations using two different lexicons. In
Figure~\ref{fig:acg-comp-constr}, this pattern is extended by adding a
new abstract signature which narrows the language of syntactic terms and
implements some linguistic constraint
\cite{pogodalla2012controlling}. Finally, in
Figure~\ref{fig:acg-comp-sem}, the abstract signature $\Sigma_{Synt}$ of
Figure~\ref{fig:acg-comp-basic} is split into two signatures,
$\Sigma_{Syntax}$ and $\Sigma_{SyntSem}$, in order to model syntactic
and semantic ambiguities in different grammars
\cite{pogodalla2007generalizing}.


\section{The Problem of Multiple Constraints}
\label{sec:constraints}

We will consider several linguistic constraints that have been given
formal treatments in grammatical formalisms.

In French, negation is signalled by prepending the particle \emph{ne} to
the negated verb in conjunction with using a properly placed
accompanying word, such as a negative determiner in one of the verb's
arguments. This phenomenon has been elegantly formalized in the Frigram
interaction grammar \cite{perrier2007french}.

\begin{exe}
  \ex \label{ex:aucun-shallow} Jean \textbf{ne} parle à \textbf{aucun} collègue. \\
      (Jean speaks to no colleague.)
  \ex \label{ex:aucun-deep-obj} Jean \textbf{ne} parle à la femme d'\textbf{aucun} collègue. \\
      (Jean speaks to the wife of no colleague.)
  \ex \label{ex:aucun-deep-subj} \textbf{Aucun} collègue de Jean \textbf{ne} parle à sa femme. \\
      (No colleague of Jean's speaks to his wife.)
\end{exe}

We see here that the negative determiner \emph{aucun} can be present in
the subject or the object of the negated verb and it can modify the
argument directly or be attached to one of its complements. Furthermore,
we note that omitting either the word \emph{ne} or the word \emph{aucun}
while keeping the other produces a sentence which is considered
ungrammatical.

This difference in syntactic behavior between noun phrases that contain a
negative determiner and those that do not has implications for our
grammar. Since two terms that have an identical type in an ACG signature can
be freely interchanged in any sentence, we are forced to assign two different
types to these two different kinds of noun phrases.

% NOTE: We could use a single constant for the two paired words.
% NOTE: We could use a higher-order type for the negative determiner
% which demands the negative particle itself.

This leads us to a grammar in which we subdivide the atomic types $N$
and $NP$ into subtypes that reflect whether or not they contain a
negative determiner inside. Types of the other constants, such as the
preposition \emph{de} seen in (\ref{ex:aucun-deep-obj}) and
(\ref{ex:aucun-deep-subj}), will constrain their arguments to have
compatible features on their types and will propagate the information
carried in the features to its result type, e.g.:

\begin{align*}
N_{de_1} &: NP\_NEG{=}F \limp N\_NEG{=}F \limp N\_NEG{=}F \\
N_{de_2} &: NP\_NEG{=}F \limp N\_NEG{=}T \limp N\_NEG{=}T \\
N_{de_3} &: NP\_NEG{=}T \limp N\_NEG{=}F \limp N\_NEG{=}T
\end{align*}

Enforcing other constraints leads us to subdivide our ``atomic'' types
even further (e.g. the authors of \cite{pogodalla2012controlling} add
features to the $S$ and $NP$ types to implement constraints about
extraction). Other phenomena, such as agreement on morphological
properties like number, gender, person or case, intuitively lead us to
make our types even more specific.

If we were to use this approach to write a grammar that enforces multiple
constraints at the same time, we would end up with complicated types, like the
one below, which provide complete specifications of the various possible
situations.

\begin{align*}
C_{de_{11}} :\ &(NP\_\textcolor{red}{NEG{=}T}\_\textcolor{green}{VAR{=}F}\_\textcolor{blue}{NUM{=}PL}) \\
&\limp (N\_\textcolor{red}{NEG{=}F}\_\textcolor{blue}{NUM{=}SG}) \\
&\limp (N\_\textcolor{red}{NEG{=}T}\_\textcolor{blue}{NUM{=}SG})
\end{align*}

This creates two problems. Firstly, the number of such types grows
exponentially with the number of features added. This can be fixed by
introducing dependent types into the type system as in
\cite{de2007type}. However, secondly, and perhaps more importantly, we have
lost the clarity and simplicity of describing the individual phenomena
independently. This makes adding new constraints more and more cumbersome and
reduces our capability to reason about our grammar. It no longer becomes
apparent which language mechanisms behave independently and their interactions
are hard to track. We firmly believe that simplicity is a fundamental
requirement for constructing a large and robust grammar and our proposal aims
to reclaim that simplicity.

In our grammar, we would like to combine several constraints
(Figure~\ref{fig:acg-comp-constr}) and possibly to also separate the syntactic
ambiguities from the purely semantic ones (Figure~\ref{fig:acg-comp-sem}).
However, these combinations are not possible in the framework of ACGs without
introducing complexity by solving all the constraints in one single type
signature (the consequences of we saw above) or contaminating the
syntax-semantics interface by the implementation details of the syntactic
layer.

We would like to have a system which would be characterized by a diagram like
the one on Figure~\ref{fig:gacg}. However, ACG diagrams are limited to
arborescences and we are obliged to generalize the formalism in order to get
the expected interpretation of Figure~\ref{fig:gacg}.


\section{Graphical ACGs}

\begin{figure}
  \centering
  \includegraphics[width=0.25\textwidth,height=0.2\textheight]{../diagrams/final.pdf}
  \caption{{\label{fig:gacg} A graphical ACG that implements two
      independent syntactic constraints and distinguishes syntactic and
      semantic ambiguities.}}
\end{figure}

We define a \emph{graphical abstract categorial grammar} as a directed
acyclic graph (DAG) whose nodes are labeled with signatures (and
distinguished types) and whose edges are labeled with lexicons, in other
words just a mathematical reification of an ACG diagram that has also
been generalized from arborescences to DAGs. We then search for an
appropriate semantics for these structures, i.e. how to determine what
languages are defined by a graphical ACG.

We first follow a paradigm in which nodes of the diagrams are
interpreted as languages with the edges telling us how these languages
are defined in terms of each other. A single arrow leading to a language
means that the target language is produced from the source by mapping it
through a lexicon. We then argue that the suitable meaning of two or
more edges arriving at the same node is intersection of languages based
both on the simplicity of the resulting definitions and on our
expectations about the desired semantics. This leads us to the following
definitions of \emph{intrinsic} and \emph{extrinsic} languages
associated with some node $v$ in a graphical ACG $\mathcal{G}$:

$$
\mathcal{I}_{\mathcal{G}}(v) = \{t \in \Lambda(\Sigma_v)
\mid\ \vdash_{\Sigma_v} t : S_v\}
$$
$$
\mathcal{E}_{\mathcal{G}}(v) = \mathcal{I}_{\mathcal{G}}(v) \cap
\bigcap_{(u,v) \in E} \mathcal{L}_{(u,v)}(\mathcal{E}_{\mathcal{G}}(u))
$$

The intrinsic language is just the set of terms built on the node's
signature and having the node's distinguished type. The extrinsic
language is established by taking the extrinsic languages of its
predecessors, mapping them through lexicons and taking their
intersection, or just taking the node's intrinsic language if it has no
predecessors.

We then examine the relationship between the languages defined by ACGs
and graphical ACGs (G-ACGs). Intrinsic languages correspond exactly to
abstract languages and therefore the sets of languages definable by both
are equal.

$$
\mathcal{I} = \mathcal{A}
$$

G-ACG extrinsic languages correspond to ACG object languages with
intersection. More formally, it can be said that ACG object languages are just
ACG abstract languages closed on transformation by a lexicon, and we then show
that G-ACG extrinsic languages are just ACG abstract languages closed on
transformation by a lexicon and intersection.

$$
\mathcal{O} = \mathcal{A}^{\mathcal{L}}
$$
$$
\mathcal{E} = \mathcal{A}^{\mathcal{L}{\cap}}
$$

We then argue that closure on intersection is an intuitive property of a
grammatical formalism that allows independent combination of linguistic
constraints. Furthermore, from the above we can see that object
languages are as expressive as extrinsic languages iff object languages
are closed on intersection, which is conjectured to be false.

However, the extrinsic languages we have defined above turn out to give
counter-intuitive interpretations to some graphical ACGs and we revisit the
question in an another paradigm.

In the new paradigm, we interpret the nodes of the graph as terms and the
edges as statements that one term is mapped into another using a lexicon. This
leads us to the definition of the \emph{pangraphical} language of a node $u$
in a G-ACG $\mathcal{G}$.

A term $t$ belongs to $\mathcal{P}_{\mathcal{G}}(u)$ whenever there
exists a labeling $T$ of the nodes of the graph such that:

\begin{itemize}
  \item $T_u = t$.
  \item For all $v \in V(G)$, $\vdash_{\Sigma_v} T_v : S_v$.
  \item For all $(v,w) \in E(G)$, $\mathcal{L}_{(v,w)}(T_v) = T_w$.
\end{itemize}

The two paradigms under which we can interpret ACGs are equivalent in
the context of plain ACGs whose diagrams are arborescent and the two
metaphors (nodes as languages and nodes as terms) can be used
interchangeably. We back this up formally by showing that for every node
$u$ in every arborescent G-ACG $\mathcal{G}$, we have
$\mathcal{E}_{\mathcal{G}}(u) = \mathcal{P}_{\mathcal{G}}(u)$. However,
as we start to consider non-arborescent graphs, we find, interestingly,
that the two paradigms diverge (i.e. $\exists \mathcal{G},
u.\ \mathcal{E}_{\mathcal{G}}(u) \neq \mathcal{P}_{\mathcal{G}}(u)$).

The newly defined pangraphical languages solve the problem of extrinsic
languages giving us counter-intuitive interpretations for some specific
G-ACGs. We show by construction that pangraphical languages are as
expressive as extrinsic languages and so the kinds of languages we have
defined here are increasingly more expressive, $\mathcal{I} \subseteq
\mathcal{E} \subseteq \mathcal{P}$. We also show that
$\mathcal{I}_{\mathcal{G}}(u) \supseteq \mathcal{E}_{\mathcal{G}}(u)
\supseteq \mathcal{P}_{\mathcal{G}}(u)$, meaning that the language
definitions are more and more constraining and that the extrinsic
languages went in the same direction as pangraphical languages when
narrowing the elements of $\mathcal{I}_{\mathcal{G}}(u)$, but not as
far.

Finally, we construct a G-ACG which integrates the French negation
constraint discussed in Section~\ref{sec:constraints}, the constraints
on extraction introduced in \cite{pogodalla2012controlling} and a
constraint handling agreement in a single grammar specification. The
negation constraint is implemented exactly as was described in
Section~\ref{sec:constraints} and the extraction constraints are taken
``as is'' from \cite{pogodalla2012controlling}. The underlying syntactic
signature is defined without any concern for the additional constraints
and the syntax-semantics interface is kept simple as well.

In the end, this lets us define the syntax in a clean way using the
idiomatic style of categorial grammars (simple atomic types like $N$,
$NP$ and $S$) and then define the constraints themselves the same way as
they are defined in papers which try to formalize them individually
(such as the case with \cite{pogodalla2012controlling}). We can also
freely combine both the constraint pattern and the syntax-semantics
ambiguities pattern from Section~\ref{sec:acg-comp}.


\section{Conclusion}

We have considered the problem of building a wide-coverage ACG, specifically
the problem of expressing a multitude of linguistic constraints. We have
examined previous techniques and found no satisfying solution. We have thus
provided an extension of the ACG formalism to solve the problem and justified
the need for the increased expressivity.

\bibliographystyle{plain}
\bibliography{../biblio}

\end{document}


% 目录
\tableofcontents
% 插图索引
\listoffigures
\addcontentsline{toc}{chapter}{\listfigurename} %将图索引加入全文目录
% 表格索引
%\listoftables
%\addcontentsline{toc}{chapter}{\listtablename}  %将表格索引加入全文目录

% 主要符号、缩略词对照表
%%==================================================
%% symbol.tex for SJTU Master Thesis
%% based on CASthesis
%% modified by wei.jianwen@gmail.com
%% version: 0.3a
%% Encoding: UTF-8
%% last update: Dec 5th, 2010
%%==================================================

\chapter{主要符号对照表}
\label{chap:symb}
\begin{tabular}{ll}

 \hspace{2em}$\limp$       & \hspace{5em}linear implication \\
 \hspace{2em}$\lambda^{\circ}$ & \hspace{5em}linear abstraction \\

\end{tabular}


%%%%%%%%%%%%%%%%%%%%%%%%%%%%%% 
%% 正文
%%%%%%%%%%%%%%%%%%%%%%%%%%%%%% 
\mainmatter

\chapter{Preliminaries}

%% 各章正文内容
\subsection{Motivation}

To develop applications capable of understanding natural language, we
have to analyze the discourse as a complex structure. The structure of
the discourse and the rhetorical relations between its constituent
propositions have shown to have an important effect on anaphora and on
the semantic content of the discourse.

Let us consider these examples from \cite{asher2003logics}.

\begin{exe}
  \ex \label{narxexp-nar} \begin{xlist}
    \ex \label{narxexp-nar-a} Max fell.
    \ex \label{narxexp-nar-b} John helped him up.
  \end{xlist}
  \ex \label{narxexp-exp} \begin{xlist}
    \ex \label{narxexp-exp-a} Max fell.
    \ex \label{narxexp-exp-b} John pushed him.
  \end{xlist}
\end{exe}

In (\ref{narxexp-nar}), we intuitively recognize that the second
proposition (\ref{narxexp-nar-b}) serves as a narrative continuation of
(\ref{narxexp-nar-a}), whereas in (\ref{narxexp-exp}), the proposition
(\ref{narxexp-exp-b}) serves as an explanation to the event described in
(\ref{narxexp-exp-a}). This distinction has interesting consequences as
to what we can infer from these two discourse excerpts. In the case of
(\ref{narxexp-nar}), we can infer that the event described in
(\ref{narxexp-nar-b}) occurred after the event described in
(\ref{narxexp-nar-a}). In the latter case (\ref{narxexp-exp}), we can
conversely infer that the events happened in the opposite order. We feel
that a complete understanding of the two examples above entails
inferring the correct temporal order and we would welcome a principled
way to make this kind of decisions.

\begin{exe}
  \ex \label{salmon} \begin{xlist}
    \ex Max had a lovely evening last night.
    \ex He had a great meal.
    \ex \label{salmon-here} He ate salmon.
    \ex He devoured lots of cheese.
    \ex He then won a dancing competition.
  \end{xlist}
\end{exe}

%% \sdrtree{
%%  &                      & \LAB{Max had a lovely evening} \\
%%           & \LAB{He had a great meal} & & \LAB{He won a dancing competition} \\
%% \LAB{He ate salmon} &     & \LAB{He devoured cheese}
%% }

The discourse in (\ref{salmon}) and its structure in (TODO) demonstrate
another feature of discourse structures. First off, seeing the
hierarchical structure of the discourse gives us important information
about the granularity of the descriptions given in the individual
propositions, information that could be useful for performing tasks such
as text summarization.

Furthermore, the discourse structure has grammatical consequences. It is
thanks to our knowledge of the discourse structure that we can predict
that a proposition like ``It was a beautiful pink.''  could not
coherently follow our excerpt. SDRT, \cite{asher2003logics}, the theory
of discourse structure that we adhere to, would not license a discourse
structure in which the new proposition connects to (\ref{salmon-here})
as its \emph{Elaboration}. As a consequence, it states that the pronoun
``it'' cannot have the salmon as its antecedent, which is a constraint
that is not enforced by prior theories of discourse such as DRT.

\begin{exe}
  \ex \label{diaimp} \begin{xlist}
    \ex A: Smith doesn't seem to have a girlfriend.
    \ex \label{diaimp-b} B: He's been paying lots of visits to New York lately.
  \end{xlist}
\end{exe}

In the example dialogue (\ref{diaimp}), based on the prosody of
proposition (\ref{diaimp-b}) the discourse structure would link the two
propositions with an \emph{Evidence} or \emph{Counter-evidence}
relation. Knowing this structure would allow us to either infer that B
believes that Smith has a girlfriend in New York or that Smith doesn't
have a girlfriend because he is too busy in New York.

Based on the findings in \cite{asher2003logics}, we surmise that a clear
picture of the discourse structure is essential to capturing the
intended meaning of a discourse.

To support our approach, we will stand on the shoulders of many giants,
the first of them being Richard Montague. As in his approach
\cite{montague1973proper}, we will assign functions and values as
denotations of wordforms and use function application to compose them
together to yield the denotations of phrases, propositions and
discourses. To account for the discourse-level phenomena, we will defer
to the theories of DRT \cite{kamp1993discourse} and SDRT
\cite{asher2003logics}.

The grammatical formalism of our choice for this task will be the
Abstract Categorial Grammars (ACG) \cite{de2001towards}. This framework
lets use lambda calculus to express the syntax-semantics interface in a
fashion similar to Montague's. Furthermore, elegant techniques for using
ACGs to implemenent DRT and SDRT using continuations have been
discovered \cite{de2006towards} \cite{asher2011sdrt}
\cite{asher2011montagovian}.

However, abstract categorial grammars are quite young and no significant
grammar has been developed under this framework. To facilitate the
creation of such a grammar, we will borrow heavily from the Frigram
Interaction Grammar \cite{perrier2007french} and its linguistic
resources. Namely, we will rely on the fact that Frigram is defined
separately from its lexicon, Frilex, which we will use in its
entirety. Frigram, the grammar itself, can also serve us as a guideline
when designing our own grammar thanks to the close ties between the
formalisms of interaction grammars and abstract categorial grammars
\cite{perrier1999intuitionistic}.

Finally, our work will be supported by the existence of a corpus
annotated with the rhetorical relations of SDRT, the Annodis corpus
\cite{afantenos2012empirical}.

\section{Outline}

In the rest of this chapter, we proceed to introduce two grammatical
formalisms that are of interest to our work, the formalism of abstract
categorial grammars and the formalism interaction grammars. We talk
about a formal connection between the two and its value with respect to
constructing our grammar.

In Chapter~\ref{chap:implementation}, we present our software
implementation of the ACG machinery which will enable us to write and
test grammars lexicalized by a lexicon.

In Chapter~\ref{chap:constraints}, we take the first steps into
designing a wide-coverage grammar by showing the treatment of several
disparate linguistic constraints in the framework of ACGs and the
challenges inherent in trying to combine them all in a single grammar.

Chapter~\ref{chap:gacg} builds on the problems highlighted in
Chapter~\ref{chap:constraints} and introduces a generalization of
abstract categorial grammars that is geared towards solving these
problems.

We conclude our findings with Chapter~\ref{chap:conclusion}, in which we
summarize our findings and point out potential directions for future
work.

\section{Abstract Categorial Grammars}
\label{sec:acg}

We present the grammatical framework in which we will develop our
system. Abstract categorial grammars are built upon two mathematical
structures, \emph{(higher-order) signatures} and \emph{lexicons}.

\subsection{Higher-order signatures}
\label{ssec:sig}

A \textbf{higher-order signature} is a set of elements that we call
\emph{constants}, each of which is associated with a type. Formally, it
is defined as a triple $\Sigma = \mathopen{<}A, C, \tau\mathclose{>}$,
where:
\begin{itemize}
  \item $C$ is the (finite) set of constants
  \item $A$ is a (finite) set of atomic types
  \item $\tau$ is the type associating mapping from $C$ to $\mathcal{T}(A)$,
    the set of types built over $A$
\end{itemize}

In our case, $\mathcal{T}(A)$ is the implicative fragment of linear and
intuitionistic logic with $A$ being the atomic propositions. This means
that $\mathcal{T}(A)$ contains all the $a \in A$ and all the $\alpha \limp
\beta$ and $\alpha \to \beta$ for $\alpha, \beta \in \mathcal{T}(A)$.

A signature $\Sigma = \mathopen{<}A, C, \tau\mathclose{>}$, by itself,
already lets us define an interesting set of structures, that is the set
$\Lambda(\Sigma)$ of \emph{well-typed lambda terms} built upon the
signature $\Sigma$. The set of well-typed terms and their types are
established through the judgment schemas in Figure
\ref{fig:type-judgments}.

\begin{figure}
  \begin{prooftree}
    \AxiomC{$\emptyset; \Gamma_i \vdash_\Sigma c : \tau(c)$ (cons)}
  \end{prooftree}
  \begin{prooftree}
    \AxiomC{$(x : \alpha); \Gamma_i \vdash_\Sigma x : \alpha$ (l-var)}
  \end{prooftree}
  \begin{prooftree}
    \AxiomC{$\emptyset; (\Gamma_i, x : \alpha) \vdash_\Sigma x : \alpha$ (i-var)}
  \end{prooftree}
  \begin{prooftree}
    \AxiomC{$(\Gamma_l, x : \alpha); \Gamma_i \vdash_\Sigma t : \beta$}
    \RightLabel{(l-abs)}
    \UnaryInfC{$\Gamma_l; \Gamma_i \vdash_\Sigma \lambda^{\circ} x. t : \alpha \limp \beta$}
  \end{prooftree}
  \begin{prooftree}
    \AxiomC{$\Gamma_l; (\Gamma_i, x : \alpha) \vdash_\Sigma t : \beta$}
    \RightLabel{(i-abs)}
    \UnaryInfC{$\Gamma_l; \Gamma_i \vdash_\Sigma \lambda x. t : \alpha \to \beta$}
  \end{prooftree}
  \begin{prooftree}
    \AxiomC{$\Gamma_l; \Gamma_i \vdash_\Sigma t : \alpha \limp \beta$}
    \AxiomC{$\Delta_l; \Delta_i \vdash_\Sigma u : \alpha$}
    \RightLabel{(l-app)}
    \BinaryInfC{$(\Gamma_l, \Delta_l); (\Gamma_i, \Delta_i) \vdash_\Sigma (t\ u) : \beta$}
  \end{prooftree}
  \begin{prooftree}
    \AxiomC{$\Gamma_l; \Gamma_i \vdash_\Sigma t : \alpha \to \beta$}
    \AxiomC{$\emptyset; \Delta_i \vdash_\Sigma u : \alpha$}
    \RightLabel{(i-app)}
    \BinaryInfC{$\Gamma_l; (\Gamma_i, \Delta_i) \vdash_\Sigma (t\ u) : \beta$}
  \end{prooftree}
  \caption{\label{fig:type-judgments} Type judgment schemas of the
    well-typed lambda terms $\Lambda(\Sigma)$ for a signature $\Sigma =
    \mathopen{<}A, C, \tau\mathclose{>}$}
\end{figure}

The definition of a well-typed lambda term already gives us an
interesting combinatorial structure. To make this structure even more
useful, we often focus ourselves only on terms that have a specific
\emph{distinguished type}. Using this notion of a signature of typed
constants and some distinguished type, we can describe languages of,
e.g. tree-like (and by extension string-like), lambda terms.

\subsection{Example signature}
\label{ssec:example-sig}

Let us take a look at an example from
\cite{pogodalla2007generalizing}. We will start with a very simple
signature of string expressions with the concatenation operator,
$\Sigma_{String} = \mathopen{<}A_{String}, C_{String},
\tau_{String}\mathclose{>}$. Our signature will need only one atomic
type, $A_{String} = \{\textsc{String}\}$. Our constants will be the
wordforms of our example fragment ($every$, $some$, $man$, $woman$,
$loves$), the empty string $\epsilon$ and the string concatention
operator, $+$. $\tau_{String}$ assigns the type $\textsc{String}$ to all
the wordforms and to $\epsilon$ and the type $\textsc{String} \limp
\textsc{String} \limp \textsc{String}$\footnote{As a convention,
  whenever we omit parentheses in a type, we presume the $\limp$ and
  $\to$ type constructors always bind from the right, meaning that $a
  \to b \to c$ should be read as $a \to (b \to c)$.}
to the concatenation operator $+$.

Now we can look at some well-typed terms from the string domain
(i.e. elements of $\Lambda(\Sigma_{String})$). The term $t_1 =
some\ +\ woman$\footnote{Here we introduce another notational
  convention. The unadorned way of writing this term would be $t_1 =
  (+\ some)\ woman$. Firstly, we will allow ourselves to drop the
  parentheses, meaning that any string of expressions $f\ x\ y$ should
  be read as $(f\ x)\ y$. Secondly, we will admit a specific infix
  notation for some constants which denote functions of two
  arguments. This will then let us write $x\ +\ y$ instead of
  $+\ x\ y$.} denotes the concatenation of $some$ and $woman$, which is
the phrase $some\ woman$; its type is $\textsc{String}$. A term such as
$t_2 = \lambda^{\circ} x. x\ +\ man$ denotes a function that appends the
string $man$ to its argument; its type is $\textsc{String} \limp
\textsc{String}$. If we restrict the set $\Lambda(\Sigma_{String})$ only
to terms having the type $\textsc{String}$, we get the set of
expressions which denote strings (a set that contains terms like $t_1$
but not $t_2$).

\subsection{Lexicons}

The idea of a signature is coupled with that of a \textbf{lexicon},
which is a mapping between two different signatures (mapping the
constants of one into well-typed terms of the other). Formally speaking,
a lexicon $\mathcal{L}$ from a signature $\Sigma_1 = \mathopen{<}A_1, C_1,
\tau_1\mathclose{>}$ (which we call the abstract signature) to a
signature $\Sigma_2 = \mathopen{<}A_2, C_2, \tau_2\mathclose{>}$ (which
we call the object signature) is a pair $\mathopen{<}F, G\mathclose{>}$
such that:
\begin{itemize}
\item $G$ is a mapping from $C_1$ to $\Lambda(\Sigma_2)$ assigning to
  every constant of the abstract signature a term in the object
  signature, which can be understood as its
  interpretation/implementation/realization.
\item $F$ is a mapping from $A_1$ to $\mathcal{T}(A_2)$ which links the
  abstract-level types with the object-level types that they realized
  in.
\item $F$ and $G$ are compatible, meaning that for any $c \in C_1$, we
  have $\vdash_{\Sigma_2} G(c) : \hat{F}(\tau_1(c))$ (we will be using
  $\hat{F}$ and $\hat{G}$ to refer to the homomorphic extensions of $F$
  and $G$ to $\mathcal{T}(A_1)$ and $\Lambda(\Sigma_1)$ respectively).
\end{itemize}

An abstract categorial grammar for us will then be just a collection of
signatures with their distinguished types and lexicons connecting these
signatures. A common pattern will have us using two object signatures
for the surface forms of utterances (strings) and their logical forms
(logical propositions) and an abstract signature which is connected to
both of the object signatures via lexicons. Parsing is then just a
matter of inverting the surface lexicon to get the abstract term and
then applying to it the logical lexicon. Generation is symmetric, we
simply invert the logical lexicon and apply the surface lexicon.

\subsection{Example lexicon}
\label{ssec:example-lex}

\newcommand{\synt}[1]{C_{\textrm{#1}}}

To illustrate the ideas of a lexicon and an abstract categorial grammar,
we will expand our example from \ref{ssec:example-sig}. We will
consider another signature, $\Sigma_{Synt}$, which will describe the
syntax of quantified noun phrases in the style of Montague
\cite{montague1973proper}. The atomic types $A_{Synt}$ will consist of
the types $np$, $n$ and $s$. Our constants, $C_{Synt} = \{\synt{every},
\synt{some}, \synt{love}, \synt{man}, \synt{woman}\}$, will have the
following types, as predicted by $\tau_{Synt}$:
\begin{itemize}
\item $\tau_{Synt}(\synt{every}) = \tau_{Synt}(\synt{some}) = n \limp
  ((np \limp s) \limp s)$
\item $\tau_{Synt}(\synt{love}) = np \limp np \limp s$
\item $\tau_{Synt}(\synt{man}) = \tau_{Synt}(\synt{woman}) = n$
\end{itemize}

Let us explore some terms from $\Lambda(\Sigma_{Synt})$. $t_3 =
\synt{some}\ \synt{woman}$ has type $(np \limp s) \limp s$, which is the
type that serves to describe quantified noun phrases in our
example. Terms of this type expect a verb phrase (or some other
predicate) of type $np \limp s$, and can yield a sentence of type $s$.

Now we will see how this approach can handle transitive verbs. In $t_4 =
\lambda^{\circ} x.\ (\synt{some}\ \synt{woman})\ (\lambda^{\circ}
y.\ \synt{love}\ x\ y)$, which is a term of type $np \limp s$
representing the verb phrase \emph{loves some woman}, we first take the
expression $\synt{love}\ x\ y$ (\emph{$x$ loves $y$}, type $s$) and we
abstract over $y$ to get the predicate $\lambda^{\circ}
y.\ \synt{love}\ x\ y$ (\emph{is loved by $x$}, type $np \limp s$). By
applying the quantified noun phrase $t_3$ to this term, we get
$(\synt{some}\ \synt{woman})\ (\lambda^{\circ} y.\ \synt{love}\ x\ y)$
(\emph{$x$ loves some woman}, type $s$). Finally, by abstracting over
$x$, we get the verb phrase $t_4$.

$t_5 = (\synt{every}\ \synt{man})\ (\lambda^{\circ}
x.\ (\synt{some}\ \synt{woman})\ (\lambda^{\circ}
y.\ \synt{love}\ x\ y))$ is the result of applying the quantified noun
phrase $\synt{every}\ \synt{man}$ (of type $(np \limp s) \limp s$) to
$t_4$. What we get is a term of type $s$ which describes the sentence
\emph{every man loves some woman}. Note that as before, we can restrict
the set $\Lambda(\Sigma_{Synt})$ to terms having the distinguished type
$s$ and we will arrive at the set of terms which describe sentences
(this set will include $t_5$, but not $t_3$ or $t_4$).

In the example above, I have been presuming some kind of implied
connection between the terms of $\Lambda(\Sigma_{Synt})$ and phrases of
some natural language, strings. Indeed, while we have defined a system
of structures which describe the quantification and predicate structures
of a microscopic fragment of English and a system for talking about
strings in concatenation, we have not given any explicit link between
the two. It is at this moment that we will introduce a lexicon to link
these two levels of description.

Our lexicon $\mathcal{L}_{syntax}$ will map the constants of
$\Sigma_{Synt}$, our abstract signature, into terms from
$\Lambda(\Sigma_{String})$, our object signature. If we view terms of
$\Lambda(\Sigma_{Synt})$ as abstract computations, the lexicon will
instantiate this abstraction by providing an implementation for the
constants of $\Sigma_{Synt}$. This way we can map an abstract
computation representing a phrase into a more specific object
computation that computes the string representing the phrase.

So, how does our lexicon $\mathcal{L}_{syntax}$ look like? It will map
all the atomic types of $\Sigma_{Synt}$ into $\textsc{String}$, which we
can write as $\mathcal{L}_{syntax}(n) = \mathcal{L}_{syntax}(np) =
\mathcal{L}_{syntax}(s) = \textsc{String}$\footnote{Whenever we talk
  about some lexicon, we will use the lexicon itself, which is formally
  a pair of mappings $\mathopen{<}F, G\mathclose{>}$, to mean either
  $F$, $G$ or their homomorphic extensions $\hat{F}$ and $\hat{G}$
  depending on whether the argument is an atomic type, a constant, a
  complex type or a term respectively.}. From that, we can use the
homomorphic extension of our type mapping to arrive at the object-level
interpretations of complex types, such as the type of predicates, $np
\limp s$, which becomes a unary string function, $\textsc{String} \limp
\textsc{String}$, or the type of quantified noun phrases, $(np \limp s)
\limp s$, which becomes a higher-order string function,
$(\textsc{String} \limp \textsc{String}) \limp \textsc{String}$. With
the types out of the way, we can give the interpretation of the
individual constants of $\Sigma_{Synt}$:
\begin{itemize}
\item $\mathcal{L}_{syntax}(\synt{every}) = \lambda^{\circ} x
  R.\ R\ (every\ +\ x)$
\item $\mathcal{L}_{syntax}(\synt{some}) = \lambda^{\circ} x
  R.\ R\ (some\ +\ x)$
\item $\mathcal{L}_{syntax}(\synt{love}) = \lambda^{\circ} x
  y.\ x\ +\ loves\ +\ y$
\item $\mathcal{L}_{syntax}(\synt{man}) = man$
\item $\mathcal{L}_{syntax}(\synt{woman}) = woman$
\end{itemize}

We can now go back to our examples from $\Lambda(\Sigma_{Synt})$ and
look at what they look like after applying the lexicon to them.
\begin{itemize}
\item $t_3 = \synt{some}\ \synt{woman}$ \\ Mapping this term using the
  lexicon and $\beta$-reducing gives us $\mathcal{L}_{syntax}(t_3)
  =_{\beta} \lambda^{\circ} R.\ R\ (some\ +\ woman)$, which is a
  type-raised version of the string \emph{some woman}.
\item $t_4 = \lambda^{\circ}
  x.\ (\synt{some}\ \synt{woman})\ (\lambda^{\circ}
  y.\ \synt{love}\ x\ y)$ \\ Our verb phrase ends up being mapped onto
  a function which appends the phrase \emph{loves some woman} to its
  argument, $\mathcal{L}_{syntax}(t_4) =_{\beta} \lambda^{\circ}
  x.\ x\ +\ loves\ +\ some\ +\ woman$.
\item $t_5 = (\synt{every}\ \synt{man})\ (\lambda^{\circ}
  x.\ (\synt{some}\ \synt{woman})\ (\lambda^{\circ}
  y.\ \synt{love}\ x\ y))$ \\ Finally, the sentence (type $s$) is mapped
  into a simple $\textsc{String}$, $\mathcal{L}_{syntax}(t_5) =_{\beta}
  every\ +\ man\ +\ loves\ +\ some\ +\ woman$.
\end{itemize}

Now we have enough machinery in play to define the set of strings which
form our fragment of English. We will consider the \emph{abstract
  language} generated by our signature $\Sigma_{Synt}$ and the
distinguished type $s$, that is, elements of $\Lambda(\Sigma_{Synt})$
having type $s$. Then we can define the \emph{object language}, which is
the image of the abstract language when transformed using the
lexicon. The object language contains strings, terms from
$\Lambda(\Sigma_{String})$ having the type $\textsc{String}$. However,
the object language does not contain all such terms, but only those
object terms, for which there exists a term in the abstract language
such that its image given by the lexicon yields the same object term,
i.e. it contains only terms which denote strings that spell out
sentences of our fragment of English.

\subsection{Example semantic lexicon}
\label{ssec:acg-sem}

One perk of working with abstract categorial grammars is that one
abstract term can be interpreted multiple ways, using different
lexicons. In the previous chapter, we considered the abstract terms of
the $\Sigma_{Synt}$ signature and interpreted them as computations on
strings. Now we will take the same abstract terms, but try to intepret
them as computations on entities and truth values.

We will introduce a new signature, $\Sigma_{Sem}$, for our semantic
computations. The atomic types $A_{Sem}$ will consist of a type for
entities, $e$, and a type for truth values, $t$. The constants $C_{Sem}$
will contain atomic predicates from our fragment, $\textbf{man}$,
$\textbf{woman}$ and $\textbf{love}$, logical connectives, $\Rightarrow$
and $\land$, and quantifiers, $\forall$ and $\exists$. Their types as
given by $\tau_{Sem}$ are:
\begin{itemize}
\item $\tau_{Sem}(\textbf{man}) = \tau_{Sem}(\textbf{woman}) = e \limp
  t$
\item $\tau_{Sem}(\textbf{love}) = e \limp e \limp t$
\item $\tau_{Sem}(\Rightarrow) = \tau_{Sem}(\land) = t \limp t \limp t$
\item $\tau_{Sem}(\forall) = \tau_{Sem}(\exists) = (e \to t) \limp t$
\end{itemize}

Now to see how this signature connects to our abstract signature
$\Sigma_{Synt}$, we will give a lexicon from the latter to the
former. First, we give the type mappings. $\mathcal{L}_{sem}(n) = e
\limp t$ meaning that nouns will yield boolean functions of one argument
(unary \emph{predicates} in functional programming
terminology). $\mathcal{L}_{sem}(np) = e$, simple noun phrases will be
mapped into terms which yield some entity. Finally,
$\mathcal{L}_{sem}(s) = t$, sentences will be assigned terms which
compute their truthiness. Let us see how this plan is achieved in the
$\mathcal{L}_{sem}$ lexicon.
\begin{itemize}
\item $\mathcal{L}_{sem}(\synt{every}) = \lambda^{\circ} P Q.\ \forall
  x.\ ((P\ x) \Rightarrow (Q\ x))$\footnote{We will be using $\forall
    x.\ M$ as a shortcut for $\forall\ (\lambda x.\ M)$ and $\exists
    x.\ M$ for $\exists\ (\lambda x.\ M)$.}
\item $\mathcal{L}_{sem}(\synt{some}) = \lambda^{\circ} P Q.\ \exists
  x.\ ((P\ x) \land (Q\ x))$
\item $\mathcal{L}_{sem}(\synt{love}) = \lambda^{\circ} s o. (\textbf{love}\ s\ o)$
\item $\mathcal{L}_{sem}(\synt{man}) = \lambda^{\circ} x. (\textbf{man}\ x)$
\item $\mathcal{L}_{sem}(\synt{woman}) = \lambda^{\circ} x. (\textbf{woman}\ x)$
\end{itemize}

Let us now consider the example terms of the $\Sigma_{Synt}$ signature
from \ref{ssec:example-lex}. $t_3 = \synt{some}\ \synt{woman}$ will be
interpreted as $\mathcal{L}_{sem}(t_3) =_{\beta} \lambda^{\circ}
Q.\ \exists x.\ ((\textbf{woman}\ x) \land (Q\ x))$ having type $(e \limp
t) \limp t$. It is a function that expects some predicate function and
tests whether that predicate holds at least for some woman (some entity
for which the $\textbf{woman}$ predicate holds).

In $t_4 = \lambda^{\circ}
x.\ (\synt{some}\ \synt{woman})\ (\lambda^{\circ}
y.\ \synt{love}\ x\ y)$, we get $\mathcal{L}_{sem}(t_4) =_{\beta}
\lambda^{\circ} x.\ \exists y.\ ((\textbf{woman}\ y) \land
(\textbf{love}\ x\ y))$ of type $e \limp t$. What we have here is a
predicate function testing whether the argument entity loves some woman.

Finally, in $t_5 = (\synt{every}\ \synt{man})\ (\lambda^{\circ}
x.\ (\synt{some}\ \synt{woman})\ (\lambda^{\circ}
y.\ \synt{love}\ x\ y))$, we have $\mathcal{L}_{sem}(t_5) =_{\beta}
\forall x.\ ((\textbf{man}\ x) \Rightarrow (\exists
y.\ ((\textbf{woman}\ y) \land (\textbf{love}\ x\ y))))$ of type
$t$. This computation yields a truth value, which tells us whether the
proposition that every man loves some woman is true given some universe
that the quantifiers range over and some values of the $\textbf{man}$,
$\textbf{woman}$ and $\textbf{love}$ predicates. The term itself, in its
$\beta$-normal form, can serve as our semantic representation for the
proposition \emph{every man loves some woman}.

We will finish our exposition of abstract categorial grammars by
highlighting a prominent feature of the example grammar we have just
described and that is its treatment of scope ambiguity. You might have
noticed that $t_5 = (\synt{every}\ \synt{man})\ (\lambda^{\circ}
x.\ (\synt{some}\ \synt{woman})\ (\lambda^{\circ}
y.\ \synt{love}\ x\ y))$ is not the only term in
$\Lambda(\Sigma_{Synt})$ for which $\mathcal{L}_{syntax}(t_5) =_{\beta}
every\ +\ man\ +\ loves\ +\ some\ +\ woman$. We could just as well take $t_6
= (\synt{some}\ \synt{woman})\ (\lambda^{\circ}
y.\ (\synt{every}\ \synt{man})\ (\lambda^{\circ}
x.\ \synt{love}\ x\ y))$ which is not $\beta$-equivalent to $t_5$. When
we apply our syntactic lexicon to $t_6$, we see that we end up with an
expression denoting the same string, $\mathcal{L}_{syntax}(t_6)
=_{\beta} every\ +\ man\ +\ loves\ +\ some\ +\ woman$.

However, when we consider a different interpretation for the abstract
constants in $\Sigma_{Synt}$, for example our semantic lexicon
$\mathcal{L}_{sem}$, we can see the differences rise to surface.

$$
\mathcal{L}_{sem}(t_5) =_{\beta} \forall x.\ ((\textbf{man}\ x)
\Rightarrow (\exists y.\ ((\textbf{woman}\ y) \land
(\textbf{love}\ x\ y))))
$$
$$
\mathcal{L}_{sem}(t_6) =_{\beta} \exists y.\ ((\textbf{woman}\ y)
\land (\forall x.\ ((\textbf{man}\ x) \Rightarrow
(\textbf{love}\ x\ y))))
$$

When we try to parse some sentence using abstract categorial grammars,
we are trying to find the abstract terms which upon transformation by
the lexicon and $\beta$-reduction yield the sentence encoded in some
object term. This is basically trying to invert the lexicon function,
modulo $\beta$-equivalence. It is therefore no surprise then that we can
find multiple abstract terms which all map to the same string term but
which can be mapped to distinct terms using the semantic lexicon. It is
this mechanism that enables abstract categorial grammars to handle
ambiguity. When we reverse the scenario and go from semantic
representations to strings, the same situation can occur (more than one
abstract term having the same semantics but different surface strings)
and this is what enables paraphrases.

\subsection{Interaction Grammars}

We will briefly discuss one more grammar formalism and that is the
formalism of Interaction Grammars \cite{guillaume2009interaction}.
Interaction grammars are a grammatical formalism centered around the
concept of \emph{polarity}. An interaction grammar is a set of
(under-specified) tree descriptions which define a set of trees that can
be constructed by superposing some of these tree descriptions while
respecting the polarities described in the nodes of the individual tree
descriptions.

We will illustrate the objects and mechanisms of interaction grammars on
a simple example sentence.

\begin{exe}
  \ex \label{ex:ig} Jean la voit.
\end{exe}

\begin{figure}
  \centering
  \includegraphics[scale=0.25]{ig-parse.pdf}
  \caption{\label{fig:ig-parse} The syntactic tree of sentence
    (\ref{ex:ig}). The numbers at the top of the nodes are labels of
    nodes from the elementary polarized tree descriptions that generated
    the syntactic tree (see Figure \ref{fig:ig-eptds}).}
\end{figure}

Figure \ref{fig:ig-parse} shows the syntactic tree assigned by Frigram
to sentence (\ref{ex:ig}). It is a rooted ordered tree with the topmost
node being the root. Solid lines indicate immediate dominance with the
higher node being the parent and the lower the child. The ordering of
children is given by the arrows, which signify the immediate precedence
relation between sister nodes.

The nodes in the tree are of three different kinds with respect to the
phonological form of their subtree. We recognize the \emph{anchor
  nodes}, which are displayed in vivid yellow and which are leaf nodes
containing some non-empty string as their phonological content (written
in a gray rectangle at the top of the node). Then we have the
\emph{empty} nodes, which are drawn in white and whose phonological form
is empty. Finally, there are the pale yellow \emph{non-empty} nodes,
internal nodes whose phonological content (the yield of the subtree) is
not empty.

The nodes of the syntactic tree are also decorated with features.

Interaction grammars are a formalism enabling us to define sets of the
structures described above, the \emph{syntactic trees}. Similar to tree
adjoining grammars, interaction grammars are formulated in terms of a
set of elementary structures which can combine to produce the final
output structures. However, unlike in tree adjoining grammars, the
output structure is not constructed from the elementary structures via
some set of algebraic operations (substitution and adjunction in case of
TAG). Instead, \emph{polarized tree descriptions} (PTDs), the elementary
structures of interaction grammars, impose constraints on the final
structure and a structure is said to be generated by some PTDs if it is
a minimal structure satisfying those constraints (it is a \emph{minimal
  model}). This distinction separates TAG as a formalism in the
generative-enumerative framework of syntax with IG as a formalism in the
model-theoretic framework.

\begin{figure}
  \centering
  \includegraphics[scale=0.25]{ig-eptds.pdf}
  \caption{\label{fig:ig-eptds} The elementary polarized tree
    descriptions used to generate the sentence (\ref{ex:ig}).}
\end{figure}

In Figure~\ref{fig:ig-eptds}, we can see the elementary polarized tree
descriptions (EPTDs) that generated the parse tree in
Figure~\ref{fig:ig-parse}.

As we said before, a PTD is a set of constraints on some syntactic
tree. Let us expound on what constraints the structures of
Figure~\ref{fig:ig-eptds} impose.

A node in a PTD can be read as a statement that there must exist a node
in the final syntactic tree that has compatible values for all the
features and that carries the same phonological string, if any. Such a
node of the syntactic tree is then called the \emph{interpretation} of
the PTD node (in Figure~\ref{fig:ig-parse}, every node of the syntactic
tree bears a list of the PTD nodes that it interprets, so you can see
exactly how the interpretation function works in our example).

A solid line between two nodes in a PTD means that the interpretations
of these two nodes must be in a parent-child relationship
(\emph{immediate dominance}). A dashed green arrow tells us that the
interpretations have to be sister nodes with the former preceding the
latter in the ordered tree (\emph{precedence}). Note that not all sister
nodes in a PTD have to be linked with this precedence relation. See for
example the nodes (1,1) and (1,5) in the EPTD of the clitic pronoun
$la$.

The formalism also allows us to specify \emph{immediate precedence} and
(large) \emph{dominance}. The latter is useful for modelling unbounded
dependencies such as those between a relative pronoun and its trace in
some embedded clause of the relative clause, but it is not used in our
present example.

Finally, one kind of constraint that is used in our example is the
orange rectangle in node (3,0), which requires that the interpretation
of (3,0) is the rightmost daughter amongst its siblings.

Now that we have covered the structural constraints imposed by the tree
descriptions, we will turn our attention to the defining characteristic
of interaction grammars, the polarities. As you have noticed on
Figure~\ref{fig:ig-eptds}, some of the features in our PTDs are
annotated with special symbols and colors. These denote the different
polarities and are used by the formalism for two distinct purposes: the
positive ($\textcolor{red}{\rightarrow}$) and negative
($\textcolor{blue}{\leftarrow}$) polarities are used to model the
resource sensitivity of languages, while the virtual polarities
($\textcolor{purple}{\sim}$) are used to for pattern matching against
the context.

The way the polarities are handled is that every model (output syntactic
tree) is required to have only saturated polarities on its features
(i.e. no positive, negative or virtual polarities). Whenever more than
one node of the PTD is mapped onto a single node of the syntactic tree,
the polarities of each feature are combined. The combination mechanism
allows us to combine a positively polarized feature with a negatively
polarized one to yield a saturated one. Virtual polarities can only be
eliminated by combining them with a saturated polarity (they are
analogous to the $=_{c}$ constraints of LFG \cite{kaplan1982lexical}).

If we look at the PTDs of Figure~\ref{fig:ig-eptds}, we can see this
polarity mechanism in action. The EPTD of $Jean$ has a positive $cat$
feature in its root node saying that it provides one $np$, and a
negative $funct$ feature saying that it expects some function. In the
EPTD of $voit$, we have two nodes for the subject and object, both of
them expecting an $np$ and providing them the $subj$ and $obj$
functions, respectively. The root of the EPTD then provides a complete
sentence of category $s$ and expects some function that this sentence
will play. The full stop EPTD finalizes the sentence by accepting a node
with category $s$ and giving it a $void$ function.

The clitic pronoun $la$ participates in the positive/negative resource
management system as well, since using the clitic fills up the object
slot in the valency of a verb. The EPTD of $la$ also uses virtual
features heavily to select the right place where to hang the pronoun.

Now that we understand the constraints imposed by PTDs, we can start to
see that the syntactic tree in Figure~\ref{fig:ig-parse} is truly a
model of the PTDs given in Figure~\ref{fig:ig-eptds} (furthermore, it is
the only minimal model). To illustrate more clearly how the polarities
and constraints of the individual PTDs end up generating the syntactic
tree of Figure~\ref{fig:ig-parse}, we can look at the result of
superposing the EPTDs of $la$ and $voit$ by merging the nodes (1,2) and
(1,5) with (2,2) and (2,4) respectively on Figure~\ref{fig:ig-partial}.

\begin{figure}
  \centering
  \includegraphics[scale=0.25]{ig-partial.pdf}
  \caption{\label{fig:ig-partial} The result of merging the EPTDs of
    $la$ and $voit$ by merging the nodes (1,5) and (1,5) with (2,2) and
    (2,4), respectively.}
\end{figure}

\subsubsection{Frigram, an interaction grammar}
\label{sssec:frigram}

The reason we are interested in interaction grammars is because of the
existence of Frigram, a large scale grammar of French which is
lexicalized by Frilex, an independent lexical resource that we will be
building our grammar on as well. Furthermore, as we will see
in~\ref{sssec:link-ig-acg}, interaction grammars are closely linked with
abstract categorial grammars, which makes Frigram a suitable grammar to
use as a guide when developing our grammar.

In this section, we will introduce the metagrammatical structure of
Frigram and talk about how it uses the formalism of interaction grammars
to solve several tricky linguistic phenomena.


Frigram is a wide-coverage interaction grammar of French. As we
mentioned before, an interaction grammar is a set of elementary
polarized tree descriptions. In a lexicalized interaction grammar, each
of these EPTDs is connected to a specific wordform whose particular use
it describes. Given the amount of wordforms that a useable grammar of
French would need to cover, the number of EPTDs in our grammar could
easily reach hundreds of thousands, if not millions.

The first step in fighting this explosion is to factor out irrelevant
differences between similar wordforms. For this purpose, Frilex was
created. Frilex is a morpho-syntactic lexicon of French compiled from
various other pre-existing lexicons of French. It is in effect a large
relation linking the wordforms of French to \emph{hypertags}, feature
structures describing the morphological properties of the wordforms and
their syntactic valencies.

With Frilex in place, Frigram can be defined as a set of unanchored
EPTDs which are paired with feature structures which delimit the subset
of the Frilex items to which the EPTDs apply. This kind of simple
``metagrammar'' already saves us a lot of effort, the number of
unanchored EPTDs defined by Frigram is somewhere around 4000 (TODO: Cite
the Frig documentation somehow).

However, defining some 4000 EPTDs manually still seems like a very
tedious, error prone and hard to maintain process. Many of the EPTDs
have to repeatedly describe common phenomena such as subject-predicate
agreement or predicate-argument saturation. Similarly as in other
software endeavors, it would be preferable to define these common
patterns in some reusable module and compose EPTDs from these building
blocks. This is where XMG steps in.

XMG is a metagrammar compiler adding yet another level of indirection
between what the grammar author writes and what ends up in the
bottom-level interaction grammar. XMG provides a language that lets the
grammar author define the lexical items of his grammar and to combine
these definitions to yield new and more elaborate lexical items.

In the case of Frigram, the lexical items (termed \emph{classes} in XMG)
take the shape of PTDs coupled with feature structures which
circumscribe the interface to Frilex. Both tree descriptions and feature
structures have very natural ways of composing together, by conjunction
and unification, respectively. This, alongside with disjunction, are the
chief tools that the grammar authors can use (and in the case of
Frigram, have used) to succintly define their grammar.

With the added capability of composing classes, the definition of
Frigram reduces to about 400 class definitions. Of these classes, 160
are terminal, meaning that they actually describe lexical items that are
to be included in the final grammar. Thanks to the disjunction
composition operator of XMG, these 160 classes end up generating the
4000 EPTDs.


\begin{figure}
  \centering
  \includegraphics[scale=0.25]{ig-neg.pdf}
  \caption{\label{fig:ig-neg} EPTDs of $aucun$ and $ne$ demonstrating
    the way polarities are used to model French negation.}
\end{figure}

We will briefly discuss how interaction grammars, and Frigram in
particular, handle a tricky linguistic phenomenon in an interesting way.
In French, negation is signalled by the particle $ne$ which must be
accompanied by one of several designated determiners, pronouns or
adverbs. The particle $ne$ assumes a position right before the inflected
verb, but its partner, e.g. the determiner $aucun$ can appear in a rich
variety of positions:

\begin{exe}
  \ex \label{ex:aucun-suj} Aucun tatou ne court.
  \ex \label{ex:aucun-obj} Jean n'aime l'odeur d'aucun tatou.
  \ex \label{ex:aucun-bad} * Le tatou qu'aucun loup chasse ne court.
  \ex \label{ex:aucun-good} Le tatou qu'aucun loup ne chasse court.
\end{exe}

Sentences (\ref{ex:aucun-suj}) and (\ref{ex:aucun-obj}) demonstrate that
the $aucun$ determiner can be used both in the subject and object
positions, as a determiner of either one of the arguments directly or of
a noun phrase which complements one of them. However, it is not
admissible to place the determiner in an embedded clause as it is in
(\ref{ex:aucun-bad}). If the determiner $aucun$ is used in an embedded
clause, it is the embedded clause itself that must be negated, as in
sentence (\ref{ex:aucun-good}).

This example demonstrates two points of interest to us.

First, it uses a new polarized feature, $neg$, to model the fact that
the two wordforms which enable negation must always co-occur. The
particle $ne$ provides a positive $neg$ feature to the clause in which
it occurs and this positive polarity must be saturated by a negative
polarity contributed by one of the possible partner words, such as
$aucun$. This shows that the usefulness of positive and negative
polarities extends beyond the $cat$/$funct$ pair that we have already
seen and which is used to model predicate/argument structure in a
similar way as was already done in tree adjoining grammars or categorial
grammars.

Second, the determiner $aucun$ can occur deep, e.g. inside a complement
of one of the arguments, but it cannot occur everywhere, as is
demonstrated by (\ref{ex:aucun-bad}), where we try to put the determiner
inside another embedded clause. There is a way to express this kind of
constraint in interaction grammars that we have not revealed
yet. Whenever we use a (large) dominance (a top-down green dashed line),
we usually want the range of nodes that this dominance relation can
cross to be somehow restricted. In interaction grammars, we can
associate a feature structure with a dominance constraint which will
force all the nodes that will span the distance between the two nodes
linked by the dominance constraint to unify with that feature
structure.

In the notation used in our illustrations (which were generated by the
Leopar parser\footnote{\url{http://leopar.loria.fr/}}), this is conveyed
by putting all the restricting features as virtual features in a new
node and splicing the new node in between the two nodes linked by the
dominance constraint before (see the node (0,4) in
Figure~\ref{fig:ig-neg}, which merely serves to ensure that all the
nodes on the path from (0,3) to (0,2) have either $np$ or $pp$ as the
value of their $cat$ feature). When using this notation, the (large)
dominance constraint should then be read as not only stating that the
top node must be an ancestor to the bottom node, but also that any
intermediate nodes between the ancestor and successor must match the
ancestor's features.

\subsubsection{The link with abstract categorial grammars}
\label{sssec:link-ig-acg}

Both abstract categorial grammars and interaction grammars, having come
out of the same research team, are both closely related to linear
logic. Not only are these two formalisms based on the same logical
framework, there is a surprisingly direct connection between the
principles guiding the syntactic composition of constants having
intuitionistic implicative linear types in abstract categorial grammars
and of polarized tree descriptions in interaction grammars.

Perrier \cite{perrier1999intuitionistic} proved an interesting result
which bridges these two worlds. His theorem states that every IILL
(intuitionistic implicative linear logic) sequent $F_1, ..., F_n \vdash
G$ is provable if and only if the syntactic description $D((F_1 \limp
... \limp F_n \limp G)^+)$ is valid. The second premise deserves some
elaboration.

What we mean by a \emph{syntactic description} is something similar to a
polarized tree description, albeit more minimalistic. Syntactic
descriptions only talk about immediate dominance and (large) dominance,
there are no precedence constraints on sister nodes and (large)
dominance cannot be restricted by a feature structure. Nodes themselves
are no longer even represented by feature structures, but by simple
atomic categories paired with polarities. Every node in a syntactic
description is thus associated with a single atomic category and a
polarity which is either positive or negative.

A syntactic description that is \emph{valid} is one for which there
exists a model. A model of a syntactic description is defined in the
same manner as for polarized tree descriptions, but instead of unifying
feature structures, we simply demand that all nodes of the description
which map to a node in the model must have the same category and the
number of positively polarized nodes must equal the number of negatively
polarized nodes.

Finally, we explain the term $D((F_1 \limp ... \limp F_n \limp
G)^+)$. The $+$ operation recursively assigns positive and negative
polarities to all formula occurrences within $F_1 \limp ... \limp F_n
\limp G$. These polarities then drive the recursive definition of $D$,
which maps this polarized formula into a syntactic description. Since we
will not cover the theorem deeply in our treatise, we will settle for
knowing that the composition of the positive polarization operation $+$
and the syntactic description producing operation $D$ maps an IILL
formula into a syntactic description such that above-stated theorem
holds. Furthermore, this mapping can be easily constructed since Perrier
gives straightforward ways of computing the results of both operations.

This theorem establishes a striking symmetry between the two
formalisms. Let us consider the case of parsing a sentence using a
lexicalized grammar in both formalism. First, the sentence is tokenized
into wordforms and for each wordform, a lexical item is selected, be it
a typed constant in some abstract signature for ACGs or an EPTD for IGs.

In ACGs, we will try to take the typed constants and to use them to
produce a term having some distinguished type $s$. Thanks to the
Curry-Howard correspondence, constructing a term having a given type is
nothing more than proving that term's type as a formula in the logic of
our type system. What this means is that our parsing problem boils down
to proving the IILL sequent $\tau_1, ..., \tau_n \vdash s$ (we omit the
inclusion of intuitionistic non-linear implications since they are not
relevant for parsing sentences). The sentence thus becomes parseable by
our grammar under the given lexical selection if and only if this
sequent is provable and furthermore, the proof of the sequent is our
desired syntactic structure in a disguise.

Now in IGs, we will conjoin all the EPTDs into a single complex
polarized tree description. Parsing is then just a matter of finding a
model for this description. The sentence is therefore parseable by our
grammar under the given lexical selection if and only if this
description is valid and furthermore, the model which proves the
description valid is our desired syntactic structure.

This symmetry turns Perrier's theorem into a strong claim that lets us
transform the types of a signature into elementary syntactic
descriptions such that some selection of descriptions will generate some
tree if and only if the selection of the corresponding typed constants
generated some properly typed term. Inverting this transformation would
then let us take the elementary syntactic descriptions of an IG like
Frigram and turn them into types for an abstract signature of our ACG.

However, this approach is not sufficient. The $D \circ +$ transformation
is far from a bijection.

Firstly, it is not injective and several types end up being represented
by the same syntactic description. What this means is that the
transformation ignores the insignificant differences between formulas
such as $a \limp (b \limp c)$ and $b \limp (a \limp c)$. This would not
pose a serious issue though, as when we would try to invert $D \circ +$,
we could simply choose one of the possible formulas which fit the
syntactic description as our canonical representation.

A bigger problem is posed by the fact that $D \circ +$ is not
surjective. Perrier \cite{perrier2001intuitionistic} characterized the
set of syntactic descriptions that can be obtained as images of IILL
sequents using our transformation $D \circ +$, the set of \emph{IILL
  tree descriptions}, which is a proper subset of the set of all
syntactic descriptions. Specifically, in an IILL tree descriptions, all
immediate dominance constraints go from negative nodes to positive
nodes, all (large) dominance constraints lead from positive nodes to
negative nodes and the root has a positive polarity.

While this property may not hold for the PTDs we find in Frigram, it
still gives us a general plan for recovering IILL types. The strategy
suggested by the description/formula connection given by the theorem
yields a mapping $F$ from the under-specified trees to IILL formulas.

$$
F(N) = F(N_1) \limp ... \limp F(N_k) \limp cat(N)
$$

where $N_1, ..., N_k$ are the children of $N$ and $cat(N)$ is the
category of $N$. If $N$ is a leaf, then $F(N)$ is simply $cat(N)$. In
this definition, we treat the tree descriptions as trees with the union
of the immediate dominance and (large) dominance constraints serving as
the edges of the tree.

This strategy corresponds to our intuitions about how should tree
descriptions be expressed using IILL types. We will consider two very
simple cases to demonstrate it.

First, picture a simple tree description with a positive root $A$ and
two (necessarily) negative children $B$ and $C$. Our strategy would give
this tree the type $B \limp C \limp A$ (or $C \limp B \limp A$), which
correctly expresses the fact that this tree needs to consume some node
of category $B$ and another one of category $C$ and is able to provide a
node of category $A$. This tree description can also be easily
translated to TAG by turning the negative leaf nodes into substitution
nodes. The conventional translation of TAG into categorial grammar then
confirms the type proposed by our strategy.

The second example will demonstrate the translation of a Frigram
EPTD. Before we do that, however, we will first discuss some of the ways
the actual PTDs of interaction grammars and Frigram differ from the IILL
tree descriptions we just talked about.

The PTDs of IG include precedence constraints. We can see them as
serving two purposes, to both provide an ordering of the constituents
and to further constrain syntactic composition. In our translation to an
ACG, we could imagine handling the first role, that of providing word
order, by defining a lexicon from our abstract signature. Each constant
would combine its arguments in the order in which their corresponding
subtrees appeared in the ordered PTD. However, the second role of
precedence constraints, to establish further restrictions on syntactic
composition, would be lost, as the only restricting factor on syntactic
composition in ACGs is the type system.

Another large difference is the use of multiple independent polarized
features per node. A possible solution here is to salvage the most
salient polarized feature, $cat$ (possibly with its complementary
feature, $funct$), and ignore the others. The results and intuitions we
have about intepreting tree descriptions as types do not offer an easy
way of modeling polarities across different dimensions, features, at the
same time. This would mean that phenomena, such as the paired
grammatical words for negation in French, which were handled using
special polarized features will need new solutions in the framework of
ACGs.

Next up are the virtual features. They serve multiple purposes as
well. Most often they are used to select nodes in the context and to
link them to the nodes contributed by the EPTD.

A common pattern using virtual features is Frigram's handling of
modifiers. The way we would model a modifier in categorial grammars
would be to have a function of type $X \limp X$ where $X$ is the type of
the consituent being modified. In TAG, we would have an adjunction tree
with a root and a foot node of category $X$ and which pins the modifier
on the $X$ phrase. This approach would work in interaction grammars as
well but the formalism offers other solutions too. The designers of
Frigram have opted for an alternate treatment in which modifying a
constituent does not increase the depth of the parse tree. Modifiers are
adjoined as sister nodes of the constituents they modify and it is
through virtual features that IGs can ensure that a modifier is inserted
only in the proper context.

See Figure~\ref{fig:ig-adj} for an example of an adjective modifier in
Frigram. Nodes (1,3) and (1,2) select the noun to be modified and its
projection, respectively. Singling out this pair of nodes lets us hang
the adjective as a child of the projection and state that it must
precede the noun in the word order. We can see that this does not
increase the depth of the noun in the parse tree, since the noun is only
referred to using virtual features in this EPTD, meaning that some other
EPTD must provide the nodes that will saturate these virtual features.

\begin{figure}
  \centering
  \includegraphics[scale=0.25]{ig-adj.pdf}
  \caption{\label{fig:ig-adj} An EPTD for the adjective $grand$ in
    Frigram.}
\end{figure}

Understanding this pattern can give us a useful tool for reading
EPTDs. Consider the EPTD of $la$ we saw in
Figure~\ref{fig:ig-eptds}. The EPTD can now be seen as applying this
pattern twice at the same time to both cliticize the pronoun to the
verbal kernel and to fill the object valency slot of the
verb.

Another use of virtual features in IGs is to restrict the range of
(large) dominance relations. We have seen this before in the EPTD of the
determiner $aucun$ (Figure~\ref{fig:ig-neg}), which had to reach up to
the nearest containing clause and hang a negative $neg$ feature there.
It is also present in the EPTD of the relative pronoun $que$
(Figure~\ref{fig:ig-que}) which will bring us back to the translation
strategy proposed above.

\begin{figure}
  \centering
  \includegraphics[scale=0.25]{ig-que.pdf}
  \caption{\label{fig:ig-que} An EPTD of the relative pronoun $que$.}
\end{figure}

Once again, at the root of the EPTD we see the modifier pattern in
action. If we take the subtree formed by the nodes (2,8) and (2,9) to
represent a modification of some constituent, say a noun, then we get a
type like $n \limp n$. However, the EPTD contains also positive and
negative $cat$ polarities. The node (2,0) has a negative $cat$ polarity
and we can consider it as an argument, meaning that the type we would
like to derive would look like $F((2,0)) \limp (n \limp n)$. However,
our strategy $F$ has nothing to say about virtual features or (large)
dominance constraints, so we will have to ignore those. If we look only
at the positive and negative $cat$ features, $F((2,0))$ becomes $np
\limp s$, which would give us $(np \limp s) \limp (n \limp n)$ as the
final type.

However, this translation is very lossy in terms of its discriminatory
power. Many of the constraints which were forced by the original EPTD
are now ignored. We can imagine improving our strategy by using the
feature structures as our types instead of atomic categories, which
would not only ensure that e.g. the $sent\_type$ of the embedded clause
must be $decl$, but would also capture the information in the polarized
$funct$ feature so that the relative pronoun $que$ is not permitted to
extract subjects. There are still constraints that we would miss out on
though, such as the fact that the subject (2,3) of the clause (2,0) must
precede any embedded clauses (2,6)...(2,5), the fact that any such
embedded clause must serve either the $modal$ or the $obj$ function and
the fact that the embedded clause (2,5) must have a subject (2,4).

Other constraints that we did not mention also do not have a direct
equivalent in ACGs. One example would be a feature of IGs that we have
not even mentioned which allows us the grammar author to provide an
\emph{exhaustive} list of children of a node in a PTD. Another kind of
tricky constraint are the orange rectangles that we can see in
Figure~\ref{fig:ig-que} which enforce that a constituent must be the
leftmost or rightmost daughter of its parent. The problem with these is
similar as with precedence relations. To model their impact on word
order would be easy using an ACG lexicon, but to model their effect on
restricting syntactic composition would be more tricky.

In the end, even though both formalisms (IGs and ACGs) are built on the
framework of linear logic and their fundamental principles of syntactic
composition have been linked together by Perrier's theorem, we conclude
that embedding Frigram into an ACG or constructing some automated
technique for translating the Frigram EPTDs into ACG types and terms
would not be feasible. We suspect that the incompatibility of these two
formalisms is due to ACG being a formalism based in the
generative-enumerative framework of syntax while IG is based in the
model-theoretic approach. Perhaps we could model the operational view of
IG, the combination of under-specified trees via superposition, by
employing some of the other connectives of linear logic to model the
fact that superposing PTDs usually involves many resources. However,
this would lead us well out of the framework of ACGs in which we wish to
conduct our research because of its elegant approach to discourse
denotations.

In our grammar, we will still gladly use the lexicon Frilex though,
which proves the usefulness of clearly defining the interface between
the lexicon and the grammar in the architecture of Frigram. In our
grammar, Frigram can still serve as a guide on how to interface with
Frilex, on which phenomena need to be covered and in what detail, and
what is a good way to structure the reusable components of a grammar of
French.


\chapter{Implementing Wide-Coverage Abstract Categorial Grammars}
\chaptermark{Implementing Wide-Coverage ACGs}
\label{chap:implementation}

Our goal is to develop large abstract categorial grammars which cover a
variety of linguistic phenomena. All of these phenomena have been
studied in detail but we would like to have a grammar which gracefully
combines all these solutions. At this scale, working out the grammar on
paper stops being sufficient to detect all the subtle interactions and
it is desirable to have a formalized representation that can be reasoned
about or tested by a computer.

To some extent, this need is already met by the \textbf{ACG development
  toolkit}\footnote{\url{http://www.loria.fr/equipes/calligramme/acg/}}. The
toolkit reads in signature and lexicon definitions and offers two modes
of operation. In the first, it checks the validity of the defined
ACG. That is to say, it checks whether the terms assigned by the
lexicons are well-typed and whether their types are consistent with the
object types dictated by the lexicon. The second mode of operation
offers an interactive experience in which the user is free to run type
inference on terms built on one of their signatures to find out whether
they are well-typed and if so, what is their principal type. The
interactive mode then lets the user apply a lexicon, or a composition of
lexicons, to a term and get the $\beta$-normal form of the object term.

We consider the above capabilities vital for doing any involved grammar
engineering in the framework of ACGs. However, there are some
limitations which make it not sufficient for our needs. The foremost of
these is that the signatures and lexicons are all explicitly realized in
(primary) memory and the toolkit expects them to be given directly in a
file. The grammars that we would like to develop will cover an
exhaustive list of wordforms and the size of our grammar definitions
would therefore grow proportionately with the size of our dictionary.

Given this setup, there are several approaches we might try, none of
them too appealing. We might write our metagrammar in a different format
and use a tool which would combine the metagrammar and the lexical
database\footnote{We will be using the term \emph{lexical database} to
  refer to what is usually called a lexicon, in order not to cause
  confusion with the lexicons of ACGs.} to produce the
lexicalized\footnote{We will be using the term \emph{lexicalized} in the
  context of ACGs to mean that the elements of the grammar (the typed
  constants and their images given by lexicons) are generated by a
  lexical database. This is not to be confused with the term used in
  \cite{yoshinaka2005complexity}.} ACG in a direct representation, ready
to be loaded by the toolkit. This workflow would make development on the
grammar tedious as every time we would change the metagrammar, we would
have to regenerate the direct representation and then load it in the ACG
toolkit. With extra effort, the former cost could be alleviated by
devising a system for regenerating only parts of the grammars which will
have changed since last time. However, we would still have to pay the
price of loading the entire grammar into the toolkit before being able
to interact with it again.

Furthermore, we could pick a small representative fragment of the
dictionary and test only on that fragment during the development of our
grammar, to reduce the time it takes to both generate the direct
representation of the grammar and the time it takes to load it into the
toolkit. This would have made developing the grammar already tenable.

However, in our approach, we have opted to spend more time on tool
development to ease subsequent work and we have developed a system which
makes defining and experimenting with lexicalized ACGs easier. The
metagrammars are defined as relational procedures which link the items
of the lexical database to elements of signatures and lexicons. The
grammars can be redefined without having to reload the lexical database
or any file of similar scale. Type inference and lexicon application
with $\beta$-normalization are provided as relations which can be used
interactively or as part of larger programs. En example of such a larger
program might be a routine to check the validity of an ACG or to test
some of its properties.

In this chapter, we will spend some time explaining the design rationale
underlying the system and the form it has now. The source code of the
system in its current state can be perused at
\url{https://github.com/jirkamarsik/acg-clj}.

\section{The Tools and Techniques}

The dominant decision which we made during the development of the system
was to use \emph{relational programming}. Relational programming is a
discipline of logic programming where relations are written in a
``pure'' style \cite{byrd2010relational}. This means that relations
cannot be picky about which arguments are passed in as inputs and which
as outputs, all modes must be permissible. Accepting such a discipline
prohibits us from using some of the special operators of Prolog such as
cut (\texttt{!}) and \texttt{copy\_term/2}. Instead, we must resort to
other techniques which do not break purity, like \emph{constraint logic
  programming}, which allows us to express properties outside of the
scope of unification by attaching constraints to terms and verifying
them at the proper time.

By insisting on purity, I/O becomes more difficult as we would not allow
the side-effectful pseudo-relations of Prolog. Therefore, we will not
force ourselves to implement the entire program front-to-back using only
relational programming. Instead, we will use a relational programming
system, miniKanren, which is embedded inside a functional programming
language. This lets us reap the benefits of relational programming
inside the core logic of our program while deferring to the functional
programming language to provide I/O and other bookkeeping operations.

\subsection{The Case for Relational Programming}

There are several reasons why we might want to choose a relational
programming system for our implementation. First off, the problems which
constitute the algorithmic core of our program are often mere textbook
examples of how to use a relational programming system. This goes for
type inference/checking/inhabitation, which are all implemented by a
single relation which is often used as a kind of ``Hello World'' program
in the world of relational programming. Similarly, implementing
$\beta$-reduction cleanly (without having to handle variable clashes by
generating new variables and $\alpha$-converting) is a motivational
example for introducing \emph{nominal logic programming}. Finally, to
reach feature parity with the ACG toolkit, one would only need to
implement the mapping of a lexicon over a term, which is a simple
functor.

The use of logic programming techniques is also very popular in our
domain. If we look at the system demonstrations at the \emph{Logical
  Aspects of Computational Linguistics 2012} conference, we see that 3
out of the 5 proposals have been implemented in a declarative/logic
programming system (Oz, Prolog). 2 of the 3 systems are parsers for
categorial grammars (automated theorem provers), which might turn out to
be an interesting direction for our system to go into as well, as
practical applications appear on the horizon.\footnote{Theorem provers
  are another problem area where miniKanren fits well, see
  $\alpha$leanTAP \cite{near2008alpha}.}

The third system at the conference was XMG, a language for writing
metagrammars, which is another goal of our system, one which
distinguishes it from the original ACG toolkit. We believe that using a
declarative programming system such as miniKanren with the ability to
extend it using a practical functional programming language presents an
empowering way of defining metagrammars. This allows the grammar
designer to compose her grammar in any way she sees fit and to use the
metaprogramming facilities of the underlying programming language to
transcribe the grammar in a way that makes the most sense in her case.

Finally, some of the structures that we will encounter in our domain are
more directly modelled in mathematics as relations rather than
functions. In a relational programming system, to implement a relation,
one just writes a relation as it is, whereas if we were forced to encode
everything in functions, we would have to simulate relational behavior
manually. Frilex, our lexical database, is an example of a structure
that has this relational shape since a single wordform can be mapped to
multiple hypertags. We will see another example of a relation within our
domain in \ref{ssec:def-sig} and we will also discuss the advantages and
drawbacks of modelling functions using relations.

\subsection{Choice of Implementation Language}

There are two programming languages which play host to a substantial
implementation of miniKanren (meaning one that has extensible
unification and/or constraints and that includes support for nominal
logic), Racket\footnote{\url{https://github.com/calvis/cKanren}} and
Clojure\footnote{\url{https://github.com/clojure/core.logic}}. The two
languages are very similar to each other and therefore the choice is of
no great significance. An appealing feature of Clojure is that it uses
abstract collection types pervasively instead of coupling the standard
operations to specific data structures such as lists and cons cells and
that it follows other similarly enlightened design decisions. Racket, on
the other hand, has had more time to mature, has better metaprogramming
facilities and is more recognized in academia. In developing our system,
we have settled for Clojure. One of the advantages of doing so is that
the Clojure implementation of miniKanren, dubbed core.logic, has
inspired the software community and the project has attracted
contributors that submit numerous issue reports and
patches.\footnote{\url{http://dev.clojure.org/jira/browse/LOGIC}}


\section{Defining Signatures and Lexicons}

In this section, we will go over the representation of signatures and
lexicons in our implementation and the basic facilities for defining
them.

\subsection{Defining Signatures}
\label{ssec:def-sig}

In~\ref{ssec:sig}, we formally presented signatures as triples
$\mathopen{<}A, C, \tau\mathclose{>}$. However, the set $C$ can be
always reconstructed as the domain of $\tau$ and the set $A$ is just the
set of all the atomic types occurring in the range of $\tau$. Therefore,
to define the signature $\mathopen{<}A, C, \tau\mathclose{>}$, it is
enough to provide the function $\tau$, which in its set-theoretical
realization is just a set of pairs. In relational programming, a
specific set can be easily encoded as a unary relation so a unary
relation will serve as our representation of a signature and defining a
signature will be a matter of constructing this relation.

\subsubsection{The Relationship Between the Lexical Database and a Signature}

We have our lexical database, a set of pairs of wordforms and hypertags,
and we want to arrive at a set of pairs of constants and their
types. This mapping can be decomposed into smaller parts by considering
the contribution of every part of the lexical database
individually. What will be the right model for the link between some
part of the lexical database and some part of our signature?

\begin{description}
  \item[Function of wordform] We could imagine that every different
    wordform of the lexical database will end up being mapped to some
    constant in the signature. However, this is unsuitable since the
    lexical database is a relation that can assign more than one
    distinct hypertag to a single wordform (e.g. a wordform with
    different possible part of speech) and we would like to have
    constants of different types for each of these hypertags.
  \item[Function of hypertag] Conversely, we might imagine that the
    hypertags of the lexical database are mapped via some function to
    the constants of the signature. Here we run into trouble when we
    consider variations in spelling or phonology. We might have two
    items in our lexical database that share the same hypertag but whose
    wordforms still differ. We would then like to have two distinct
    constants in our signature as well, so that when these two constants
    are mapped by the surface lexicon, they will yield two different
    strings.
  \item[Function of lexical entry] Since neither the wordform nor the
    hypertag is enough, we could consider the entire lexical entry (an
    element of the lexical database, a pair of a wordform and one of its
    hypertags). This solves both of the above problems but it is still
    not perfect. Consider the case of a determiner and some semantic
    signature. Since our link between elements of the lexical database
    and of our signature is a function, there must be a constant in the
    signature for every item of our lexical database. If you recall
    in~\ref{ssec:acg-sem}, the constants of the semantic signature
    $\Sigma_{Sem}$ contained predicates for the common nouns \emph{man}
    and \emph{woman} and for the verb \emph{loves} and some common
    logical furniture. However, it did not contain any constants which
    would correspond to the determiners \emph{every} and \emph{some} and
    therefore there would be no constant that the function could assign
    to these two wordforms.
  \item[Partial function of lexical entry] We could fix the issue above
    by using a partial function. This accounts for cases when a lexical
    entry has no constants to represent it in a signature. Nevertheless,
    this only fixes half of the problem. We will consider the single
    lexical entry which represents e.g. the 3rd person singular present
    tense form \emph{loves} of the transitive verb \emph{love}. A
    legitimate account of subject/object scope ambiguity would be to
    consider two constants in our abstract syntactic signature that have
    distinct images in some semantic lexicon, one giving the outer scope
    to the subject and the other to the object. This means that in order
    to define such a signature, we would need to link a single lexical
    entry in our database to more than one constant in the signature.
  \item[Relation between lexical entry and constant] Finally,
    considering all the above scenarios, the only mathematical model
    general enough to handle all of these cases is a relation. In our
    system, this will correspond exactly to a specific relation that
    will be provided by the grammar designer.
\end{description}

\subsubsection{Representing Constants and Defining Signatures}

We have established that signatures will be defined by a relation, a
relation between a lexical entry (a wordform and a hypertag) and a
constant (an identity\footnote{The identity of a constant is simply the
  thing that distinguishes it from other constants of the same signature
  having the same type.} and a type). The identity of a lexical constant
also has a complex structure. We know that the lexical constants we
define using the relation described above are always linked to a lexical
entry. This lexical entry at least partially identifies the
constant. For cases where we have more than one constant per lexical
entry, such as in the example of the two different scope readings of
\emph{loves}, we need some extra piece of information to specify exactly
which of the two constants we are referring to (we call this extra piece
of information a \emph{specifier}). This specifier can have any shape
that the grammar designer desires. We can see an example of a lexical
constant in Figure~\ref{fig:lex-const}.

\begin{figure}[h]
  \centering
\begin{verbatim}
{:type [-> NP [-> NP S]]
 :id {:lex-entry {:wordform "loves"
                  :hypertag ...}
      :spec {:scope :subject-wide}}}
\end{verbatim}
  \caption{\label{fig:lex-const} An example of a lexical constant from a
    hypothetical syntactic signature (the hypertag is elided for brevity).}
\end{figure}

The system takes care of wiring up the relations and assembling the
signature that yields these structures. The grammar designer's duty is
to provide a 4-ary relation that links the four parts of a lexical
constant: the wordform, the hypertag, the specifier and the type. This
relation can be thought of as a multi-valued ``function'' that receives
as input a wordform and a hypertag (that is, a single lexical entry) and
outputs (possibly multiple) pairs of specifiers and types.

Since we are working in a functional programming language that supports
manipulation of higher-order functions, we can provide helper functions
which construct the signature relations for the grammar designer in some
simple cases. Our library thus provides functions like
\texttt{unitypedr}, which expects a type and returns a signature that
assigns to every lexical item a single constant whose specifier is $nil$
and whose type is the supplied type, or \texttt{ht->typer}, which
expects a mapping from hypertag patterns to types and returns a
signature which tries to map the hypertags of lexical items against the
patterns and yields constants with the corresponding types and $nil$
specifiers. When other patterns emerge, the grammar designer is
completely free to implement her new abstractions using the full power
of the functional programming language.

See Figure~\ref{fig:lex-sig-impl} for an example of implementing a
lexicalized version of the $\Sigma_{Synt}$ signature
from~\ref{ssec:example-sig} using the \texttt{ht->typer} abstraction.

\begin{figure}
  \centering
\begin{verbatim}
(ht->typer {{:head {:cat "n"}}      'N
            {:head {:cat "v"
                    :trans "true"}} (-> 'NP 'NP 'S)
            {:head {:cat "det"}}    (-> 'N (-> 'NP 'S) 'S)})
\end{verbatim}
  \caption{\label{fig:lex-sig-impl} A lexicalized version of the
    syntactic signature from~\ref{ssec:example-sig}.}
\end{figure}

In the above paragraphs, we have been referring to the constants that we
were defining as \emph{lexical constants}. That was to distinguish them
from the \emph{non-lexical constants} that we will introduce now. The
reason for having non-lexical constants is that not every constant of
every signature is linked to (generated by) some item of our lexical
database. Consider, for example, the empty string $\epsilon$ and the
string concatenation operator $+$ of the $\Sigma_{String}$ signature we
gave in~\ref{ssec:example-sig} or the quantifiers and logical
connectives of the $\Sigma_{Sem}$ signature in~\ref{ssec:acg-sem}.
These constants are not associated to any lexical entry and therefore
their representations must be different. See
Figure~\ref{fig:nonlex-const} for an example of the representation we
use for non-lexicalized constants.

\begin{figure}
  \centering
\begin{verbatim}
{:type [-> T [-> T T]]
 :id {:constant-name and?}}
\end{verbatim}
  \caption{\label{fig:nonlex-const} An example of the representation of
    a non-lexicalized constant (in this case, it is the conjunction
    operator in a semantic signature).}
\end{figure}

Signatures of non-lexical constants can be simply defined by a mapping
from the constant names to their types as in
Figure~\ref{fig:nonlex-sig-impl}. Such signatures will generally not be
sufficient by themselves and will need to be complemented by other
signatures which contain lexical constants. Since we represent
signatures as unary relations, combining them to produce their unions or
intersections is just a matter of taking their disjunctions
(\texttt{ors}) or conjunctions (\texttt{ands}), respectively.

\begin{figure}
  \centering
\begin{verbatim}
(nonlex-sigr {'and?    (-> 'T 'T 'T)
              'imp?    (-> 'T 'T 'T)
              'forall? (-> (=> 'E 'T) 'T)
              'exists? (-> (=> 'E 'T) 'T)})
\end{verbatim}
  \caption{\label{fig:nonlex-sig-impl} A signature of the non-lexical
    semantic constants belonging to the semantic signature
    of~\ref{ssec:acg-sem}. The single arrow \texttt{->} corresponds to
    linear implication while the double arrow \texttt{=>} corresponds to
    intuitionistic implication.}
\end{figure}

With this, we have enough tools to fully define all the signatures of
section~\ref{sec:acg} in a lexicalized manner, see
Figure~\ref{fig:example-sig-impl}.

\begin{figure}
  \centering
\begin{verbatim}
(def synt-sig
  (ht->typer {{:head {:cat "n"}}      'N
              {:head {:cat "v"
                      :trans "true"}} (-> 'NP 'NP 'S)
              {:head {:cat "det"}}    (-> 'N (-> 'NP 'S) 'S)}))

(def string-sig
  (ors (nonlex-sigr {'++ (-> 'Str 'Str 'Str)
                     'empty-str 'Str})
       (unitypedr 'Str)))

(def sem-sig
  (ors (nonlex-sigr {'and?    (-> 'T 'T 'T)
                     'imp?    (-> 'T 'T 'T)
                     'forall? (-> (=> 'E 'T) 'T)
                     'exists? (-> (=> 'E 'T) 'T)})
       (ht->typer {{:head {:cat "n"}}      (-> 'E 'T)
                   {:head {:cat "v"
                           :trans "true"}} (-> 'E 'E 'T)})))
\end{verbatim}
  \caption{\label{fig:example-sig-impl} The definitions of the
    lexicalized versions of the example signatures of
    section~\ref{sec:acg}.}
\end{figure}


\subsection{Defining Lexicons}
\label{ssec:def-lex}

We have explained how we can use elements of our lexical database to
define signatures. Now, we will turn to lexicons. Similar to how we
reduced signatures from the formal triples $\mathopen{<}A, C,
\tau\mathclose{>}$ to just the type assignment function $\tau$, we can
reduce lexicons from the pairs $\mathopen{<}F, G\mathclose{>}$ to just
$G$, the part which operates on terms. $F$, or at least its relevant
subset, can be easily retrieved from $G$. We take the pairs of abstract
constants and object terms that $G$ is composed of, we take their types
and we arrive at pairs of abstract-level and object-level types. These
pairs form a subset of $\hat{F}$. $F$ (or at least its relevant subset),
which only maps the atomic abstract types, can be inferred from this
subset of $\hat{F}$.

In the relational framework, a mapping can be represented most directly
as a binary relation and that is the representation of lexicons that we
use in our system. In the previous subsection, we talked a lot about how
individual entries in the lexical database generate the items of a
signature. Solving the problem for signatures also solved it for
lexicons, since a lexicon contains exactly one object term for every
constant in its abstract signature. Therefore, a pair of an abstract
constant and an object term belonging to some lexicon is generated by
the same lexical entry that generated the abstract constant and its
type.

Our representation of lexical constants was purposefully engineered so
as to make the lexical entry directly accessible to the lexicon
implementation. Since the lexicon is implemented by the grammar designer
as a binary relation between constants and terms, the grammar designer
can inspect the abstract constant and directly access the hypertag and
wordform of the lexical entry that generated it, the specifier that was
assigned to it and the type of the constant, and then express the object
term assigned to the constant using any combination of these.

\subsubsection{Succinct Definitions of Lexicons}

The above is already enough to define any lexicon. However, our toolkit
presents convenience facilities that capture common forms of
lexicons. With these, we will be able to define lexicalized versions of
the lexicons of section~\ref{sec:acg} in a concise point-free
style. Furthermore, the grammar designer is not limited to the
convenience facilities we have provided and is completely free to use
the same programming language we have used to build her own abstractions
on top of ours right within her grammar definition.

\begin{figure}
  \centering
\begin{verbatim}
(def syntax-lexo
  (with-sig-consts string-sig
    (lexicalizer string-sig
                 (ht-lexiconr {{:head {:cat "n"}}
                               ,(rt (ll [_] _))
                               {:head {:cat "v"
                                       :trans "true"}}
                               ,(rt (ll [_ x y] (++ (++ x _) y)))
                               {:head {:cat "det"}}
                               ,(rt (ll [_ x R] (R (++ _ x))))}))))

(def sem-lexo
  (with-sig-consts sem-sig
    (orr (lexicalizer sem-sig
                      (ht-lexiconr {{:head {:cat "n"}}
                                    ,(rt (ll [_ x] (_ x)))
                                    {:head {:cat "v"
                                            :trans "true"}}
                                    ,(rt (ll [_ s o] (_ s o)))}))
         (ht-lexiconr {{:head {:cat "det"
                               :lemma "un"}}
                       ,(rt (ll [P Q] (exists? (il [x] (and? (P x)
                                                             (Q x))))))
                       {:head {:cat "det"
                               :lemma "chaque"}}
                       ,(rt (ll [P Q] (forall? (il [x] (imp? (P x)
                                                             (Q x))))))}))))
\end{verbatim}
  \caption{\label{fig:lex-impl} The definitions of the lexicalized
    versions of the lexicons of section~\ref{sec:acg}.}
\end{figure}

We have the lexicon analogues to \texttt{nonlex-sigr} and
\texttt{ht->typer}, they are called \texttt{nonlex-lexiconr} and
\texttt{ht-lexiconr} and they work the same way as the signature
versions but assign object terms instead of types.

When mapping lexical constants by a lexicon, we usually produce object
terms which contain constants of the object signature that are generated
by the same lexical entry as the abstract constant
(e.g. $\mathcal{L}_{syntax}(\synt{love}) = \lambda^{\circ} x y.\ x +
loves + y$). Managing this involves some tedious boilerplate that can be
avoided by another convenience facility. First, we define a lexicon that
produces the desired object term but with the object constant abstracted
out (e.g. $\lambda^{\circ} c x y.\ x + c + y$). Then, we can apply the
\texttt{lexicalizer} function to the object signature and to this
lexicon to get the desired lexicon which fills in the correct object
constants.

It is also useful to be able to easily insert the non-lexical constants
of the object signature into the object terms we are giving when
defining a lexicon. This service is provided by the anaphoric macro
\texttt{with-sig-consts} which introduces all of the non-lexical
constants of a signature into scope for easy use inside of lexicon
definitions.

Finally, since lexicons are again just relations, we can take their
unions and intersections using disjunction (\texttt{orr}) and
conjunction (\texttt{andr}), respectively.

This gives us everything we need to define the lexicalized versions of
the lexicons of section~\ref{sec:acg} in a concise and elegant
manner. See Figure~\ref{fig:lex-impl} for the definitions.


\section{Checking Signatures and Lexicons}

Since we let the grammar designer define signatures and lexicons using
arbitrary relations, she can also end up defining relations which cannot
be interpreted as well-formed signatures or lexicons. For this reason,
our toolkit provides a set of testing facilities which help check that
the signatures and lexicons are well-defined.

\subsection{Checking Properties of Large Structures}

As opposed to the existing \textbf{ACG development toolkit}, we cannot
easily enumerate all of the elements in a given signature or lexicon and
check that they are all consistent. The size of the lexicalized grammar
is too large to make this approach practical.

We solve this problem by demanding that the grammar designer names some
subset of the structure and testing is done only on that subset. In our
case, the most practical solution has been to let the designer just list
wordforms that generate a sufficient subset of the structure, i.e. the
designer lists one wordform per every category she has handled in her
grammar.

A more comfortable solution, which would also be more technically
involved, would be to have the system test the grammar on an
automatically deduced selection of constants such that they are
guaranteed to cover all the cases and code paths in the grammar
designer's definitions. Implementing such a system could still be
manageable given the small number of primitives the relational
programming system is built upon.

\subsection{Checking Signatures}
\label{ssec:check-sig}

The crucial property that we need to check for in signatures is that the
relation between the identities of constants and their types is a
function. This simply reduces to querying the system for the number of
typed constants belonging to the signature and having a specific
identity. If this number is one for all the identities in our test set,
we have verified that (the tested subset of) the signature assigns
exactly one type to every one of its constants.

\subsection{Checking Lexicons}

For lexicons, we check the three following properties:

\begin{description}
\item[The lexicon is a function,] assigning exactly one object term to
  each abstract constant. The verification proceeds analogously to the
  case of checking signatures (subsection \ref{ssec:check-sig}).

\item[The lexicon assigns only well-typed object terms.] This is a
  consequence of the homomorphism property of lexicons. We verify it by
  checking whether we can infer some type for every object term assigned
  to a constant in the test set.

\item[The lexicon has the homomorphism property.] We use the process
  that we described in subsection~\ref{ssec:def-lex} for inferring the
  type mapping of a lexicon from its term mapping.

  We first enumerate the pairs of abstract constants belonging to the
  test set and the object terms assigned to them by the lexicon. We then
  take the types of these pairs. Finally, we state, by using
  unification, that this set of pairs is a subset of the homomorphic
  extension of some mapping of atomic abstract types and let the
  relational programming system find us that mapping. If it succeeds, we
  have verified the homomorphism property of the lexicon.
\end{description}

The above technique of verifying the signature and lexicon definitions
by automated tests is not only useful for checking their
well-formedness. The grammar designer can go further and implement new
tests which demonstrate grammar-specific properties, an example of which
would be the $cat$-$funct$ principle in Frigram that imposes constraints
on the relationship between the $cat$ and $funct$ features in the EPTDs
of the grammar.


\chapter{Treatment of Multiple Linguistic Constraints}
\label{chap:constraints}

In this chapter, we will attempt to incorporate accounts of several
linguistic phenomena in a single framework, highlight the modularity
issues this brings up.

\section{Negation}
\label{sec:negation}

We will start with an account of the paired grammatical words mechanism
for negation in French, specifically the interaction between negative
noun phrases and the particle \emph{ne}. Recall the EPTDs we have seen
in~\ref{ssec:frigram}, repeated here on Figure~\ref{fig:ig-neg-rep}.

\begin{figure}
  \centering
  \includegraphics[scale=0.25]{images/ig-neg.pdf}
  \caption{\label{fig:ig-neg-rep} Frigram EPTDs for a negative
    determiner and the paired grammatical word \emph{ne}.}
\end{figure}

At a basic level, \emph{aucun} functions like any other determiner and
\emph{ne} as any verb modifier. However, we want to encode the
constraint that whenever a verb is modified by \emph{ne}, the modified
verb will demand that one of its arguments contains a negative
determiner. Since we need to make a distinction between phrases that
contain negative determiners and those that do not, we will need to
provide different types for both (two terms with the same type are
indistinguishable w.r.t. syntactic composition in a type-logical grammar
such as ACGs). Our type for \emph{aucun} will thus end up
being\footnote{Strictly following the EPTD for \emph{aucun} would lead
  us to give it a type $DET\_NEG{=}F$ and then give the two types
  $DET\_NEG{=}F \limp ((NP\_NEG{=}F \limp S) \limp S)$ and $DET\_NEG{=}T
  \limp ((NP\_NEG{=}T \limp S) \limp S)$ to nouns. In this
  demonstration, we prefer to align our examples with the treatment
  which is customary to ACG research.} $$N\_NEG{=}F \limp ((NP\_NEG{=}T
\limp S) \limp S)$$ instead of the usual $$N \limp ((NP \limp S) \limp
S)$$ The type for \emph{le} will need to be able handle cases where its
argument noun either already contains a negative determiner or not.

\begin{align*}
N_{aucun} &: N\_NEG{=}F \limp ((NP\_NEG{=}T \limp S) \limp S) \\
N_{le_1} &: N\_NEG{=}F \limp ((NP\_NEG{=}F \limp S) \limp S) \\
N_{le_2} &: N\_NEG{=}T \limp ((NP\_NEG{=}T \limp S) \limp S) \\
\end{align*}

The type of \emph{ne} will be more verbose, as it will consume a verb
and transform its valency so that it demands that exactly one of its
$NP$ arguments contains a negative determiner.

\begin{align*}
N_{ne_{tv_1}} &: (NP\_NEG{=}F \limp NP\_NEG{=}F \limp S) \limp (NP\_NEG{=}T \limp NP\_NEG{=}F \limp S) \\
N_{ne_{tv_2}} &: (NP\_NEG{=}F \limp NP\_NEG{=}F \limp S) \limp (NP\_NEG{=}F \limp NP\_NEG{=}T \limp S)
\end{align*}

Furthermore, we could add the case when both the subject and the object
contain negative determiners, such as in~(\ref{ex:aucun2}). This case is
currently not covered by Frigram (though it could be), but is considered
grammatical by French speakers. Finally, in a complete grammar,
\emph{ne} would have to provide types capable of transforming all the
other syntactic valencies.

\begin{exe}
  \ex \label{ex:aucun2} Aucune fourmi n'aime aucun tatou.
\end{exe}

\begin{align*}
N_{ne_{tv_3}} &: (NP\_NEG{=}F \limp NP\_NEG{=}F \limp S) \limp (NP\_NEG{=}T \limp NP\_NEG{=}T \limp S)
\end{align*}

\begin{figure}
  \centering
  \includegraphics[scale=0.25]{images/parse-aucun.pdf}
  \caption{\label{fig:parse-aucun} One of the parse trees assigned to
    sentence (\ref{ex:aucun-obj}) by Frigram/Leopar.}
\end{figure}

However, this is not the only change we have to effect on our grammar in
order to properly handle the paired grammatical words for
negation. Consider the example sentence (\ref{ex:aucun-obj}) and the
parse tree assigned to it by Frigram on Figure~\ref{fig:parse-aucun}. It
is not enough that the embedded noun phrase \emph{aucun tatou} has a
type that tells us it contains a negative determiner. We need this bit
of internal information also at the level of the enclosing noun phrase
\emph{l'odeur d'aucun tatou} in which the former noun phrase is embedded
since this is a property of the enclosing noun phrase that is crucial to
its syntactic combinatorics. This presupposes the existence of a
mechanism for propagating this kind of information up the parse tree. In
the case of our ACG, this means that all the functions which modify
phrases must be aware whether or not their argument contains a negative
determiner and to convey this information in its result type as
well. Moreover, since the negative determiner can also come from a
complement of a prepositional phrase, the types of prepositions will
need to take into account the presence of negative determiners in both
their complements and the phrases they modify.

\begin{align*}
N_{que_1} &: (NP\_NEG{=}F \limp S) \limp N\_NEG{=}F \limp N\_NEG{=}F \\
N_{que_2} &: (NP\_NEG{=}F \limp S) \limp N\_NEG{=}T \limp N\_NEG{=}T \\
N_{de_1} &: NP\_NEG{=}F \limp N\_NEG{=}F \limp N\_NEG{=}F \\
N_{de_2} &: NP\_NEG{=}F \limp N\_NEG{=}T \limp N\_NEG{=}T \\
N_{de_3} &: NP\_NEG{=}T \limp N\_NEG{=}F \limp N\_NEG{=}T
\end{align*}

In the above types, we not only propagate information about the presence
of a negative determiner, we also introduce a constraint in the
mechanism stating that it cannot be the case that there is a negative
determiner both in the modified noun and in its noun phrase complement
(i.e. phrases like \emph{aucun tatou d'aucun homme} are not acceptable).

If we suppose the presence of the appropriate lexical items typed as in
the examples of~\ref{sec:acg}, we have a type system in which we can
type the syntactic terms corresponding to the sentences
(\ref{ex:good-neg-double}), (\ref{ex:good-neg-embed}) and
(\ref{ex:good-neg-rel}) while making it impossible to type the
structures corresponding to sentences (\ref{ex:bad-neg-noneg}) and
({\ref{ex:bad-neg-rel}}).

In sentence (\ref{ex:bad-neg-noneg}), we have no constant
$N_{ne_{tv_?}}$ for \emph{ne} which would yield a verb capable of
consuming two noun phrases that contain no negative determiners.

In sentence (\ref{ex:bad-neg-rel}), there are multiple problems with
typing. First, $N_{chasse}$ requires its first argument to be of type
$NP\_NEG{=}F$, a noun phrase with no negative determiner. This means
that $\lambda^{\circ} x.\ N_{chasse}\ x\ y$ has type $NP\_NEG{=}F \limp
S$ which makes it an invalid argument for the function
$N_{aucun}\ N_{loup}$, which has type $(NP\_NEG{=}T \limp S) \limp
S$. Even if we were able to assign it a type compatible with
$N_{que_1}$, the final relative clause would map $N_{tatou}$ to another
term of type $N\_NEG{=}F$, which has no ``free'' negative
determiners. $N_{le_1}$ will map this to a term typed $(NP\_NEG{=}F
\limp S) \limp S$. Finally, this term is not capable of taking
$(\lambda^{\circ} x.\ N_{ne_{iv}}\ N_{court}\ x)$ as its argument, since
its type is $NP\_NEG{=}T \limp S$.

\begin{exe}
  \ex \label{ex:good-neg-double} Aucune fourmi n'aime aucun tatou. \\
      $(N_{aucune}\ N_{fourmi})\ (\lambda^{\circ} x.\ (N_{aucun}\ N_{tatou})\ (\lambda^{\circ} y.\ N_{ne_{tv_3}}\ N_{aime}\ x\ y))$
  \ex \label{ex:good-neg-embed} Jean n'aime l'odeur d'aucun tatou. \\
      $(N_{le_2}\ (N_{de_3}\ (N_{aucun}\ N_{tatou})\ N_{odeur}))\ (\lambda^{\circ} y.\ N_{ne_{tv_2}}\ N_{aime}\ N_{Jean}\ y)$
  \ex \label{ex:good-neg-rel} Le tatou qu'aucun loup ne chasse court. \\
      $(N_{le_1}\ (N_{que_1}\ (\lambda^{\circ} y.\ (N_{aucun}\ N_{loup})\ (\lambda^{\circ} x.\ N_{ne_{tv_1}}\ N_{chasse}\ x\ y))\ N_{tatou}))\ (\lambda^{\circ} x.\ N_{court}\ x)$
  \ex * \label{ex:bad-neg-noneg} Jean n'aime le tatou. \\
      * $(N_{le_1}\ N_{tatou})\ (\lambda^{\circ} y.\ N_{ne_{tv_?}}\ N_{aime}\ N_{Jean}\ y)$
  \ex * \label{ex:bad-neg-rel} Le tatou qu'aucun loup chasse ne court. \\
      * $(N_{le_1}\ (N_{que}\ (\lambda^{\circ} y.\ (N_{aucun}\ N_{loup})\ (\lambda^{\circ} x.\ N_{chasse}\ x\ y))\ N_{tatou}))\ (\lambda^{\circ} x.\ N_{ne_{iv}}\ N_{court}\ x)$
\end{exe}

There are two things that we would like to highlight about this
grammatical treatment. First off, we had to refine our types to make
them convey more than one piece of information in a single type
(e.g. something is a noun phrase and at the same time it is something
that contains a negative determiner). Second, a property that we wish to
express in the type of a term may be due to one of its subconstituents
and it is therefore necessary to encode the propagation of this property
up the parse tree in the types of all the intervening operators. In our
case, it meant that handling negative determiners and the paired
grammatical word for negation required us to not only define and type
constants for these two categories, but to also change the types of noun
modifiers (\emph{que ...}) and prepositions (\emph{de}).


\section{Extraction}
\label{sec:extraction}

We will now turn to another phenomenon, extraction in relative clauses,
focus on one of its properties, the fact that subjects can be extracted
using \emph{qui} only from the relative clause itself whereas objects
can be extracted using \emph{que} from embedded clauses contained
therein, and present an implementation of this constraint by way of
types in an ACG. The solution that we will show was published in
\cite{pogodalla2012controlling}. Our presentation is slightly adapted to
conform to the patterns seen in previous examples and it does not use
dependent types as they are not essential to this specific constraint.

The key idea in the treatment presented below is to give a special type
to extracted constituents. We will thus have two types for noun phrases:
$NP\_VAR{=}T$, which stands for ``empty'' noun phrases introduced by
extraction, and $NP\_VAR{=}F$, which stands for the other noun phrases,
i.e.  proper names and nouns with determiners. The grammar gives no
constant capable of constructing a value of type $NP\_VAR{=}T$. The only
way to obtain a term with this type is to abstract over a variable
having the type. Then, the types of relative pronouns end up being
$(NP\_VAR{=}T \limp S) \limp N \limp N$, ensuring that the incomplete
clause given as the first argument is a $\lambda$-abstraction whose
argument has type $NP\_VAR{=}T$.

Once we have this kind of information in the type system, we can start
distinguishing clauses containing rooted extraction, where a subject or
object is extracted directly from the relative clause, embedded
extraction, where an object is extracted from a clause embedded inside
the relative clause, or no extraction. Types of functions which produce
values of type $S$ will be sensitive to the presence of extracted
variables in their noun phrase and clausal arguments.

\begin{align*}
E_{dort_1} &: NP\_VAR{=}F \limp S\_EXT{=}NO \\
E_{dort_2} &: NP\_VAR{=}T \limp S\_EXT{=}ROOT \\
E_{aime_1} &: NP\_VAR{=}F \limp NP\_VAR{=}F \limp S\_EXT{=}NO \\
E_{aime_2} &: NP\_VAR{=}T \limp NP\_VAR{=}F \limp S\_EXT{=}ROOT \\
E_{aime_3} &: NP\_VAR{=}F \limp NP\_VAR{=}T \limp S\_EXT{=}ROOT \\
E_{dit\ que_1} &: NP\_VAR{=}F \limp S\_EXT{=}NO \limp S\_EXT{=}NO \\
E_{dit\ que_2} &: NP\_VAR{=}T \limp S\_EXT{=}NO \limp S\_EXT{=}ROOT \\
E_{dit\ que_3} &: NP\_VAR{=}F \limp S\_EXT{=}ROOT \limp S\_EXT{=}EMB \\
E_{dit\ que_4} &: NP\_VAR{=}F \limp S\_EXT{=}EMB \limp S\_EXT{=}EMB
\end{align*}

Now all of the infrastructure is in place for us to express the type of
the relative pronouns \emph{qui} and \emph{que}.

\begin{align*}
E_{qui} &: (NP\_VAR{=}T \limp S\_EXT{=}ROOT) \limp N \limp N \\
E_{que_1} &: (NP\_VAR{=}T \limp S\_EXT{=}ROOT) \limp N \limp N \\
E_{que_2} &: (NP\_VAR{=}T \limp S\_EXT{=}EMB) \limp N \limp N
\end{align*}

The above types ensure that the abstracted variables corresponding to
traces of extracted constituents will be given the type $NP\_VAR{=}T$,
that \emph{qui} can only be used to extract constituents from root
positions while \emph{que} allows for extraction from both root and
embedded positions.\footnote{NB: The constraint that \emph{qui} can only
  extract subjects and that \emph{que} can only extract objects is not
  enforced here. Likewise, this fragment does not handle multiple
  extraction.}

We also note that the splitting of the clause type $S$ into three finer
types $S\_EXT={NO}$, $S\_EXT{=}ROOT$ and $S\_EXT{=}EMB$ means that types
that work with $S$ regardless of its extraction status will have to
provide alternatives for all the possible variants.

\begin{align*}
E_{le_1} &: N \limp ((NP\_VAR{=}F \limp S\_EXT{=}NO) \limp S\_EXT{=}NO) \\
E_{le_2} &: N \limp ((NP\_VAR{=}F \limp S\_EXT{=}ROOT) \limp S\_EXT{=}ROOT) \\
E_{le_3} &: N \limp ((NP\_VAR{=}F \limp S\_EXT{=}EMB) \limp S\_EXT{=}EMB)
\end{align*}

While we have introduced the notion of an $NP\_VAR{=}T$ for describing
traces, we can cover another element of our running fragment, the
preposition \emph{de}, and enforce the constraint that
\emph{qui}/\emph{que} cannot extract prepositional complements by
providing only a single type for \emph{de} that accepts only unmoved
noun phrases as complements.

\begin{align*}
E_{de} &: NP\_VAR{=}F \limp N \limp N
\end{align*}

Given this type system, we can assign the type $S\_EXT{=}NO$ to the term
expressing the syntactic structure of (\ref{ex:good-ext}) but we cannot
a find a typable term for sentence (\ref{ex:bad-ext}), where \emph{qui}
is used to extract a subject from a clause embedded in a relative
clause.  The problem with sentence (\ref{ex:bad-ext}) is that the type
of the relative clause is $NP\_VAR{=}T \limp S\_EXT{=}EMB$ which is not
a valid argument for any constant representing the relative pronoun
\emph{qui}. Typing the variable $t$ with $NP\_VAR{=}F$ instead of
$NP\_VAR{=}T$ and using $E_{aime_1}$ instead of $E_{aime_2}$ would not
help, since the resulting relative clause would then have type
$NP\_VAR{=}F \limp S\_EXT{=}NO$, which would not be admitted by any
relative pronoun.

\begin{exe}
  \ex * \label{ex:bad-ext} Le tatou qui Jean dit que \_ aime Marie dort. \\
      * $(E_{le_1}\ (E_{qui_?}\ (\lambda^{\circ} t.\ E_{dit\ que_3}\ E_{Jean}\ (E_{aime_2}\ t\ E_{Marie}))\ E_{tatou}))\ (\lambda^{\circ} x.\ E_{dort_1}\ x)$
  \ex \label{ex:good-ext} Le tatou que Jean dit que Marie aime \_ dort. \\
      $(E_{le_1}\ (E_{que_2}\ (\lambda^{\circ} t.\ E_{dit\ que_3}\ E_{Jean}\ (E_{aime_3}\ E_{Marie}\ t))\ E_{tatou}))\ (\lambda^{\circ} x.\ E_{dort_1}\ x)$
\end{exe}

As was the case with our treatment of negation, we refined our familiar
types by tacking on new kinds of information and defined rules for how
these pieces of information percolate up the parse tree all the way to
the topmost level still pertinent for ensuring grammaticality.


\section{Agreement}
\label{sec:agreement}

Finally, we will look at another example of implementing a linguistic
constraint in ACGs by covering the phenomenon of number
agreement.

Number agreement motivates a straightforward refinement of our types,
where all noun and noun phrase types carry number information.

\begin{align*}
A_{Marie} &: NP\_NUM{=}SG \\
A_{tatou} &: N\_NUM{=}SG \\
A_{tatous} &: N\_NUM{=}PL \\
A_{le} &: N\_NUM{=}SG \limp ((NP\_NUM{=}SG \limp S) \limp S) \\
A_{les} &: N\_NUM{=}PL \limp ((NP\_NUM{=}PL \limp S) \limp S) \\
A_{dort} &: NP\_NUM{=}SG \limp S \\
A_{dorment} &: NP\_NUM{=}PL \limp S \\
A_{aime_1} &: NP\_NUM{=}SG \limp NP\_NUM{=}SG \limp S \\
A_{aime_2} &: NP\_NUM{=}SG \limp NP\_NUM{=}PL \limp S \\
\end{align*}
\begin{align*}
A_{aiment_1} &: NP\_NUM{=}PL \limp NP\_NUM{=}SG \limp S \\
A_{aiment_2} &: NP\_NUM{=}PL \limp NP\_NUM{=}PL \limp S \\
A_{qui_1} &: (NP\_NUM{=}SG \limp S) \limp N\_NUM{=}SG \limp N\_NUM{=}SG \\
A_{qui_2} &: (NP\_NUM{=}PL \limp S) \limp N\_NUM{=}PL \limp N\_NUM{=}PL \\
A_{de_1} &: NP\_NUM{=}SG \limp N\_NUM{=}SG \limp N\_NUM{=}SG \\
A_{de_2} &: NP\_NUM{=}PL \limp N\_NUM{=}SG \limp N\_NUM{=}SG \\
A_{de_3} &: NP\_NUM{=}SG \limp N\_NUM{=}PL \limp N\_NUM{=}PL \\
A_{de_4} &: NP\_NUM{=}PL \limp N\_NUM{=}PL \limp N\_NUM{=}PL
\end{align*}

Given the above signature, we can assign the type $S$ to the syntactic
term of sentence (\ref{ex:good-agr}) but not to the one of sentence
(\ref{ex:bad-agr}). In the latter sentence, the type of the variable $t$
has to be $NP\_NUM{=}SG$ since it is used as an argument to
$A_{dort}$. This means that the relative clause \emph{dort} has type
$NP\_NUM{=}SG \limp S$. However, there is no type for \emph{qui} which
would allow us to modify the plural noun $A_{tatous}$ of type
$N\_NUM{=}PL$ by a relative clause with a missing singular noun phrase,
which has the type $NP\_NUM{=}SG \limp S$.

\begin{exe}
  \ex \label{ex:good-agr} Les tatous de Marie dorment. \\
      $(A_{les}\ (A_{de}\ A_{Marie}\ A_{tatous}))\ (\lambda^{\circ} x.\ A_{dorment}\ x)$
  \ex * \label{ex:bad-agr} Marie aime les tatous qui dort. \\
      * $(A_{les}\ (A_{qui_?}\ (\lambda^{\circ} t.\ A_{dort}\ t)\ A_{tatous}))\ (\lambda^{\circ} y.\ A_{aime_2}\ A_{Marie}\ y)$
\end{exe}

Once again, we have expressed a linguistic constraint by refining our
types, augmenting them with a new bit of information, we constrained the
arguments of functions using these new types and we have described how
this new type information propagates from the given arguments to the
produced values.


\section{Putting It All Together}
\label{sec:together}

In this section, we will see what it takes to enforce all of the
constraints introduced in the preceding sections (negation, extraction
and agreement) at the same time, which will give us some idea on how to
go about building a grammar which handles a wide variety of such
phenomena.

In a grammar that handles negation, extraction and agreement, our types
will need to carry all of the refinements we introduced before. To
recapitulate, $NP$ will be parameterized by whether or not it contains a
negative determiner ($\_\textcolor{red}{NEG{=}T}$ and
$\_\textcolor{red}{NEG{=}F}$ respectively), whether or not it is a trace
from an extraction ($\_\textcolor{green}{VAR{=}T}$ and
$\_\textcolor{green}{VAR{=}F}$) and its number
($\_\textcolor{blue}{NUM{=}SG}$ or $\_\textcolor{blue}{NUM{=}PL}$). For
$N$, we have the same refinements except for $\textcolor{green}{VAR}$
since extraction only moves $NP$s. Clauses, of type $S$, will be
discriminated based on the level of extraction within them
($\_\textcolor{green}{EXT{=}NO}$, $\_\textcolor{green}{EXT{=}ROOT}$ and
$\_\textcolor{green}{EXT{=}EMB}$).

The individual types will have to respect all of the constraints at the
same time, each one being a complete specification of a possible
situation (complete w.r.t. the features listed above). To see and
appreciate what this means in practice, we give the types for the
preposition \emph{de}:

\begin{align*}
C_{de_1} &: (NP\_\textcolor{red}{NEG{=}F}\_\textcolor{green}{VAR{=}F}\_\textcolor{blue}{NUM{=}SG}) \limp (N\_\textcolor{red}{NEG{=}F}\_\textcolor{blue}{NUM{=}SG}) \limp
(N\_\textcolor{red}{NEG{=}F}\_\textcolor{blue}{NUM{=}SG}) \\
C_{de_2} &: (NP\_\textcolor{red}{NEG{=}F}\_\textcolor{green}{VAR{=}F}\_\textcolor{blue}{NUM{=}SG}) \limp (N\_\textcolor{red}{NEG{=}F}\_\textcolor{blue}{NUM{=}PL}) \limp
(N\_\textcolor{red}{NEG{=}F}\_\textcolor{blue}{NUM{=}PL}) \\
C_{de_3} &: (NP\_\textcolor{red}{NEG{=}F}\_\textcolor{green}{VAR{=}F}\_\textcolor{blue}{NUM{=}SG}) \limp (N\_\textcolor{red}{NEG{=}T}\_\textcolor{blue}{NUM{=}SG}) \limp
(N\_\textcolor{red}{NEG{=}T}\_\textcolor{blue}{NUM{=}SG}) \\
C_{de_4} &: (NP\_\textcolor{red}{NEG{=}F}\_\textcolor{green}{VAR{=}F}\_\textcolor{blue}{NUM{=}SG}) \limp (N\_\textcolor{red}{NEG{=}T}\_\textcolor{blue}{NUM{=}PL}) \limp
(N\_\textcolor{red}{NEG{=}T}\_\textcolor{blue}{NUM{=}PL}) \\
C_{de_5} &: (NP\_\textcolor{red}{NEG{=}T}\_\textcolor{green}{VAR{=}F}\_\textcolor{blue}{NUM{=}SG}) \limp (N\_\textcolor{red}{NEG{=}F}\_\textcolor{blue}{NUM{=}SG}) \limp
(N\_\textcolor{red}{NEG{=}T}\_\textcolor{blue}{NUM{=}SG}) \\
C_{de_6} &: (NP\_\textcolor{red}{NEG{=}T}\_\textcolor{green}{VAR{=}F}\_\textcolor{blue}{NUM{=}SG}) \limp (N\_\textcolor{red}{NEG{=}F}\_\textcolor{blue}{NUM{=}PL}) \limp
(N\_\textcolor{red}{NEG{=}T}\_\textcolor{blue}{NUM{=}PL}) \\
C_{de_7} &: (NP\_\textcolor{red}{NEG{=}F}\_\textcolor{green}{VAR{=}F}\_\textcolor{blue}{NUM{=}PL}) \limp (N\_\textcolor{red}{NEG{=}F}\_\textcolor{blue}{NUM{=}SG}) \limp
(N\_\textcolor{red}{NEG{=}F}\_\textcolor{blue}{NUM{=}SG}) \\
C_{de_8} &: (NP\_\textcolor{red}{NEG{=}F}\_\textcolor{green}{VAR{=}F}\_\textcolor{blue}{NUM{=}PL}) \limp (N\_\textcolor{red}{NEG{=}F}\_\textcolor{blue}{NUM{=}PL}) \limp
(N\_\textcolor{red}{NEG{=}F}\_\textcolor{blue}{NUM{=}PL}) \\
C_{de_9} &: (NP\_\textcolor{red}{NEG{=}F}\_\textcolor{green}{VAR{=}F}\_\textcolor{blue}{NUM{=}PL}) \limp (N\_\textcolor{red}{NEG{=}T}\_\textcolor{blue}{NUM{=}SG}) \limp
(N\_\textcolor{red}{NEG{=}T}\_\textcolor{blue}{NUM{=}SG}) \\
C_{de_{10}} &: (NP\_\textcolor{red}{NEG{=}F}\_\textcolor{green}{VAR{=}F}\_\textcolor{blue}{NUM{=}PL}) \limp (N\_\textcolor{red}{NEG{=}T}\_\textcolor{blue}{NUM{=}PL}) \limp
(N\_\textcolor{red}{NEG{=}T}\_\textcolor{blue}{NUM{=}PL}) \\
C_{de_{11}} &: (NP\_\textcolor{red}{NEG{=}T}\_\textcolor{green}{VAR{=}F}\_\textcolor{blue}{NUM{=}PL}) \limp (N\_\textcolor{red}{NEG{=}F}\_\textcolor{blue}{NUM{=}SG}) \limp
(N\_\textcolor{red}{NEG{=}T}\_\textcolor{blue}{NUM{=}SG}) \\
C_{de_{12}} &: (NP\_\textcolor{red}{NEG{=}T}\_\textcolor{green}{VAR{=}F}\_\textcolor{blue}{NUM{=}PL}) \limp (N\_\textcolor{red}{NEG{=}F}\_\textcolor{blue}{NUM{=}PL}) \limp
(N\_\textcolor{red}{NEG{=}T}\_\textcolor{blue}{NUM{=}PL})
\end{align*}

This highlights two important problems with the current approach. One is
the untenable complexity. Even though the three mechanisms are
completely independent of each other, here we are forced to refer to all
of them in every type. It is no longer easy to glance at the types
assigned to \emph{de} and see how it behaves with respect to, for
example, the mechanism of negation (compare with the type assignments we
gave to \emph{de} in previous subsections). Another cause of
unreadability of this signature is due to the sheer number of types on
display, which leads us to our second issue.

The signature demonstrates an exponential growth in the number of typed
constants included. In cases where the behaviors of a wordform across
the different mechanisms are independent, the complete type signature
will need to provide a type for every combination of situations in all
of the mechanisms. Assuming that every mechanism will assign on average
more than one type to every wordform, the number of the wordform's types
in the final signature will be exponential w.r.t. the number of
different mechanisms handled.

We can try making the enumeration of these types easier for the grammar
author by introducing metavariables for the values of the features. This
means that we could write the following 3 templates instead of the 12
types above:

$$
\forall \textcolor{blue}{x}, \textcolor{blue}{y} \in \{\textcolor{blue}{SG}, \textcolor{blue}{PL}\}
$$
\begin{align*}
C_{de_{1(\textcolor{blue}{x}, \textcolor{blue}{y})}} &: (NP\_\textcolor{red}{NEG{=}F}\_\textcolor{green}{VAR{=}F}\_\textcolor{blue}{NUM{=}x}) \limp (N\_\textcolor{red}{NEG{=}F}\_\textcolor{blue}{NUM{=}y}) \limp (N\_\textcolor{red}{NEG{=}F}\_\textcolor{blue}{NUM{=}y}) \\
C_{de_{2(\textcolor{blue}{x}, \textcolor{blue}{y})}} &: (NP\_\textcolor{red}{NEG{=}F}\_\textcolor{green}{VAR{=}F}\_\textcolor{blue}{NUM{=}x}) \limp (N\_\textcolor{red}{NEG{=}T}\_\textcolor{blue}{NUM{=}y}) \limp (N\_\textcolor{red}{NEG{=}T}\_\textcolor{blue}{NUM{=}y}) \\
C_{de_{3(\textcolor{blue}{x}, \textcolor{blue}{y})}} &: (NP\_\textcolor{red}{NEG{=}T}\_\textcolor{green}{VAR{=}F}\_\textcolor{blue}{NUM{=}x}) \limp (N\_\textcolor{red}{NEG{=}F}\_\textcolor{blue}{NUM{=}y}) \limp (N\_\textcolor{red}{NEG{=}T}\_\textcolor{blue}{NUM{=}y})
\end{align*}

This is getting more concise and easier to comprehend. We can take this
approach further by not only quantifying over variables but also
computing some of the values using functions:

$$
\forall \textcolor{blue}{x}, \textcolor{blue}{y} \in\{\textcolor{blue}{SG}, \textcolor{blue}{PL}\}, \forall \textcolor{red}{p}, \textcolor{red}{q} \in \{\textcolor{red}{F}, \textcolor{red}{T}\}, \textcolor{red}{p \land q} \neq \textcolor{red}{T}
$$
\begin{align*}
C_{de(\textcolor{blue}{x}, \textcolor{blue}{y}, \textcolor{red}{p}, \textcolor{red}{q})} &: (NP\_\textcolor{red}{NEG{=}p}\_\textcolor{green}{VAR{=}F}\_\textcolor{blue}{NUM{=}x}) \limp (N\_\textcolor{red}{NEG{=}q}\_\textcolor{blue}{NUM{=}y}) \limp (N\_\textcolor{red}{NEG{=}(p \lor q)}\_\textcolor{blue}{NUM{=}y})
\end{align*}

The above is vastly more concise and manages to convey the intent more
clearly (the intent being that it cannot be the case that there are
negative determiners in both the complement noun phrase and the modified
noun and that the value of the result's \textcolor{red}{NEG} feature is
always the disjunction of the \textcolor{red}{NEG} features of the two
arguments).

As of right now, we have introduced this kind of rule only as
meta-notation. This means that the real underlying grammar that will end
up being used by some algorithm still suffers from the same exponential
cardinality of the signature. Fortunately, there exists a well-studied
construction in type theory that lets us assign to terms generic types
such as the one defined in the template above. This construction is the
\emph{dependent product} and its utility in ACGs has already been
noted.\footnote{See \cite{de2007two} for its introduction into the
  domain of ACGs, \cite{de2007type} for a small example grammar, which
  handles the interaction between number and gender agreement and
  coordination, and finally see \cite{pogodalla2012controlling} for
  using dependent types to handle several constraints regarding
  extraction (including the one we covered in~\ref{sec:extraction}).}

While extending the type system by including dependent types is
definitely useful and it helps to solve the problem of having our
signatures grow exponentially in size, it does not solve our problem
with the complexity of the new grammar. Granted, the new meta-type
written in the dependent style is more readable and easier to reason
about than the enumeration of 12 basic types we presented at the
beginning of this section, however, the result is still complex in that
it forces us to describe the behavior of three independent mechanisms in
one place. This one place happens to be a wordform which has seemingly
little connection to these mechanisms, the preposition \emph{de}. We can
therefore imagine that it will participate in many other phenomena and
that the type we assign to it will grow beyond our abilities to reason
about it effectively.

Furthermore, the current notation does not even make it obvious that
these are independent mechanisms. One has to examine the type to make
sure that there is no interaction.

TODO: Change this after splitting the chapters.

In the rest of this chapter, we will devote ourselves to investigating
ways of disentangling these.

\section{Intersections of Types}

Our goal is to split the types into smaller parts that only talk about
features belonging to one of the three mechanisms covered above. This
means that we would like to take some phrase about which we know a lot
of disparate pieces of information (presence of negative determiner,
trace status, number\ldots), having type
e.g. $NP\_\textcolor{red}{NEG{=}T}\_\textcolor{green}{VAR{=}F}\_\textcolor{blue}{NUM{=}PL}$,
and use it as an argument for some function which only cares about one
channel of information and would have a type like
$NP\_\textcolor{red}{NEG{=}T} \limp N\_\textcolor{red}{NEG{=}F} \limp
N\_\textcolor{red}{NEG{=}T}$. To make a type such as
$NP\_\textcolor{red}{NEG{=}T}\_\textcolor{green}{VAR{=}F}\_\textcolor{blue}{NUM{=}PL}$
be a legal argument to a function which expects an argument of type
$NP\_\textcolor{red}{NEG{=}T}$, we will need to introduce some notion of
\emph{subtyping}.

The types we have been showing up to now propose a very straightforward
notion of subtyping. All of the types can be seen as conjunctions of
properties where $\_$ (underscore) is the conjunction operator. A type
$\tau$ can then be seen as a subtype of type $\sigma$ whenever the
conjuncts of $\tau$ form a superset of the conjuncts of $\sigma$. This
is not unlike the structural typing of records in OCaml (CITE SOME OCAML
SPEC HERE), where a record type is a subtype of another record type if
the former provides, among others, all the fields provided by the latter
and gives them the same type. In our case, one type is a subtype of
another if the former gives, among others, all the properties given by
the latter. This notion of subtyping can be expressed by adding the rule
of inference in Figure~\ref{fig:type-superset} to our type
system.\footnote{In the rule \ldots (talk about $\land$ instead of $\_$
  (underscore) here...)}

\begin{figure}
  \begin{prooftree}
    \AxiomC{$\Gamma_l; \Gamma_i \vdash_\Sigma M : A_1 \land \ldots A_{i-1} \land A_i \land A_{i+1} \ldots \land A_n$}
    \RightLabel{(subtype)}
    \UnaryInfC{$\Gamma_l; \Gamma_i \vdash_\Sigma M : A_1 \land \ldots A_{i-1} \land A_{i+1} \ldots \land A_n$}
  \end{prooftree}
  \caption{\label{fig:type-superset} A rule of inference giving us
    subtyping for conjunctions of atomic types. For readability, we use
    $\land$ instead of $\_$ (underscore) to designate the conjunction of
    properties.}
\end{figure}

(Talk about whether or not this is a conservative extension...)

With the new subtyping rule, we can

$$
S_{tatou} : N \land NEG{=}F \land NUM{=}SG
$$
$$
S_{le_{NEG_1}} : (N \land NEG{=}F) \limp (((NP \land NEG{=}F) \limp S) \limp S)
$$
$$
S_{le_{NEG_2}} : (N \land NEG{=}T) \limp (((NP \land NEG{=}T) \limp S) \limp S)
$$
$$
S_{le_{NUM}} : (N \land NUM{=}SG) \limp (((NP \land NUM{=}SG) \limp S) \limp S)
$$
$$
S_{le_{VAR}} : N \limp (((NP \land VAR{=}F) \limp S) \limp S)
$$

\ldots


\chapter{Graphical Abstract Categorial Grammars}
\label{chap:gacg}

In this chapter, we will introduce a generalization of abstract
categorial grammars that gracefully solves the modularity issues that we
saw in the previous chapter.

\section{Intersections of Languages}
\label{sec:sects-of-langs}

Before we proceed to define our extension of ACGs, we will first need to
present the manner in which ACGs and their compositions are formulated
in ACG literature.

\subsection{The Usual Formalization of ACGs}
\label{ssec:usual-acgs}

Let us consider a system with three signatures $\Sigma_1$, $\Sigma_2$
and $\Sigma_3$, their respective distinguished types $S_1$, $S_2$ and
$S_3$ and two lexicons $\mathcal{L}_{12} : \Sigma_1 \to \Sigma_2$ and
$\mathcal{L}_{23} : \Sigma_2 \to \Sigma_3$ such that
$\mathcal{L}_{12}(S_1) = S_2$ and $\mathcal{L}_{23}(S_2) = S_3$. See
Figure~\ref{fig:acg-serial-comp} for a diagrammatic visualization of
this structure.

\begin{figure}[t]
  \centering
  \begin{subfigure}[b]{0.4\textwidth}
    \centering
    \includegraphics[height=0.2\textheight]{diagrams/acg-serial-comp.pdf}
    \caption{\label{fig:acg-serial-comp} Serial composition of ACGs.}
  \end{subfigure}
  \qquad
  \begin{subfigure}[b]{0.4\textwidth}
    \centering
    \includegraphics[height=0.2\textheight]{diagrams/acg-parallel-comp.pdf}
    \caption{\label{fig:acg-parallel-comp} Parallel composition of ACGs.}
  \end{subfigure}
  \caption{\label{fig:acg-comp-modes} Two modes of composition for
    ACGs.}
\end{figure}

We can now define languages $A_1$, $A_2$ and $A_3$ where $A_i$ is the
set of terms from $\Lambda(\Sigma_i)$ that have the type $S_i$. Such
languages, which are defined only in terms of a signature and some
distinguished type, are called \emph{abstract languages} in the
terminology of ACGs. They consist of terms whose structure is
constrained only by the types assigned by the signature.

Further than that, we can look at the languages $\mathcal{L}_{12}(A_1)$
and $\mathcal{L}_{23}(A_2)$, images of the abstract languages given by
the lexicons. Because of the homomorphism property of lexicons, all
elements of $\mathcal{L}_{12}(A_1)$ will have the type
$\mathcal{L}_{12}(S_1) = S_2$ and will therefore also belong to $A_2$
(similarly, we have that $\mathcal{L}_{23}(A_2) \subseteq A_3$). These
new languages, which are formed by mapping an ACG abstract language
using a lexicon, are called \emph{object languages}. Their terms are
constrained not only by the types of the signature they are built upon
but also by the type system of the abstract signature.

Finally, we can also take the composition of lexicons $\mathcal{L}_{13}
= \mathcal{L}_{23} \circ \mathcal{L}_{12}$, which itself is also a
lexicon, and map it over $A_1$. This gives us a language
$\mathcal{L}_{13}(A_1) = \mathcal{L}_{23}(\mathcal{L}_{12}(A_1))$ which
is at the same time the image of the object language
$\mathcal{L}_{12}(A_1)$ given by $\mathcal{L}_{23}$ and also the image
of the abstract language $A_1$ given by $\mathcal{L}_{13}$ (thus it is
itself also an object language). This points to an interesting property
of ACG languages: they are closed under transformation by a lexicon.

In the ACG literature, ACGs are defined as quadruples $\mathcal{G} =
\mathopen{<}\Sigma_A, \Sigma_O, \mathcal{L}, S\mathclose{>}$ where
$\Sigma_A$ is the abstract signature, $\Sigma_O$ is the object
signature, $\mathcal{L} : \Sigma_A \to \Sigma_O$ is a lexicon linking
the two and $S$ is a distinguished type of the abstract signature
$\Sigma_A$. From this grammar, we can define two languages. The abstract
language $\mathcal{A}(\mathcal{G}) = \{t \in \Lambda(\Sigma_A) \mid
\ \vdash_{\Sigma_A} t : S\}$ and the object language
$\mathcal{O}(\mathcal{G}) = \mathcal{L}(\mathcal{A}(\mathcal{G})) = \{t
\in \Lambda(\Sigma_O) \mid \exists u \in \mathcal{A}(\mathcal{G}),
\mathcal{L}(u) = t\}$. The composition of two ACGs $\mathcal{G}_1 =
\mathopen{<}\Sigma_1, \Sigma_2, \mathcal{L}_{12}, S_1\mathclose{>}$ and
$\mathcal{G}_2 = \mathopen{<}\Sigma_2, \Sigma_3, \mathcal{L}_{23},
S_2\mathclose{>}$ is defined as $\mathcal{G}_2 \circ \mathcal{G}_1 =
\mathopen{<}\Sigma_1, \Sigma_3, \mathcal{L}_{23} \circ \mathcal{L}_{12},
S_1\mathclose{>}$.

The definition of ACGs given above is quite practical in that it is
complete, i.e. it gives a precise account of what set of languages are
generated by ACGs (abstract and object languages for which there exist
the quadruples described above). This contrasts with the presentation of
ACGs that we gave in~\ref{sec:acg}, where we defined ACGs as collections
of signatures associated with distinguished types and connected with
lexicons. Our definition only introduced the requisite machinery, but
delayed the delimitation of the way languages can be defined. Leaving
this question open will let us answer it now by providing an extension
of ACGs which will help us overcome the modularity issues that have been
the theme of Chapter~\ref{chap:constraints}.\footnote{Incidentally, this
  was not the reason for the way we presented ACGs in~\ref{sec:acg}. We
  chose this style as it grouped together the information in a more
  natural way. We approach signatures as descriptions of domains
  (strings, syntax, semantics), each one being associated with a
  distinguished type telling us which objects the domain has set out to
  describe (strings, sentences, truth values). Having distinguished
  types directly attached to signatures allows us to omit them from the
  ACG quadruples, $\mathopen{<}\Sigma_A, \Sigma_O, \mathcal{L},
  \cancel{S}\mathclose{>}$. However, this only leaves us with triples
  containing the two signatures and the lexicon connecting them, and
  since we can infer the signatures themselves from the domain and the
  range of the lexicon $\mathcal{L} : \Sigma_A \to \Sigma_O$, we do not
  have much motivation to introduce these tuples,
  $\mathopen{<}\cancel{\Sigma_A}, \cancel{\Sigma_O}, \mathcal{L},
  \cancel{S}\mathclose{>}$. Instead of composing ACGs, we compose just
  the lexicons.

On top of that, we will want to define signatures and lexicons to yield
multiple different object languages (at least one for the string
representations of forms and one for the semantic representations of
meaning), which would mean having at least two different ACGs which are
somehow linked together (e.g. by sharing the same abstract
signature). However, since these two grammars are so interrelated, we
would like to reason about them in a way that makes these relationships
explicit and thus easier to study as opposed to considering a collection
of grammars which are somehow implicitly similar.}

\subsection{Patterns of Composition in ACGs}
\label{ssec:acg-patterns}

Before we turn to our proposal, we will quickly recall some of the other
styles of composition used in ACGs (a more detailed exposition can be
found in \cite{pogodalla2012controlling}). Besides making the object
signature of one grammar be the abstract signature of another as we saw
before (Figure~\ref{fig:acg-serial-comp}), we can have two grammars
share the same abstract signature allowing us to transduce between terms
in the two object signatures (Figure~\ref{fig:acg-parallel-comp}).

NB: \emph{Transduction} is a process in which mapping an object $o_1$ to
(potentially many) other objects $o_2$ is realized by first inverting
one function $f_1$ to find the object's antecedents $a$ and then
applying to them another function $f_2$, i.e. an object $o_1$ is
transduced to an object $o_2$ whenever there exists an $a$ such that
$f_1(a) = o_1$ and $f_2(a) = o_2$. In our example, terms from
$\mathcal{O}( \mathopen{<} \Sigma_1, \Sigma_2, \mathcal{L}_{12}, S
\mathclose{>} )$ are transduced to terms from $\mathcal{O}( \mathopen{<}
\Sigma_1, \Sigma_3, \mathcal{L}_{13}, S \mathclose{>} )$ by inverting
the lexicon $\mathcal{L}_{12}$ to find an antecedent in $\mathcal{A}(
\mathopen{<} \Sigma_1, \Sigma_2, \mathcal{L}_{12}, S \mathclose{>} )$
and then applying to it the lexicon $\mathcal{L}_{13}$.

\begin{figure}[t]
  \centering
  \begin{subfigure}[b]{0.4\textwidth}
    \centering
    \includegraphics[height=0.2\textheight]{diagrams/parallel-over-serial.pdf}
    \caption{\label{fig:parallel-over-serial} Transduction from an
      abstract signature.}
  \end{subfigure}
  \qquad
  \begin{subfigure}[b]{0.4\textwidth}
    \centering
    \includegraphics[height=0.2\textheight]{diagrams/serial-over-parallel.pdf}
    \caption{\label{fig:serial-over-parallel} Constraint on an abstract
      signature.}
  \end{subfigure}
  \caption{\label{fig:acg-comp-patterns} Mixing the ACG modes of
    composition.}
\end{figure}

These two modes of composition can be fruitfully mixed. Consider the
graph on Figure~\ref{fig:parallel-over-serial} in which we introduce a
syntactic signature $Syntax$ where different terms are only allowed to
yield the same string whenever they represent different parses of a
syntactically ambiguous phrase. This means that purely semantic
distinctions such as scope ambiguity have to be moved to a more abstract
signature ($SyntSem$) from whose terms the semantic representation can
be derived by a function ($\mathcal{L}_{interSem}$). This approach is
studied more deeply in \cite{pogodalla2007generalizing}.

Another interesting composition pattern is to take an existing structure
of grammars and add a new abstract signature on top
(Figure~\ref{fig:serial-over-parallel}). This new signature ($Constr$)
can be used to further constrain the items in the previously
abstract-most signature ($Synt$). In \cite{pogodalla2012controlling},
this is what the authors use to develop their constraints on
extraction. Each of their constraints is presented as a new abstract
signature constraining the original syntactic signature.

However, a question that the authors do not pose, but which is of great
interest to us, is how to combine all these constraints. If we were to
apply the same pattern repeatedly to get a structure like the one on
Figure~\ref{fig:stacked-constraints}, we would have to deal with
successively more complex type systems on each level. This is due to the
condition that a lexicon has to be a homomorphism and thus it must,
among other things, always map well-typed terms to well-typed
terms. Because of this, the terms in the abstract language of the
signature we introduce to handle the second constraint must also satisfy
the first constraint, since mapping them by $\mathcal{L}_{constr_2}$
must yield a well-typed term in the signature corresponding to the first
constraint. In the end, whenever we wish to add a new signature to
handle another constraint, we have to acknowledge and reimplement in it
all of the existing constraints. This effectively renders the pattern
useless since we might just as well consider the final abstract-most
signature handling all the constraints ($Constr_{1 \land 2}$) and attach
it directly to the syntactic signature ($Synt$) by taking the
composition of the lexicons in between ($\mathcal{L}_{constr_1} \circ
\mathcal{L}_{constr_2}$).

\begin{figure}[t]
  \centering
  \begin{subfigure}[b]{0.4\textwidth}
    \centering
    \includegraphics[height=0.2\textheight]{diagrams/stacked-constraints.pdf}
    \caption{\label{fig:stacked-constraints} Stacking the constraints
      vertically.}
  \end{subfigure}
  \qquad
  \begin{subfigure}[b]{0.4\textwidth}
    \centering
    \includegraphics[height=0.2\textheight]{diagrams/constraints-side-by-side.pdf}
    \caption{\label{fig:constraints-side-by-side} Laying out the
      constraints side by side.}
  \end{subfigure}
  \caption{Constraining a syntactic signature with multiple
    constraints.}
\end{figure}

What we would like instead is to put the signatures side by side such
that they do not depend on each other
(Figure~\ref{fig:constraints-side-by-side}). However, our language of
syntactic structures is no longer defined by mapping a single abstract
language through a lexicon. It is outside of the scope of the common
formalization of ACGs that we presented in~\ref{ssec:usual-acgs}. How do
we define this language in a manner that is consistent with our current
definitions of ACGs? We can arrive at the answer by examining how the
graphical representations of ACGs are connected to the languages they
define.

\subsection{Interpreting ACG Diagrams}

We will associate a language to every node of an ACG diagram. Before we
begin, we notice that every ACG diagram formed by the above modes of
composition is an arborescence\footnote{An arborescence is a directed
  rooted tree, where every edge points away from the root.} and
therefore every node has at most one predecessor (the parent). To the
root of the tree, the abstract-most signature in the graph, we will
assign the abstract language generated by that signature and its
distinguished type. To all the other nodes, we will assign the result of
mapping their parent's language by the lexicon linking the two. Thanks
to the homomorphism property of the lexicons, we can find distinguished
types for all the non-root nodes. They are simply the images of their
parents' distinguished types by the connecting lexicons. For all of the
nodes, it is true that every term in the language assigned to the node
has the node's distinguished type.

We can confirm that the above graphical model of defining ACGs is
coherent.

First, we can show that our interpretation is sound meaning that any
arborescence adorned with signatures on the nodes, compatible lexicons
on the edges and a distinguished type on the root can only define ACG
languages (abstract or object languages). This is trivial, since the
language corresponding to the root is an ACG abstract language and the
class of ACG languages is closed under transformation by a lexicon and
therefore all of the languages corresponding to the descendants of the
root are ACG languages as well.

Second, we can try and show some kind of completeness. However, there is
not much formalization done to describe a ``system of ACGs that share
some common signatures'' (something that our generalization will
fix). We can nevertheless show that it is possible to express any ACG
using our graphical model. Given a grammar $\mathcal{G} =
\mathopen{<}\Sigma_A, \Sigma_O, \mathcal{L}, S\mathclose{>}$, we just
produce a tree whose root has the signature $\Sigma_A$ and distinguished
type $S$ and which is connected to a single child node having the
signature $\Sigma_O$ by an edge labelled by the lexicon
$\mathcal{L}$. Then the two nodes of our tree define both the abstract
and the object language defined by $\mathcal{G}$. Similarly, we could
argue that the languages we aim to define using ACG diagrams (such as
the ones on Figures \ref{fig:acg-comp-modes} and
\ref{fig:acg-comp-patterns}) are covered by our assignment of languages
to nodes of the diagrams.

We have now given a formal account of how we can read language
definitions from ACG diagrams. Still, we have kept our analysis to the
arborescent diagrams used in current ACG literature, where every node
has at most one parent. We will now proceed to generalize it to cases
where nodes have more than just one parent.

To make the generalization more visible, we will switch from an
algebraic manner of presentation to a relational one. Before, we stated
that the language of a node is the image of its parent's language.

$$
Language(u) = Lexicon(Language(Parent(u)))
$$

Now, we will restate this as saying that the language of a node is the
set of terms that have an antecedent in the parent's language.

$$
Language(u) = \{x \in \Lambda(Signature(u)) \mid \exists a \in
Language(Parent(u)), \; Lexicon(a) = x\}
$$

This definition works for non-root nodes of an ACG diagram. In the root
case, we have to constrain the language by stating that that the terms
of the language must have the root's distinguished type. This is a
property that is also true in the non-root cases due to the homomorphism
property of lexicons. If we wanted to unify the definitions of the root
language and the non-root languages, we would have to check that a term
of the language has the distinguished type and that it has an antecedent
in the parent, if it has any parent. A relational statement of this
unification could look like this.

\begin{align*}
Language(u) = \{&x \in \Lambda(Signature(u)) \mid Type(x) = DistinguishedType(u) \\
&\land \forall p \in Parents(u), \; \exists a \in Language(p), \; Lexicon_{(p,u)}(a) = x\}
\end{align*}

If the node has no parents, no constraint is enforced. If it does have
some, we enforce the antecedent constraint. This shows us a natural
generalization to the case where nodes can have an arbitrary number of
parents. If we transpose this back to the algebraic style of
presentation, we get the following expression.

$$
Language(u) = DistinguishedlyTyped(u) \quad \cap \bigcap_{p \in Parents(u)} Lexicon_{(p,u)}(Language(p))
$$


\section{Definitions}

In this section, we distill the reasoning about generalizing ACG
diagrams to graphs in a definition of an extension for ACGs.

We define a \emph{graphical abstract categorial grammar} as a quadruple
$\mathcal{G} = \mathopen{<} G, \Sigma, S, {\mathcal{L}} \mathclose{>}$
where $G$ is a directed graph with vertices $V(G)$ and edges $E(G)$,
$\Sigma$ and $S$ are labelings assigning signatures and distinguished
types, respectively, to the vertices $V(G)$ and ${\mathcal{L}}$ is a
labeling assigning lexicons to the edges $E(G)$. Furthermore, a
well-formed graphical ACG satisfies the following conditions:

\begin{itemize}
\item $G$ is a directed acyclic graph.
\item For all $(u,v) \in E(G)$, $\mathcal{L}_{(u,v)} : \Sigma_u \to
  \Sigma_v$.
\item For all $(u,v) \in E(G)$, $\mathcal{L}_{(u,v)}(S_u) = S_v$.
\end{itemize}

We will say that a node $u$ is \emph{more abstract} than a node $v$
whenever $(u,v)$ belongs to the strict partial order induced by the
edges of $G$ (i.e. there is a path from $u$ to $v$).

We let the nodes in a graphical ACG $\mathcal{G} = \mathopen{<} G,
\Sigma, S, \mathcal{L} \mathclose{>}$ define two kinds of languages. The
\emph{intrinsic languages} $\mathcal{I}_{\mathcal{G}}$, which are
constrained only by the type signature described in the node itself, and
the \emph{extrinsic languages} $\mathcal{E}_{\mathcal{G}}$, which are
also constrained by type signatures in more abstract nodes of the
graph. These two notions correspond to those of abstract languages and
object languages in ACGs.

$$
\mathcal{I}_{\mathcal{G}}(v) = \{t \in \Lambda(\Sigma_v)
\mid\ \vdash_{\Sigma_v} t : S_v\}
$$
$$
\mathcal{E}_{\mathcal{G}}(v) = \mathcal{I}_{\mathcal{G}}(v) \cap
\bigcap_{(u,v) \in E} \mathcal{L}_{(u,v)}(\mathcal{E}_{\mathcal{G}}(u))
$$

From the above definition, we see that the ACG diagrams we introduced
before are a special case of graphical ACGs where the graph is an
arborescence.

\section{General Remarks on Graphical ACGs}

In this section, we then examine the key properties and implications of
graphical ACGs.

\subsection{On the Expressivity of Graphical ACGs}

\begin{observation}
  The set of intrinsic languages definable by graphical ACGs is by
  definition the same as the set of abstract languages definable by
  ACGs.
\end{observation}

\begin{theorem}
  The set of extrinsic languages definable by graphical ACGs is the set
  of object languages definable by ACGs closed under intersection and
  transformation by a lexicon.
\end{theorem}

\begin{proof}
  First, we prove that an extrinsic language can be always constructed
  from object languages by intersections and transformations by
  lexicons. Let us consider any graphical ACG $\mathcal{G} =
  \mathopen{<} G, \Sigma, S, \mathcal{L} \mathclose{>}$ and some
  topological ordering $v_1, \ldots, v_n$ of its nodes $V(G)$ (an
  ordering such that more abstract nodes always precede less abstract
  ones). Before we start, we will remark that all the intrinsic
  languages $\mathcal{I}_{\mathcal{G}}(v_i)$ are object languages, since
  intrinsic languages are abstract languages which are in turn just a
  special case of object languages.

  We proceed by induction on the topological ordering:

  \begin{itemize}
    \item For $v_1$, the case is trivial. $v_1$ has no predecessors and
      so $\mathcal{E}_{\mathcal{G}}(v_1) =
      \mathcal{I}_{\mathcal{G}}(v_1)$, which is an object language.
    \item For any other node $v_n$, we look at the definition of
      $\mathcal{E}_{\mathcal{G}}(v_n)$ and remark that the only
      operators are intersection and application of a lexicon with the
      operands being $\mathcal{I}_{\mathcal{G}}(v_n)$ (which is an
      object language) and the extrinsic languages of its predecessors
      (which can be constructed from object languages using
      intersections and lexicons thanks to the induction
      hypothesis). Therefore even $\mathcal{E}_{\mathcal{G}}(v_n)$ can
      be constructed from object languages using only intersections and
      lexicons.
  \end{itemize}


  Now we have to prove the converse. We do so by showing that extrinsic
  languages contain object languages and are closed under intersection
  and transformation by a lexicon.

  \begin{figure}
    \centering
    \begin{subfigure}[b]{0.2\textwidth}
      \centering
      \includegraphics[height=0.3\textheight]{diagrams/proof-object.pdf}
      \caption{\label{fig:proof-object} The case for object languages.}
    \end{subfigure}
    \quad
    \begin{subfigure}[b]{0.4\textwidth}
      \centering
      \includegraphics[height=0.3\textheight]{diagrams/proof-intersection.pdf}
      \caption{\label{fig:proof-intersection} The case for intersections.}
    \end{subfigure}
    \quad
    \begin{subfigure}[b]{0.3\textwidth}
      \centering
      \includegraphics[height=0.3\textheight]{diagrams/proof-lexicon.pdf}
      \caption{\label{fig:proof-lexicon} The case for transformations by
        a lexicon.}
    \end{subfigure}
    \caption{Figures demonstrating that extrinsic languages contain
      object languages closed on transformation by a lexicon and
      intersection.}
  \end{figure}

  \begin{itemize}
    \item Let $L$ be an object language defined by the ACG $\mathcal{G}
      = \mathopen{<} \Sigma_A, \Sigma_O, \mathcal{L},
      S\mathclose{>}$. We take the graph $G$ with $V(G) = \{A, O\}$ and
      $E(G) = \{(A,O)\}$, we label the nodes $A$ and $O$ with $\Sigma_A$
      and $\Sigma_O$ as signatures and $S$ and $\mathcal{L}(S)$ as
      distinguished types respectively . Finally, we label the edge
      $(A,O)$ with $\mathcal{L}$ and we get a graphical ACG $G'$ in
      which $\mathcal{E}_{\mathcal{G'}}(O) = \mathcal{O}(G) = L$
      (Figure~\ref{fig:proof-object}).

    \item Let $L_1$ and $L_2$ be two extrinsic languages defined by the
      nodes $u_1$ and $u_2$ in some graphical ACGs $\mathcal{G}_1$ and
      $\mathcal{G}_2$, respectively. We can construct a new graphical
      ACG $\mathcal{G}'$ by taking the union of the two ACGs (union of
      graphs (union of vertices and edges) and unions of the
      labelings).\footnote{Assuming, without loss of generality, that
        the sets of vertices of the two graphs are disjoint.} We will
      then add a new node $v$ such that $\mathcal{E}_{\mathcal{G}'}(v) =
      L_1 \cap L_2$ (Figure~\ref{fig:proof-intersection}).

      The signature labeling $v$ will be the union of the two signatures
      labeling $u_1$ and $u_2$, ensuring that $\forall u \in \{u_1,
      u_2\}$, $\mathcal{I}_{\mathcal{G}'}(v) \supseteq
      \mathcal{I}_{\mathcal{G}'}(u) \supseteq
      \mathcal{E}_{\mathcal{G}'}(u)$. We can use either of the two
      distinguished types that label $u_1$ or $u_2$ to label $v$ (if
      they are different, then the intersection $L_1 \cap L_2$ is
      trivially empty). Finally, we will add two edges leading from
      $u_1$ and $u_2$ to $v$, both labeled with the identity lexicon. We
      will now show that $\mathcal{E}_{\mathcal{G}'}(v) = L_1 \cap L_2$.

      \begin{align*}
        \mathcal{E}_{\mathcal{G}'}(v) &= \mathcal{I}_{\mathcal{G}'}(v)
        \cap \mathcal{L}_{(u_1,v)}(\mathcal{E}_{\mathcal{G}'}(u_1)) \cap
        \mathcal{L}_{(u_2,v)}(\mathcal{E}_{\mathcal{G}'}(u_2)) \\ &=
        \mathcal{I}_{\mathcal{G}'}(v) \cap
        \mathcal{E}_{\mathcal{G}'}(u_1) \cap
        \mathcal{E}_{\mathcal{G}'}(u_2) \\ &=
        \mathcal{E}_{\mathcal{G}'}(u_1) \cap
        \mathcal{E}_{\mathcal{G}'}(u_2) \\ &=
        \mathcal{E}_{\mathcal{G}_1}(u_1) \cap
        \mathcal{E}_{\mathcal{G}_2}(u_2) \\ &= L_1 \cap L_2
      \end{align*}

      The simplifications use the facts that: the two new lexicons are
      identities, that $\mathcal{I}_{\mathcal{G}'}(v) \supseteq
      \mathcal{E}_{\mathcal{G}'}(u_1) \cap
      \mathcal{E}_{\mathcal{G}'}(u_2)$, that
      $\mathcal{E}_{\mathcal{G}'}(u_1) =
      \mathcal{E}_{\mathcal{G}_1}(u_1)$ and
      $\mathcal{E}_{\mathcal{G}'}(u_2) =
      \mathcal{E}_{\mathcal{G}_2}(u_2)$ and finally that
      $\mathcal{E}_{\mathcal{G}}(u_1) = L_1$ and
      $\mathcal{E}_{\mathcal{G}}(u_2) = L_2$.

  \item Let $L \subseteq \Lambda(\Sigma_O)$ be a language which is the
    result of mapping an extrinsic language $L' \subseteq
    \Lambda(\Sigma_A)$ through the lexicon $\mathcal{L} : \Sigma_A \to
    \Sigma_O$. We will show that $L$ is also an extrinsic language. Let
    $u$ be the node that defines the extrinsic language $L$ in some
    graphical ACG $\mathcal{G}$. We construct a new graphical ACG
    $\mathcal{G}'$ by adding a new vertex $v$ to $\mathcal{G}$ labeled
    with the signature $\Sigma_O$ and the distinguished type
    $\mathcal{L}(S)$ where $S$ is the distinguished type labeling
    $u$. To this graph, we conjoin the edge $(u,v)$ and label it with
    $\mathcal{L}$ (Figure~\ref{fig:proof-lexicon}).

    Now we verify that $\mathcal{E}_{\mathcal{G}'}(v) = L$.

    \begin{align*}
      \mathcal{E}_{\mathcal{G}'}(v) &= \mathcal{I}_{\mathcal{G}'}(v) \cap
      \mathcal{L}(\mathcal{E}_{\mathcal{G}'}(u)) \\
      &= \mathcal{L}(\mathcal{E}_{\mathcal{G}'}(u)) \\
      &= \mathcal{L}(\mathcal{E}_{\mathcal{G}}(u)) \\
      &= \mathcal{L}(L') \\
      &= L
    \end{align*}

    The assumptions used here are in turn: that
    $\mathcal{I}_{\mathcal{G}'}(v) \supseteq
    \mathcal{L}(\mathcal{E}_{\mathcal{G}'}(u))$, that
    $\mathcal{E}_{\mathcal{G}'}(u) = \mathcal{E}_{\mathcal{G}}(u)$, that
    $\mathcal{E}_{\mathcal{G}}(u) = L'$ and that $\mathcal{L}(L') = L$.
  \end{itemize}
\end{proof}

\begin{corollary}
  The set of extrinsic languages definable by graphical ACGs is the same
  as the set of object languages definable by ACGs if and only if the
  set of object languages definable by ACGs is closed under
  intersection.
\end{corollary}

We conclude this subsection with a few words to motivate the need we
feel for this (possibly) more expressive formalism. We believe that in
order to manage the development of wide-coverage grammars, the grammars
need to be kept as simple as possible, which means, among other things,
that independent constraints should be implemented separately and not
complected together (we saw the difficulties caused by this already on a
microscopic fragment in section~\ref{sec:together}). This means that if
we were to define, e.g., a partial grammar in which we would only
enforce agreement and another grammar in which we would only enforce the
proper use of relative pronouns, we would like the language in which
both agreement and proper use of relative pronouns are respected to be
expressible in our formalism, and on top of that, we would like this
language to be easily expressed in terms of the two grammars for the
individual constraints. We now have a requirement on the formalism in
which we would like to use: we want a formalism where the set of all the
``useful'' definable languages is closed on intersection. If the object
languages of ACGs are indeed closed on intersection, then our extension
is just a convenience for expressing intersection more easily. However,
if they are not, we believe our extension adds a critical level of
expressivity needed for simple and modular wide-coverage grammars.


\subsection{On the Decidability of Graphical ACGs}
\label{ssec:graphical-decidability}

We turn to the problem of parsing with graphical ACGs. Parsing with ACGs
is a problem on the limits of decidability. When we consider ACGs whose
abstract signatures only contain types of order 2 or less, parsing is
possible in polynomial time for both linear \cite{salvati2005problemes}
and non-linear \cite{kanazawa2007parsing} type systems (such as the one
we introduced in~\ref{sec:acg}). If the order of the abstract signature
exceeds 2, the membership problem for linear ACGs becomes decidable if
and only if the satisfiability problem in the multiplicative exponential
fragment of linear logic is decidable \cite{de2004vector}, which is an
open problem. On the other hand, by extending the type system to include
dependent types, parsing can become undecidable too since in such a
system type checking itself is undecidable.

Our intent here is to show that by extending the formalism we have not
made the problem of parsing more difficult, at least in terms of
decidability if not in terms of complexity. Let us therefore imagine a
graphical ACG where every edge (two signatures, distinguished types and
a lexicon) is an ACG for which there is an effective parsing algorithm
and type checking is decidable in all the signatures of the graphical
ACG. Then we can prove the existence of a simple effective parsing
algorithm for any of the extrinsic languages defined in the graphical
ACG.

We will be testing whether the term $o$ belongs to
$\mathcal{E}_{\mathcal{G}}(v)$. Our algorithm first checks whether $o$
belongs to $\mathcal{I}_{\mathcal{G}}(v)$ by checking its type. If it
does, it proceeds to verify that it is also in
$\mathcal{E}_{\mathcal{G}}(v)$ by finding antecedents in all of its
abstract predecessors $u_i$, if there are any. We can use the
presupposed parsing algorithm for the ACG $\mathopen{<} \Sigma_{u_i},
\Sigma_v, \mathcal{L}_{(u_i,v)}, S_{u_i}\mathclose{>}$ to find all the
abstract terms $\alpha_{i,j}$ having type $S_{u_i}$ and being
antecedents to $o$ w.r.t. the lexicon $\mathcal{L}_{(u_i,v)}$. Such
terms belong to $\mathcal{I}_{\mathcal{G}}(u_i)$, but in order to prove
that $o$ is in $\mathcal{E}_{\mathcal{G}}(v)$, we need to find an
antecedent that belongs to $\mathcal{E}_{\mathcal{G}}(u_i)$. What we do
is we recursively apply our parsing algorithm to all of the
antecedent-candidates $\{\alpha_{i,j} \mid \forall j\}$ until we find
one which belongs $\mathcal{E}_{\mathcal{G}}(u_i)$. Since the graph is
acyclic, we are guaranteed that this recursive traversal will
terminate. We repeat this for all the predecessors $u_i$ and if we
succeed to find antecedents in all of their extrinsic languages, we have
confirmed the membership of $o$ in $\mathcal{E}_{\mathcal{G}}(v)$ and
also built its abstract antecedent(s).


\subsection{On Grammar Engineering with Graphical ACGs}
\label{ssec:graphical-engineering}

Having covered some of the formal properties of graphical ACGs, we will
get back to building grammars. Armed with this formalism, we can solve
the puzzle of incorporating the three constraints of sections
\ref{sec:negation}, \ref{sec:extraction} and \ref{sec:agreement}. Our
graph $G$ will consist of 6 nodes, $V(G) = \{Neg, Ext, Agr, Synt, Str,
Sem\}$ with edges going from $Neg$, $Ext$ and $Agr$ to $Synt$ and from
$Synt$ to $Str$ and $Sem$ (see Figure~\ref{fig:puzzle-solution}). The
signatures and distinguished types for the first three vertices are
exactly those that have been presented in the first three sections of
Chapter~\ref{chap:constraints}. The lexicons for the edges connecting
these constraint nodes to $Synt$ all behave according to the scheme
below:

\begin{align*}
  \mathcal{L}_{(constraint,Synt)}(X_{constant_{index}}) &= C_{constant} \\
  \mathcal{L}_{(constraint,Synt)}(TYPE\_FEATURES)  &= TYPE
\end{align*}

The $\{Synt, Str, Sem\}$ subgraph is identical to the pair of ACGs
described in \ref{sec:acg} except for some additions that need to be
made so that the fragment covers the wordforms that we have introduced
in the course of discussing the three constraints (prepositions,
relative pronouns, the negative particle $ne$\ldots). However, the types
of these new constants will not be tied to any of those constraints.

\begin{align*}
  C_{de} &: NP \limp N \limp N \\
  C_{qui} &: (NP \limp S) \limp N \limp N \\
  C_{dit\ que} &: NP \limp S \limp S \\
  C_{dorment} &: NP \limp S \\
  C_{ne_{tv}} &: (NP \limp NP \limp S) \limp (NP \limp NP \limp S) \\
  C_{ne_{iv}} &: (NP \limp S) \limp (NP \limp S) \\
  C_{ne_{cv}} &: (NP \limp S \limp S) \limp (NP \limp S \limp S)
\end{align*}

This approach lets us describe the high-level combinatorics of language
that are of interest to the categorial grammarian in a manner which is
idiomatic in academic literature (using the simple and focused types
$N$, $NP$ and $S$). At the same time, we can actually use this textbook
grammar as a basis for a more realistic grammar which handles
constraints such as agreement and extraction islands. Furthermore, these
additional constraints are also defined in a manner that has its
precedent in existing research. When the authors of
\cite{pogodalla2012controlling} decided to put the constraints of
extraction under a microscope and try to encode them in ACGs, they
defined their constraints in exactly the same way as our architecture
does. This graphical architecture lets us write the syntactic kernel and
the additional constraints independently and using two different styles
that have both been shown useful by their own merits, which we see as
indicative of enabling good modularity.

\begin{figure}[t]
  \centering
  \begin{subfigure}[b]{0.4\textwidth}
    \centering
    \includegraphics[height=0.2\textheight]{diagrams/puzzle-solution.pdf}
    \caption{{\label{fig:puzzle-solution} The graph of a grammar
        enforcing all of the constraints discussed in
        Chapter~\ref{chap:constraints}.}}
  \end{subfigure}
  \qquad
  \begin{subfigure}[b]{0.4\textwidth}
    \centering
    \includegraphics[height=0.2\textheight]{diagrams/both-patterns.pdf}
    \caption{{\label{fig:both-patterns} Combining the two ACG patterns
        from~\ref{ssec:acg-patterns} in a single graphical ACG.}}
  \end{subfigure}
  \caption{Practical architectures for graphical ACGs.}
\end{figure}

In the above approach, we have defined a language by writing a grammar
which provides us with useful structures but which does not enforce
complete correctness. We have then refined this grammar by tacking on
constraints ``on the side''. This approach is generally useful and can
be seen, e.g., in \cite{van2004concepts} to define the formal
programming language kernel on which a formal semantics is
established. In our case, we are interested in having intelligible
syntactic structures too as we would like to compute their semantic
representations. The syntax-semantics interface already presents a
significant challenge and having less-than-ideal syntactic structures to
start with does not make formalizing it any easier.

We will show one more persuasive scenario whose goal is to show the
importance of graphical ACGs for having a clean syntax-semantics
interface. In \ref{ssec:acg-patterns}, we have seen an ACG pattern that
put semantic ambiguities into an abstract signature that is higher than
syntax. But there was also another pattern, that of adding a new
abstract signature to constrain syntactic terms. We can combine both of
them in graphical ACGs by superposing their two graphs to get the
structure we can see on Figure~\ref{fig:both-patterns}.

If we try to reap the benefits of both patterns and try to replicate the
same in classical ACGs (arborescent graphs), we hit a snag: we cannot
have both the $Constr$ and $SyntSem$ nodes as the direct predecessors of
the $Syntax$ node. We can make $Constr$ be the predecessor of $Syntax$,
but then $SyntSem$ will be connecting $Constr$ and $Sem$ and the
syntax-semantics interface will have to deal with the types which were
needed to implement syntactic constraints and have no place in a
syntax-semantics interface. If we try it the other way, it is not much
better. $Constr$ will end up having to work with the more complicated
type system of $SyntSem$ instead of being able to talk only about the
$Syntax$ it is interested in.

By putting the constraining signature in a part of the graph unreachable
from the syntax-semantics transducer, we can shield the syntax-semantics
interface from the gnarly implementation details of the rules of
syntax. This in turn gives us a guideline as to what information should
be expressed in the syntactic signature itself and which should be
hidden in constraints: the syntactic signature should only contain
information relevant to the syntax-semantics interface.

This might then raise a question whether the chosen syntactic signature
will end up being suitable for expressing all of the syntactic
constraints. However, this can be resolved by deepening the constraint
subgraph and adding a ``compatibility layer'' in the form of a new
signature which provides different syntactic structures and maps them to
the canonical ones. Constraints can then opt in to controlling the
syntactic structures offered by this new signature.


\section{On Alternative Interpretations of Graphical ACGs}

Our definitions of intrinsic and extrinsic languages have consequences
which might be counter-intuitive and not correspond to the grammarian's
intent. We will describe what these consequences are and propose an
alternative notion of a graphical ACG language which aims to reconcile
them.

The first phenomenon that deserves pointing out occurs whenever two
paths from nodes $u_1$ and $u_2$ converge on a single node $v$. If we
then take a term $\alpha$ from $\mathcal{E}_{\mathcal{G}}(u_1)$, the
term $\beta = \mathcal{L}_{(u_1,v)}(\alpha)$ belongs to
$\mathcal{I}_{\mathcal{G}}(v)$ but not necessarily to
$\mathcal{E}_{\mathcal{G}}(v)$ since it might be the case that there is
no antecedent for $\beta$ in $\mathcal{E}_{\mathcal{G}}(u_2)$. In other
words, mapping a member of $\mathcal{E}_{\mathcal{G}}(u)$ using the
lexicon $\mathcal{L}_{(u,v)}$ does not guarantee that the result belongs
to $\mathcal{E}_{\mathcal{G}}(v)$, which is a different behavior from
that of object languages in ACGs (or that of extrinsic languages in
arborescent graphical ACGs).

This highlights some limitations of our notion of an extrinsic
language. Consider once more the graphical ACG of
Figure~\ref{fig:both-patterns}. We have more than one path converging to
$Syntax$. We will consider a term $\alpha$ of
$\mathcal{E}_{\mathcal{G}}(Sem)$. $\alpha$ has an antecedent in
$\mathcal{E}_{\mathcal{G}}(SyntSem)$ which has a descendant $\beta$ in
$\mathcal{I}_{\mathcal{G}}(Syntax)$. However, because of the constraint
$Constr$ on $Syntax$, $\beta$ can be a syntactic term outside of
$\mathcal{E}_{\mathcal{G}}(Syntax)$ and yielding an ungrammatical
sentence. The notion of an extrinsic language is insufficient to express
the set of meanings from $\Lambda(\Sigma_{Sem})$ which are expressible
only by grammatical sentences. Defining languages like this would be of
particular interest in cases like the one in
Figure~\ref{fig:transducer-grammar} where we have a chain of signatures
representing the different levels of linguistic description whose terms
are in many-to-many relationships expressed through transducers. In such
a scenario, it would be desirable to be able to express the set of
descriptions belonging to one level, e.g. morphology, that correspond to
valid descriptions on all the other levels.

\begin{figure}[t]
  \centering
  \begin{subfigure}[b]{0.4\textwidth}
    \centering
    \includegraphics[width=\textwidth]{diagrams/transducer-grammar.pdf}
    \caption{{\label{fig:transducer-grammar} A graphical ACG connecting
        (constrained) levels of linguistic description through
        transducers.}}
  \end{subfigure}
  \qquad
  \begin{subfigure}[b]{0.4\textwidth}
    \centering
    \includegraphics[height=0.2\textheight]{diagrams/diamond-grammar.pdf}
    \caption{{\label{fig:diamond-grammar} A graphical ACG demonstrating
        two different paths between two nodes.}}
  \end{subfigure}
  \caption{Graphical ACGs posing challenges to extrinsic languages.}
\end{figure}

Compared to ACG object languages (or to extrinsic languages of
arborescent graphical ACGs), we lose another guarantee. In
arborescences, the central role of the root abstract-most signature
guarantees us that whenever a term belongs to an extrinsic language, we
can find terms in all the other extrinsic languages that share the same
abstract-most term. On the other hand, in non-arborescent graphical
ACGs, having a term in an extrinsic language of some node $v$ only
guarantees the presence of corresponding terms in the extrinsic
languages of its ancestors and the intrinsic languages of their
descendants.

The second puzzling scenario is the diamond shape on
Figure~\ref{fig:diamond-grammar}. If we consider a term $\delta$ of
$\mathcal{E}_{\mathcal{G}}(Bottom)$, we know that there exist
antecedents $\beta$ and $\gamma$ in $\mathcal{E}_{\mathcal{G}}(Left)$
and $\mathcal{E}_{\mathcal{G}}(Right)$. And while there definitely exist
antecedents $\alpha$ and $\alpha'$ to both $\beta$ and $\gamma$ in
$\mathcal{E}_{\mathcal{G}}(Top)$, these antecedents can be different,
meaning there does not have to be a single $\alpha \in
\mathcal{E}_{\mathcal{G}}(Top)$ which yields $\beta \in
\mathcal{E}_{\mathcal{G}}(Left)$ and $\gamma \in
\mathcal{E}_{\mathcal{G}}(Right)$ such that they both yield
$\delta$. This means that the result of parsing a term in some
$\mathcal{E}_{\mathcal{G}}(v)$ is a collection of abstract terms with
one term not just per every node $u$ that is more abstract than $v$ but
per every different path leading to $v$. Parsing a term $\delta \in
\mathcal{E}_{\mathcal{G}}(Bottom)$ in our diamond grammar would thus
produce four antecedents $\beta$, $\gamma$, $\alpha$ and $\alpha'$
corresponding to the four paths $[Left,Bottom]$, $[Right,Bottom]$,
$[Top,Left,Bottom]$ and $[Top,Right,Bottom]$, respectively. We contrast
this again with the case of classical ACGs where the result of parsing
can always be represented by associating a term to every node of the
arborescence.\footnote{One could also say that the result of parsing in
  an ACG is just the antecedent in the abstract-most signature and that
  the other terms can then be easily computed by application of
  lexicons. Similarly, we could say that the result of parsing in a
  general graphical ACG is a collection of abstract terms for every
  \emph{maximal} path leading to the node defining the language in
  question. What might be more intuitive, though, is to have some theory
  of graphical ACGs where the result of parsing is a collection of
  antecedents in all the abstract-most nodes.}

We will propose a new way of assigning languages to nodes in a graphical
ACG that will conserve the intuitive properties of ACGs we talked about
above. We do not need to modify the definition of a graphical ACG in
order to do so since the definition is only concerned with the structure
of the graph and its decoration with ACG paraphernalia, not the
languages defined. We will define a third kind of languages based on
graphical ACGs.

Let $\mathcal{G} = \mathopen{<} G, \Sigma, S, \mathcal{L} \mathclose{>}$
be a graphical ACG. We define the \emph{pangraphical language} of node
$u$, $\mathcal{P}_{\mathcal{G}}(u)$, to be the set of terms $t$ such
that there exists a labeling of nodes with terms $T$ satisfying the
following conditions:

\begin{itemize}
  \item $T_u = t$.
  \item For all $v \in V(G)$, $\vdash_{\Sigma_v} T_v : S_v$.
  \item For all $(v,w) \in E(G)$, $\mathcal{L}_{(v,w)}(T_v) = T_w$.
\end{itemize}

It is easy to see that the newly introduced pangraphical languages solve
all of the counter-intuitive quirks of extrinsic languages that we
discussed in this subsection.

\begin{itemize}
\item Applying the lexicon $\mathcal{L}_{(u,v)}$ to an element of
  $\mathcal{P}_{\mathcal{G}}(u)$ always yields an element of
  $\mathcal{P}_{\mathcal{G}}(v)$, since the labeling of nodes with terms
  $T$ that proved the former term's membership in the former language
  also proves the latter term's membership in the latter language.

  This definition allows us to directly establish the language of
  meanings which are expressible only by grammatical sentences in
  Figure~\ref{fig:both-patterns} and the language of morphological terms
  that have valid corresponding representations on all the other levels
  of linguistic description in Figure~\ref{fig:transducer-grammar}.
\item The diamond grammar of Figure~\ref{fig:diamond-grammar} is also no
  longer problematic since the labeling of nodes with terms guarantees
  that there is always a single term per node that can generate the term
  being parsed.
\end{itemize}

In the previous section, we took a stab at formalizing ACG diagrams and
finding the simplest extension capable of handling non-arborescent
graphs. The notion of extrinsic languages that we arrived at allowed us
to define a nicely delimited set of languages, the object languages of
ACGs with intersection. Discrepancies in the behavior of extrinsic
languages in arborescent and non-arborescent graphs led us to
pangraphical languages, which behave more in line with our
intuitions. However, the expressivity of pangraphical languages,
especially w.r.t. to extrinsic languages, still remains to be
explored. Due to limitations of both space and time, we will cover the
interesting properties of pangraphical languages only briefly and
partially here.

\begin{description}
  \item[Extrinsic languages can be defined similarly to pangraphical
    languages] \hfill \\

    In the definition of $\mathcal{P}_{\mathcal{G}}(u)$, we replace the
    labeling of nodes with terms by a labeling of all the paths leading
    to $u$.

  \item[$\mathcal{P}_{\mathcal{G}}(u) \subseteq
    \mathcal{E}_{\mathcal{G}}(u) \subseteq
    \mathcal{I}_{\mathcal{G}}(u)$] \hfill \\

    Given a labeling of nodes with terms such that it satisfies the
    conditions in the definition of a pangraphical language, we can show
    by induction on the topological ordering of the graph that every
    term assigned by the labeling belongs to the extrinsic language of
    the same node. Therefore, $\alpha \in \mathcal{P}_{\mathcal{G}}(u)
    \implies \alpha \in \mathcal{E}_{\mathcal{G}}(u)$.

    $\mathcal{E}_{\mathcal{G}}(u) \subseteq
    \mathcal{I}_{\mathcal{G}}(u)$ comes straight from the definition of
    $\mathcal{E}_{\mathcal{G}}(u)$.

    We have thus came up with a series of successively more constrained
    languages: intrinsic languages are constrained by one node,
    extrinsic languages are also constrained by its ancestors and
    pangraphical languages are constrained by the entire graph.

  \item[On arborescences, $\mathcal{P}_{\mathcal{G}}(u) =
    \mathcal{E}_{\mathcal{G}}(u)$.] \hfill \\

    On an arborescence, being a member of an extrinsic language means
    there exists an antecedent in the abstract-most language of the
    arborescence. By virtue of the one-parent property of arborescences,
    we can simply apply all the lexicons in the arborescence to this
    abstract-most antecedent to find a labeling of nodes with terms
    which proves that the member of the extrinsic language is also a
    member of the pangraphical language.

    This confirms that both extrinsic languages and pangraphical
    languages are reasonable generalizations of ACG diagrams since their
    behaviors on arborescences are consistent with each other and with
    that of ACG object languages.

  \item[The set of extrinsic languages is included in the set of
    pangraphical languages.] \hfill \\

    Let us have some node $u$ in a graphical ACG $\mathcal{G}$. We build
    a graphical ACG $\mathcal{G}'$ where every node corresponds to a
    path in $\mathcal{G}$ which ends at $u$ and two nodes are connected
    with an edge whenever one path is an immediate extension of the
    other. We then have $\mathcal{P}_{\mathcal{G'}}([u]) =
    \mathcal{E}_{\mathcal{G}}(u)$.

  \item[Is the set of pangraphical languages included in the set of
    extrinsic languages?] \hfill \\

    That is an interesting question indeed, one whose resolution is left
    as a possible avenue for future work.
\end{description}


\chapter{In Conclusion}
\label{chap:conclusion}

In this chapter, we first discuss some the consequences of using G-ACGs
for writing grammars. We then summarize the contributions that we have
put forward and suggest problems we believe to be interesting.

\section{On the Practical Consequences of Graphical ACGs}
\label{sec:practical}

We will enumerate some of the practical benefits and challenges of using
graphical ACGs for implementing grammar-based systems.

\begin{itemize}
\item The lexicons that go from our constraint signatures ($Neg$, $Ext$
  and $Agr$ in subsection \ref{ssec:graphical-engineering}) are all
  trivial in the sense they map constants to constants. Finding an
  antecedent to a term is then just a question of finding an antecedent
  for every constant in the term (i.e. ``tagging'' the constants of the
  term with antecedent constants). This means that there is an effective
  parsing algorithm for verifying the constraints. Furthermore, this
  algorithm could also be made efficient by introducing filtering
  techniques \cite{guillaume2008toolchain} and statistical supertagging
  \cite{moot2012categorial}.

\item Constraints can introduce type extensions in their abstract
  signatures and the added complexity of these extensions is then
  contained within the constraint. Furthermore, the constraint signature
  can be treated as a ``black box'' that verifies a constraint and in
  the context of an implementation, it can be replaced with a
  special-purpose hand-written procedure that does the same more
  efficiently.

\item The distributed nature of graphical ACGs has its own practical
  applications. Parsing can be made more robust by being able to parse
  sentences which violate some of the constraints. Parsing an
  ungrammatical sentence and identifying the constraints it violates can
  also serve as a basis for a syntax checker.

\item If we decide to define all of our syntactic constraints in
  separate signatures, we will end up with a lot of signatures which
  look exactly like the syntactic signature but with some small
  refinements (see the signatures in sections \ref{sec:negation},
  \ref{sec:extraction} and \ref{sec:agreement}). Thus we will desire
  some concise way of formulating these refinements so that we will not
  need to replicate the entire syntactic signature for every
  constraint.

  We hope that the class of useful refinements is small and that these
  constraint signatures could be systematically constructed from the
  syntactic signature by applying a series of refining
  transformations. This is an interesting engineering challenge awaiting
  anyone willing to design a truly wide-coverage graphical ACG using
  constraints.
\end{itemize}


\section{Conclusion}
\label{sec:conclusion}

The contributions of our work can be summarized in the following three
points:

\begin{enumerate}
\item We have highlighted the challenges inherent in translating an
  existing grammar in the formalism of IGs to the formalism of
  ACGs. Based on the breadth of different constraints built in to the IG
  formalism, we have conjectured that a direct translation of IG lexical
  items to ACG lexical items based on the deep underlying similarity
  between the two formalisms is not practical.

\item We have implemented a system for experimenting with (lexicalized)
  abstract categorial grammars that can serve as the basis for
  formulating wide-coverage graphical ACGs
  (\url{https://github.com/jirkamarsik/acg-clj}).

\item We have presented the graphical abstract categorial grammars
  (G-ACGs) which generalize ACGs. We have determined the set of
  languages definable by G-ACGs in terms of ACG languages and we have
  shown how G-ACGs allow us to write categorial grammars that handle
  multiple disparate linguistic constraints without sacrificing
  simplicity.
\end{enumerate}

Here are some of the items that we believe warrant future investigation:

\begin{itemize}
\item Try to express the multiple independent polarized features of IGs
  using G-ACGs.
\item Determine whether ACG object languages are closed on intersection
  or not.
\item Make a closer examination of the relationship between extrinsic
  and pangraphical languages in G-ACGs.
\item Write grammars using the G-ACG constraint architecture to verify
  its applicability and test the claims of modularity we have made here.
\item Study the problem of concisely defining constraint signatures from
  syntactic signatures and rules of feature propagation.
\end{itemize}




%%%%%%%%%%%%%%%%%%%%%%%%%%%%%% 
%% 附录(章节编号重新计算,使用字母进行编号)
%%%%%%%%%%%%%%%%%%%%%%%%%%%%%% 
\appendix

% 附录中编号形式是"A-1"的样子
\renewcommand\theequation{\Alph{chapter}--\arabic{equation}}
\renewcommand\thefigure{\Alph{chapter}--\arabic{figure}}
\renewcommand\thetable{\Alph{chapter}--\arabic{table}}

%%%==================================================
%% app1.tex for SJTU Master Thesis
%% based on CASthesis
%% modified by wei.jianwen@gmail.com
%% version: 0.3a
%% Encoding: UTF-8
%% last update: Dec 5th, 2010
%%==================================================

\chapter{模板更新记录}
\label{chap:updatelog}

\textbf{2013年5月26日} v0.5.3发布,更正subsubsection格式错误,这个错误导致如"1.1 小结"这样的标题没有被正确加粗。

\textbf{2012年12月27日} v0.5.2发布,更正拼写错误:从``个人建立''更正为``个人简历''。在diss.tex加入ack.tex,更名后忘了引用。

\textbf{2012年12月21日} v0.5.1发布,在 \LaTeX 命令和中文字符之间留了空格,在Makefile中增加release功能。

\textbf{2012年12月5日} v0.5发布,修改说明文件的措辞,更正Makefile文件,使用metalog宏包替换xltxtra宏包,使用mathtools宏包替换amsmath宏包,移除了所有CJKtilde(\verb+~+)符号。

\textbf{2012年5月30日} v0.4发布,包含交大学士、硕士、博士学位论文模板。模板在\href{https://github.com/weijianwen/sjtu-thesis-template-latex}{github}上管理和更新。

\textbf{2010年12月5日} v0.3a发布,移植到 \XeTeX/\LaTeX 上。

\textbf{2009年12月25日} v0.2a发布,模板由CASthesis改名为sjtumaster。在diss.tex中可以方便地改变正文字号、切换但双面打印。增加了不编号的一章“全文总结”。
添加了可伸缩符号(等号、箭头)的例子,增加了长标题换行的例子。

\textbf{2009年11月20日} v0.1c发布,增加了Linux下使用ctex宏包的注意事项、.bib条目的规范要求,
修正了ctexbook与listings共同使用时的断页错误。

\textbf{2009年11月13日} v0.1b发布,完善了模板使用说明,增加了定理环境、并列子图、三线表格的例子。

\textbf{2009年11月12日} 上海交通大学硕士学位论文 \LaTeX 模板发布,版本0.1a。

 % 更新记录
%%% app2.tex for SJTU Master Thesis
%% based on CASthesis
%% modified by wei.jianwen@gmail.com
%% version: 0.3a
%% Encoding: UTF-8
%% last update: Dec 5th, 2010
%%==================================================

\chapter{Maxwell Equations}

选择二维情况,有如下的偏振矢量
\begin{subequations}
  \begin{eqnarray}
    {\bf E}&=&E_z(r,\theta)\hat{\bf z} \\
    {\bf H}&=&H_r(r,\theta))\hat{ \bf r}+H_\theta(r,\theta)\hat{\bm
      \theta}
  \end{eqnarray}
\end{subequations}
对上式求旋度
\begin{subequations}
  \begin{eqnarray}
    \nabla\times{\bf E}&=&\frac{1}{r}\frac{\partial E_z}{\partial\theta}{\hat{\bf r}}-\frac{\partial E_z}{\partial r}{\hat{\bm\theta}}\\
    \nabla\times{\bf H}&=&\left[\frac{1}{r}\frac{\partial}{\partial
        r}(rH_\theta)-\frac{1}{r}\frac{\partial
        H_r}{\partial\theta}\right]{\hat{\bf z}}
  \end{eqnarray}
\end{subequations}
因为在柱坐标系下,$\overline{\overline\mu}$是对角的,所以Maxwell方程组中电场$\bf
E$的旋度
\begin{subequations}
  \begin{eqnarray}
    &&\nabla\times{\bf E}=\mathbf{i}\omega{\bf B} \\
    &&\frac{1}{r}\frac{\partial E_z}{\partial\theta}{\hat{\bf
        r}}-\frac{\partial E_z}{\partial
      r}{\hat{\bm\theta}}=\mathbf{i}\omega\mu_rH_r{\hat{\bf r}}+\mathbf{i}\omega\mu_\theta
    H_\theta{\hat{\bm\theta}}
  \end{eqnarray}
\end{subequations}
所以$\bf H$的各个分量可以写为:
\begin{subequations}
  \begin{eqnarray}
    H_r=\frac{1}{\mathbf{i}\omega\mu_r}\frac{1}{r}\frac{\partial
      E_z}{\partial\theta } \\
    H_\theta=-\frac{1}{\mathbf{i}\omega\mu_\theta}\frac{\partial E_z}{\partial r}
  \end{eqnarray}
\end{subequations}
同样地,在柱坐标系下,$\overline{\overline\epsilon}$是对角的,所以Maxwell方程组中磁场$\bf
H$的旋度
\begin{subequations}
  \begin{eqnarray}
    &&\nabla\times{\bf H}=-\mathbf{i}\omega{\bf D}\\
    &&\left[\frac{1}{r}\frac{\partial}{\partial
        r}(rH_\theta)-\frac{1}{r}\frac{\partial
        H_r}{\partial\theta}\right]{\hat{\bf
        z}}=-\mathbf{i}\omega{\overline{\overline\epsilon}}{\bf
      E}=-\mathbf{i}\omega\epsilon_zE_z{\hat{\bf z}} \\
    &&\frac{1}{r}\frac{\partial}{\partial
      r}(rH_\theta)-\frac{1}{r}\frac{\partial
      H_r}{\partial\theta}=-\mathbf{i}\omega\epsilon_zE_z
  \end{eqnarray}
\end{subequations}
由此我们可以得到关于$E_z$的波函数方程:
\begin{eqnarray}
  \frac{1}{\mu_\theta\epsilon_z}\frac{1}{r}\frac{\partial}{\partial r}
  \left(r\frac{\partial E_z}{\partial r}\right)+
  \frac{1}{\mu_r\epsilon_z}\frac{1}{r^2}\frac{\partial^2E_z}{\partial\theta^2}
  +\omega^2 E_z=0
\end{eqnarray}
 % 麦克斯韦方程
% \include{body/app3}


%%%%%%%%%%%%%%%%%%%%%%%%%%%%%% 
%% 文后(无章节编号)
%%%%%%%%%%%%%%%%%%%%%%%%%%%%%% 
\backmatter

% 参考文献
% 使用 BibTeX
% 包含参考文献文件.bib
\bibliography{../biblio}

%% 个人简历(硕士学位论文没有个人简历要求)
% \newpage

\pagenumbering{gobble}

\begin{description}[style=nextline]
\item[Résumé]

Le but final de notre travail est la création d'une grammaire
catégorielle abstraite (ACG) à large échelle qu'on pourra utiliser pour
construire les représentations au niveau de discours d'une manière
automatique.

On commence par examiner les ressources linguistiques qui existent déjà,
en particulier la grammaire d'interaction Frigram et son lexique Frilex,
et on examine leur utilité pour la construction d'une ACG à large
échelle. Ensuite, on présente une implémentation des mécanismes ACG qui
nous permet d'expérimenter avec des grammaires lexicalisées par
Frilex. Enfin, on considère les problèmes posés par le traitement des
plusieurs contraintes linguistiques dans une seule ACG et on propose une
généralisation du formalisme: les grammaires catégorielles abstraites
graphiques.

\item[Mots-clés]

Grammaires catégorielles abstraites (ACG), Génie des grammaires,
Grammaires d'interaction, Metagrammaires, Programmation relationnelle,
Formalismes grammaticaux, Grammaires formelles, Linguistique
computationnelle, Lambda-calcul, Théorie des types
\end{description}

\vfill

\begin{description}[style=nextline]
\item[Summary]

We present work whose ultimate goal is the creation of a wide-coverage
abstract categorial grammar (ACG) that could be used to automatically
build discourse-level representations.

We first examine existing language resources, in particular the Frigram
interaction grammar and its lexicon Frilex, and assess their utility to
building a wide-coverage ACG. We then present our implementation of the
ACG machinery which allows us to experiment with grammars lexicalized by
Frilex. Finally, we consider the challenge of integrating the treatment
of disparate linguistic constraints in a single ACG and propose a
generalization of the formalism: graphical abstract categorial
grammars.

\item[Keywords]

Abstract Categorial Grammars (ACG), Grammar Engineering, Interaction
Grammars, Metagrammars, Relational Programming, Grammatical Formalisms,
Formal Grammars, Computational Linguistics, Lambda Calculus, Type Theory
\end{description}


% 致谢
%%==================================================
%% thanks.tex for SJTU Master Thesis
%% based on CASthesis
%% modified by wei.jianwen@gmail.com
%% version: 0.3a
%% Encoding: UTF-8
%% last update: Dec 5th, 2010
%%==================================================

\begin{thanks}

  I thank the European Union and the Erasmus Mundus Language and Communication
  Technologies Master program for funding my studies.

  I thank my supervisors, Maxime Amblard in Nancy and Zhao Hai in Shanghai,
  for their leadership and assistance.

  I thank my colleague Sai Qian for helping me with the Chinese translations.

  I thank my friends in Shanghai and Nancy for two wonderful years.

  I thank my parents in supporting me in everything I chose to do.

  Thank you!

\end{thanks}


% 发表文章目录
%%==================================================
%% pub.tex for SJTU Master Thesis
%% based on CASthesis
%% modified by wei.jianwen@gmail.com
%% version: 0.3a
%% Encoding: UTF-8
%% last update: Dec 5th, 2010
%%==================================================

\begin{publications}{99}

    \item\textsc{Marsik, Jiri and Amblard, Maxime}. {Integration of Multiple Constraints in ACG}[C].
      Logic and Engineering of Natural Language Semantics 10, 2013.

\end{publications}


% 参与项目列表
%%==================================================
%% projects.tex for SJTU Master Thesis
%% based on CASthesis
%% modified by wei.jianwen@gmail.com
%% version: 0.3a
%% Encoding: UTF-8
%% last update: Dec 5th, 2010
%%==================================================

\begin{projects}{99}

    \item Erasmus Mundus Language and Communication Technologies Master Program
    
\end{projects}


\end{document}
