\documentclass{beamer}

\usepackage{color}
\usepackage{xcolor}

\definecolor{darkgreen}{rgb}{0., 0.6, 0.}
\definecolor{darkgold}{rgb}{8., 0.6, 0.}
\definecolor{limegreen}{RGB}{175, 255, 175}
\definecolor{limeblue}{RGB}{200, 200, 235}

\hypersetup{pdfstartview={Fit}}
\def\limp {\mathbin{{-}\mkern-3.5mu{\circ}}}

% http://tex.stackexchange.com/questions/33401/a-version-of-colorbox-that-works-inside-math-environments
\newcommand{\highlight}[2]{\colorbox{#1}{$\displaystyle #2$}}

\AtBeginSection[]
{
  \begin{frame}
    \frametitle{Outline}
    \tableofcontents[currentsection]
    \addtocounter{framenumber}{-1}
  \end{frame}
}

\setbeamertemplate{navigation symbols}{}
\setbeamertemplate{footline}
  {\hfill {\normalsize \insertframenumber{}/\inserttotalframenumber{}}}


\begin{document}

\title[G-ACGs]{Towards a Wide-Coverage Grammar}
\subtitle{Graphical Abstract Categorial Grammars}
\author{Ji\v{r}\'{i} Mar\v{s}\'{i}k}
\institute[INRIA Loria]
{
  INRIA Loria
  \and
  UFR Math\'{e}matiques et Informatiques \\
  Universit\'{e} de Lorraine
}
\date[June 2013]{July 11, 2013}

\frame{\titlepage \setcounter{framenumber}{1}}

\begin{frame}
\frametitle{Outline}
\tableofcontents
\end{frame}

\section{Introduction}

\begin{frame}
  \frametitle{Towards a Wide-Coverage Grammar}

  \textbf{Objective:} Build discourse representations using ACGs: \\a
  wide-coverage ACG grammar.

  \vfill

  \begin{itemize}
  \item Having linguistic resources to build upon
    \vfill
  \item Debugging the grammar
    \vfill
  \item Using a formalism that allows for a simple grammar
    \begin{itemize}
    \item Express multiple independent constraints \\ $\implies$ graphical abstract categorial grammars
    \end{itemize}
  \end{itemize}
\end{frame}


\section{ACG Preliminaries}

\newcommand{\synt}[1]{C_{\textrm{#1}}}

\begin{frame}
  \begin{block}{Abstract languages}
    Lambda terms constrained by the types of constants. \\ Defined by a
    \textcolor{red}{signature} and a \textcolor{darkgreen}{distinguished
      type}.

    \begin{columns}[t]
      \begin{column}{0.5\textwidth}
        \begin{align*}
          \synt{tatous} &: \textcolor{red}{N} \\
          \synt{Laura} &: \textcolor{red}{NP}
        \end{align*}
      \end{column}
      \begin{column}{0.5\textwidth}
        \begin{align*}
          \synt{les} &: \textcolor{red}{N \limp NP} \\
          \synt{aime} &: \textcolor{red}{NP \limp NP \limp S}
        \end{align*}
      \end{column}
    \end{columns}

    $$\synt{aime}\ \synt{Laura}\ (\synt{les}\ \synt{tatous}) :
    \textcolor{darkgreen}{S}$$
  \end{block}

  \begin{block}{Object languages}
    Languages of (object) terms which are obtained by replacing the
    constants in an existing (abstract) term using a \textcolor{darkgold}{lexicon}.

    \begin{columns}[t]
      \begin{column}{0.5\textwidth}
        \begin{align*}
          \mathcal{L}(\synt{tatous}) &= \textcolor{darkgold}{tatous} \\
          \mathcal{L}(\synt{Laura}) &= \textcolor{darkgold}{Laura}
        \end{align*}
      \end{column}
      \begin{column}{0.5\textwidth}
        \begin{align*}
          \mathcal{L}(\synt{les}) &= \textcolor{darkgold}{\lambda^{\circ} x.\ les + x} \\
          \mathcal{L}(\synt{aime}) &= \textcolor{darkgold}{\lambda^{\circ} x y.\ x + aime + y}
        \end{align*}
      \end{column}
    \end{columns}

    $$\mathcal{L}(\synt{aime}\ \synt{Laura}\ (\synt{les}\ \synt{tatous}))
    =_{\beta} Laura + aime + les + tatous$$
  \end{block}
\end{frame}


\begin{frame}
  \frametitle{ACG Diagrams}

  \begin{center}
    \includegraphics[height=0.8\textheight]{../diagrams/double-acg.pdf}
  \end{center}
\end{frame}


\begin{frame}
  \frametitle{ACG Patterns}

  \begin{columns}[c]
    \begin{column}{0.5\textwidth}
      \begin{block}{Semantic ambiguities}
        \vspace{2 mm}
        \begin{center}
          \includegraphics[height=0.7\textheight]{../diagrams/parallel-over-serial.pdf}
        \end{center}
      \end{block}
    \end{column}
    \begin{column}{0.5\textwidth}
      \begin{block}{Syntactic constraint}
        \vspace{2 mm}
        \begin{center}
          \includegraphics[height=0.7\textheight]{../diagrams/serial-over-parallel.pdf}
        \end{center}
      \end{block}
    \end{column}
  \end{columns}
\end{frame}


\section{Why Graphical ACGs?}

\begin{frame}
  \frametitle{Composing ACG Patterns I}

  ACG diagrams are arborescences.

  \begin{columns}[c]
    \begin{column}{0.25\textwidth}
      {\small Syntactic constraints deal with the syntax-semantics
        interface.}

      \onslide<2->{$\implies$ Complex.}
    \end{column}
    \begin{column}{0.25\textwidth}
      \begin{center}
        \includegraphics[height=0.7\textheight]{../diagrams/fail2.pdf}
      \end{center}
    \end{column}
    \begin{column}{0.25\textwidth}
      {\small Syntax-semantics interface deals with syntactic
        constraints.}

      \onslide<3->{$\implies$ Complex.}
    \end{column}
    \begin{column}{0.25\textwidth}
      \begin{center}
        \includegraphics[height=0.7\textheight]{../diagrams/fail1.pdf}
      \end{center}
    \end{column}
  \end{columns}
\end{frame}


\begin{frame}
  \frametitle{Composing ACG Patterns II}

  Writing a simple wide-coverage grammar implies specifying constraints
  independently.

  \begin{columns}[c]
    \begin{column}{0.25\textwidth}
      Second constraint reimplements first constraint.

      \onslide<2->{$\implies$ Ineffective.}
    \end{column}
    \begin{column}{0.25\textwidth}
      \begin{center}
        \includegraphics[height=0.7\textheight]{../diagrams/constr-fail-1.pdf}
      \end{center}
    \end{column}
    \begin{column}{0.25\textwidth}
      Third constraint reimplements first and second constraint.

      \onslide<3->{$\implies$ Ineffective.}
    \end{column}
    \begin{column}{0.25\textwidth}
      \begin{center}
        \includegraphics[height=0.7\textheight]{../diagrams/constr-fail-2.pdf}
      \end{center}
    \end{column}
  \end{columns}
\end{frame}


\begin{frame}
  \frametitle{Graphical Abstract Categorial Grammars (G-ACGs)}

  \begin{columns}[c]
    \begin{column}{0.6\textwidth}
      \begin{center}
        \includegraphics[height=0.7\textheight]{../diagrams/final.pdf}
      \end{center}
    \end{column}

    \pause

    \begin{column}{0.4\textwidth}
        \begin{itemize}
        \item A DAG.
        \item Nodes labelled with:
          \begin{itemize}
          \item Signatures.
          \item Distinguished types.
          \end{itemize}
        \item Edges labelled with:
          \begin{itemize}
          \item Lexicons.
          \end{itemize}
        \end{itemize}
    \end{column}
  \end{columns}
\end{frame}


\section{Semantics of G-ACGs}

\begin{frame}
  \frametitle{Algebraic Intuition}

  \begin{columns}[c]
    \begin{column}{0.7\textwidth}
      \setbeamercolor{block title}{bg=limeblue}
      \setbeamercolor{block body}{bg=limeblue}
      \begin{block}{}
        \begin{itemize}
        \item Every node is a language.
        \item An arrow means that a language is produced from another by
          mapping through a \textcolor{darkgreen}{lexicon}.
        \item Two arrows converging on a node mean an
          \textcolor{red}{intersection} of languages.
        \end{itemize}
      \end{block}

$$
\mathcal{I}_{\mathcal{G}}(v) = \{t \in \Lambda(\Sigma_v)
\mid\ \vdash_{\Sigma_v} t : S_v\}
$$
$$
\mathcal{E}_{\mathcal{G}}(v) = \mathcal{I}_{\mathcal{G}}(v) \cap
\textcolor{red}{\bigcap_{(u,v) \in E}}
\textcolor{darkgreen}{\mathcal{L}_{(u,v)}}(\mathcal{E}_{\mathcal{G}}(u)).
$$
    \end{column}
    \begin{column}{0.3\textwidth}
      \includegraphics[width=\textwidth]{../diagrams/algebraic-perspective.pdf}
    \end{column}
  \end{columns}

\end{frame}


\begin{frame}
  \frametitle{Generative Intuition}

  \begin{columns}[c]
    \begin{column}{0.7\textwidth}
\setbeamercolor{block title}{bg=limegreen}
      \setbeamercolor{block body}{bg=limegreen}
      \begin{block}{}
        \begin{itemize}
        \item Every node is a \textcolor{red}{term}.
        \item An arrow means that a term is produced from another by
          applying a \textcolor{darkgreen}{lexicon}.
        \item Two arrows converging on a node mean that the two (abstract)
          terms yield \textcolor{red}{the same} (object) term.
        \end{itemize}
      \end{block}

      \vspace{3 mm}

      $t$ belongs to $\mathcal{P}_{\mathcal{G}}(u)$ iff \\$\exists T :
      \textcolor{red}{V(G) \to \Lambda}$ such that:
      \begin{itemize}
      \item $T_u = t$.
      \item For all $v \in V(G)$, $\vdash_{\Sigma_v} T_v : S_v$.
      \item For all $(v,w) \in E(G)$,
        $\textcolor{darkgreen}{\mathcal{L}_{(v,w)}(T_v) = T_w}$.
      \end{itemize}
    \end{column}
    \begin{column}{0.3\textwidth}
      \includegraphics[width=\textwidth]{../diagrams/generative-perspective.pdf}
    \end{column}
  \end{columns}
\end{frame}


\begin{frame}
  \frametitle{Algebraic and Generative Relations}

  \begin{itemize}
  \item $\forall \, \mathcal{G}, u$ where $\mathcal{G}$ is arborescent:
    $$\highlight{limeblue}{\mathcal{E}_{\mathcal{G}}(u)}
    = \highlight{limegreen}{\mathcal{P}_{\mathcal{G}}(u)}
    = \mathcal{O}_{\mathcal{G}}(u)$$
    \pause
  \item However, $\exists \, \mathcal{G}, u$ such that
    $$\highlight{limeblue}{\mathcal{E}_{\mathcal{G}}(u)}
       \neq \highlight{limegreen}{\mathcal{P}_{\mathcal{G}}(u)}$$
    \vfill
    \pause
  \item[$\Rightarrow$] They are not interchangeable points of view.
  \end{itemize}
\end{frame}


\begin{frame}
  \frametitle{Properties of G-ACG languages}

  \begin{itemize}
  \item The three language families we defined above are\ldots
    \begin{itemize}
    \item {\ldots}increasingly constraining, $\forall \, \mathcal{G},
      u. \; \mathcal{I}_{\mathcal{G}}(u) \supseteq
      \mathcal{E}_{\mathcal{G}}(u) \supseteq
      \mathcal{P}_{\mathcal{G}}(u)$.
    \item {\ldots}increasingly expressive, $\mathcal{I} \subseteq
      \mathcal{E} \subseteq \mathcal{P}$.
    \end{itemize}
  \vfill
  \item Intrinsic languages are just the abstract languages of ACGs.
    $$\mathcal{I} = \mathcal{A}$$
  \vfill
  \item Extrinsic languages are simple extensions of ACGs that add
    intersections.
    $$\mathcal{E} = \mathcal{O}^{\mathcal{L}{\cap}}$$
  \vfill
  \item Pangraphical languages are faithful to the generative
    perspective and subsume the previous classes of languages.
  \end{itemize}
\end{frame}

\end{document}
