\documentclass{beamer}

\usepackage[utf8]{inputenc}
\usepackage{xcolor}
\usepackage{pgfpages}

\definecolor{darkgreen}{rgb}{0., 0.6, 0.}
\definecolor{darkgold}{rgb}{8., 0.6, 0.}
\definecolor{limegreen}{RGB}{175, 255, 175}
\definecolor{limeblue}{RGB}{200, 200, 235}

\hypersetup{pdfstartview={Fit}}
\def\limp {\mathbin{{-}\mkern-3.5mu{\circ}}}

\AtBeginSection[]
{
  \begin{frame}
    \frametitle{Outline}
    \tableofcontents[currentsection]
    \addtocounter{framenumber}{-1}
  \end{frame}
}

\setbeamertemplate{navigation symbols}{}
\setbeamertemplate{footline}
  {\hfill {\normalsize \insertframenumber{}/\inserttotalframenumber{}}}


\begin{document}

\title[G-ACGs]{Towards a Wide-Coverage Grammar}
\subtitle{Graphical Abstract Categorial Grammars}
\author{Jiří Maršík}
\institute[INRIA Loria]
{
  Department of Computer Science \& Engineering \\
  Shanghai Jiao Tong University
  \and
  INRIA Loria
}
\date[December 2013]{December 26, 2013}

\frame{\titlepage \setcounter{framenumber}{1}}

\begin{frame}
\frametitle{Outline}
\tableofcontents
\end{frame}

\section{Introduction}

\begin{frame}
  \frametitle{Towards a Wide-Coverage Grammar}

  \textbf{Objective:} Build discourse representations using ACGs: \\a
  wide-coverage ACG grammar.

  \vfill

  \begin{itemize}
  \item Having linguistic resources to build upon
    \begin{itemize}
    \item Frilex, a formalism-independent lexical database
    \item Frigram, a French interaction grammar
    \end{itemize}
    \vfill
  \item Debugging the grammar
    \begin{itemize}
    \item A toolkit for formulating ACGs as simple metagrammars, for
      examining how they work and for checking their consistency.
    \end{itemize}
    \vfill
  \item Using a formalism that allows for a simple grammar
    \begin{itemize}
    \item Express multiple independent constraints \\ $\implies$ graphical abstract categorial grammars.
    \end{itemize}
  \end{itemize}
\end{frame}


\begin{frame}
  \frametitle{ACGs and IGs}

  \begin{itemize}
  \item ACGs are based on types (IILL formulas).
  \item IGs are based on polarized tree descriptions.
    \vfill
  \item[$\Rightarrow$] Perrier has linked the two.

    \vspace{2 mm}
    \begin{columns}[c]
      \begin{column}{0.1\textwidth}\end{column}
      \begin{column}{0.4\textwidth}
        $$(NP \limp S) \limp N \limp N$$
        $$N \limp (NP \limp S) \limp N$$
      \end{column}
      \begin{column}{0.1\textwidth}$\to$\end{column}
      \begin{column}{0.2\textwidth}
        \includegraphics[height=0.3\textheight]{../diagrams/ptd.pdf}
      \end{column}
      \begin{column}{0.25\textwidth}\end{column}
    \end{columns}
    \vfill
  \item IG of Frigram use different constraints in tree descriptions.
    \begin{itemize}
    \item The deductive syntactic model of IGs is built on a wider range
      of linear logic primitives ($\otimes$, $\&$).
    \end{itemize}
  \end{itemize}
\end{frame}


\section{ACG Preliminaries}

\newcommand{\synt}[1]{C_{\textrm{#1}}}

\begin{frame}
  \begin{block}{Abstract languages}
    Defined by a \textcolor{red}{signature} and a
    \textcolor{darkgreen}{distinguished type}.

    \begin{columns}[t]
      \begin{column}{0.5\textwidth}
        \begin{align*}
          \synt{tatous} &: \textcolor{red}{N} \\
          \synt{Laura} &: \textcolor{red}{NP}
        \end{align*}
      \end{column}
      \begin{column}{0.5\textwidth}
        \begin{align*}
          \synt{les} &: \textcolor{red}{N \limp NP} \\
          \synt{aime} &: \textcolor{red}{NP \limp NP \limp S}
        \end{align*}
      \end{column}
    \end{columns}

    $$\synt{aime}\ \synt{Laura}\ (\synt{les}\ \synt{tatous}) :
    \textcolor{darkgreen}{S}$$
  \end{block}

  \begin{block}{Object languages}
    A projection of an abstract language with a \textcolor{darkgold}{lexicon}.

    \begin{columns}[t]
      \begin{column}{0.5\textwidth}
        \begin{align*}
          \mathcal{L}(\synt{tatous}) &= \textcolor{darkgold}{tatous} \\
          \mathcal{L}(\synt{Laura}) &= \textcolor{darkgold}{Laura}
        \end{align*}
      \end{column}
      \begin{column}{0.5\textwidth}
        \begin{align*}
          \mathcal{L}(\synt{les}) &= \textcolor{darkgold}{\lambda^{\circ} x.\ les + x} \\
          \mathcal{L}(\synt{aime}) &= \textcolor{darkgold}{\lambda^{\circ} x y.\ x + aime + y}
        \end{align*}
      \end{column}
    \end{columns}

    $$\mathcal{L}(\synt{aime}\ \synt{Laura}\ (\synt{les}\ \synt{tatous}))
    =_{\beta} Laura + aime + les + tatous$$
  \end{block}
\end{frame}


\begin{frame}
  \frametitle{ACG Diagrams}

  \begin{center}
    \includegraphics[height=0.8\textheight]{../diagrams/double-acg.pdf}
  \end{center}
\end{frame}


\begin{frame}
  \frametitle{ACG Patterns}

  \begin{columns}[c]
     \begin{column}{0.5\textwidth}
      \begin{block}{Constraint}
        \vspace{2 mm}
        \begin{center}
          \includegraphics[height=0.7\textheight]{../diagrams/serial-over-parallel-slides.pdf}
        \end{center}
      \end{block}
    \end{column}
    \begin{column}{0.5\textwidth}
      \begin{block}{Elaboration}
        \vspace{2 mm}
        \begin{center}
          \includegraphics[height=0.7\textheight]{../diagrams/parallel-over-serial-slides.pdf}
        \end{center}
      \end{block}
    \end{column}
 \end{columns}
\end{frame}


\section{Why Graphical ACGs?}

\begin{frame}
  \frametitle{Composing ACG Patterns I}

  \begin{columns}[c]
    \begin{column}{0.5\textwidth}
      \begin{center}
        \includegraphics[height=0.7\textheight]{../diagrams/fail1.pdf}
      \end{center}
    \end{column}
    \begin{column}{0.5\textwidth}
      \begin{center}
        \includegraphics[height=0.7\textheight]{../diagrams/fail2.pdf}
      \end{center}
    \end{column}
  \end{columns}
\end{frame}


\begin{frame}
  \frametitle{Composing ACG Patterns II}

  \begin{columns}[c]
    \begin{column}{0.5\textwidth}
      \begin{center}
        \includegraphics[height=0.7\textheight]{../diagrams/constr-fail-1.pdf}
      \end{center}
    \end{column}
    \begin{column}{0.5\textwidth}
      \begin{center}
        \includegraphics[height=0.7\textheight]{../diagrams/constr-fail-2.pdf}
      \end{center}
    \end{column}
  \end{columns}
\end{frame}


\begin{frame}
  \frametitle{Multiple Constraints}
  Agreement:
    \begin{align*}
      A_{aime_1} &: NP_{\textcolor{darkgold}{NUM{=}SG}} \limp NP_{\textcolor{darkgold}{NUM{=}SG}} \limp S \\
      A_{aime_2} &: NP_{\textcolor{darkgold}{NUM{=}SG}} \limp NP_{\textcolor{darkgold}{NUM{=}PL}} \limp S
    \end{align*}

  Extraction:
    \begin{align*}
      E_{aime_1} &: NP_{\textcolor{darkgreen}{VAR{=}F}} \limp NP_{\textcolor{darkgreen}{VAR{=}F}} \limp S_{\textcolor{darkgreen}{EXT{=}NO}} \\
      E_{aime_2} &: NP_{\textcolor{darkgreen}{VAR{=}T}} \limp NP_{\textcolor{darkgreen}{VAR{=}F}} \limp S_{\textcolor{darkgreen}{EXT{=}ROOT}} \\
      E_{aime_3} &: NP_{\textcolor{darkgreen}{VAR{=}F}} \limp NP_{\textcolor{darkgreen}{VAR{=}T}} \limp S_{\textcolor{darkgreen}{EXT{=}ROOT}
}    \end{align*}

  Combined:
  \begin{small}
  \begin{align*}
      C_{aime_1} &: NP_{\textcolor{darkgold}{NUM{=}SG},\textcolor{darkgreen}{VAR{=}F}} \limp NP_{\textcolor{darkgold}{NUM{=}SG},\textcolor{darkgreen}{VAR{=}F}} \limp S_{\textcolor{darkgreen}{EXT{=}NO}} \\
      C_{aime_2} &: NP_{\textcolor{darkgold}{NUM{=}SG},\textcolor{darkgreen}{VAR{=}T}} \limp NP_{\textcolor{darkgold}{NUM{=}SG},\textcolor{darkgreen}{VAR{=}F}} \limp S_{\textcolor{darkgreen}{EXT{=}ROOT}} \\
      C_{aime_3} &: NP_{\textcolor{darkgold}{NUM{=}SG},\textcolor{darkgreen}{VAR{=}F}} \limp NP_{\textcolor{darkgold}{NUM{=}SG},\textcolor{darkgreen}{VAR{=}T}} \limp S_{\textcolor{darkgreen}{EXT{=}ROOT}} \\
      C_{aime_4} &: NP_{\textcolor{darkgold}{NUM{=}SG},\textcolor{darkgreen}{VAR{=}F}} \limp NP_{\textcolor{darkgold}{NUM{=}PL},\textcolor{darkgreen}{VAR{=}F}} \limp S_{\textcolor{darkgreen}{EXT{=}NO}} \\
      C_{aime_5} &: NP_{\textcolor{darkgold}{NUM{=}SG},\textcolor{darkgreen}{VAR{=}T}} \limp NP_{\textcolor{darkgold}{NUM{=}PL},\textcolor{darkgreen}{VAR{=}F}} \limp S_{\textcolor{darkgreen}{EXT{=}ROOT}} \\
      C_{aime_6} &: NP_{\textcolor{darkgold}{NUM{=}SG},\textcolor{darkgreen}{VAR{=}F}} \limp NP_{\textcolor{darkgold}{NUM{=}PL},\textcolor{darkgreen}{VAR{=}T}} \limp S_{\textcolor{darkgreen}{EXT{=}ROOT}
}  \end{align*}
  \end{small}
\end{frame}


\begin{frame}
  \frametitle{Graphical Abstract Categorial Grammars (G-ACGs)}

  \begin{columns}[c]
    \begin{column}{0.6\textwidth}
      \begin{center}
        \includegraphics[height=0.7\textheight]{../diagrams/final-slides.pdf}
      \end{center}
    \end{column}

    \begin{column}{0.4\textwidth}
        \begin{itemize}
        \item DAG
        \item Nodes:
          \begin{itemize}
          \item Signatures
          \item Distinguished types
          \end{itemize}
        \item Edges:
          \begin{itemize}
          \item Lexicons
          \end{itemize}
        \end{itemize}
    \end{column}
  \end{columns}
\end{frame}



\section{Semantics of G-ACG Nodes}

\begin{frame}
  \frametitle{Two Intuitions for G-ACGs}

  \setbeamercolor{block title}{bg=limeblue}
  \setbeamercolor{block body}{bg=limeblue}
  \begin{block}{Algebraic Perspective}
    \begin{itemize}
    \item Nodes as \emph{languages}
    \end{itemize}
  \end{block}
  \setbeamercolor{block title}{bg=limegreen}
  \setbeamercolor{block body}{bg=limegreen}
  \begin{block}{Generative Perspective}
    \begin{itemize}
    \item Nodes as \emph{terms}
    \end{itemize}
  \end{block}
\end{frame}


\begin{frame}
  \frametitle{Algebraic Intuition}

  \begin{columns}[c]
    \begin{column}{0.7\textwidth}
      \setbeamercolor{block title}{bg=limeblue}
      \setbeamercolor{block body}{bg=limeblue}
      \begin{block}{}
        \begin{itemize}
        \item Node = Language
        \item Edge = Projection with a \textcolor{darkgreen}{lexicon}
        \item Two incoming edges = \textcolor{red}{Intersection} of languages
        \end{itemize}
      \end{block}

$$
\mathcal{I}_{\mathcal{G}}(v) = \{t \in \Lambda(\Sigma_v)
\mid\ \vdash_{\Sigma_v} t : S_v\}
$$
$$
\mathcal{E}_{\mathcal{G}}(v) = \mathcal{I}_{\mathcal{G}}(v) \cap
\textcolor{red}{\bigcap_{(u,v) \in E}}
\textcolor{darkgreen}{\mathcal{L}_{(u,v)}}(\mathcal{E}_{\mathcal{G}}(u)).
$$
    \end{column}
    \begin{column}{0.3\textwidth}
      \includegraphics[width=\textwidth]{../diagrams/algebraic-perspective.pdf}
    \end{column}
  \end{columns}

\end{frame}


\begin{frame}
  \frametitle{Two Problems with the Algebraic View}

  \begin{columns}[t]
    \begin{column}{0.5\textwidth}
      \begin{center}
        \includegraphics[height=0.6\textheight]{../diagrams/both-patterns-slides.pdf}
      \end{center}

      How do we get the language of meanings expressible only by
      syntactically correct sentences?
    \end{column}
    \begin{column}{0.5\textwidth}
      \begin{center}
        \includegraphics[height=0.6\textheight]{../diagrams/diamond-grammar.pdf}
      \end{center}

      How do we express the language of $Bottom$ terms whose $Left$ and
      $Right$ antecedents are generated by the same $Top$ antecedent?
    \end{column}
  \end{columns}
\end{frame}


\begin{frame}
  \frametitle{Generative Intuition}

  \begin{columns}[c]
    \begin{column}{0.7\textwidth}
\setbeamercolor{block title}{bg=limegreen}
      \setbeamercolor{block body}{bg=limegreen}
      \begin{block}{}
        \begin{itemize}
        \item Node = \textcolor{red}{Term}
        \item Edge = Image with a \textcolor{darkgreen}{lexicon}
        \item Two incoming edges = \textcolor{red}{Equality} of images
        \end{itemize}
      \end{block}

      \vspace{3 mm}

      $t \in \mathcal{P}_{\mathcal{G}}(u)$ iff $\exists T :
      \textcolor{red}{V(G) \to \Lambda}$ s.t.:
      \begin{itemize}
      \item $T_u = t$.
      \item $\forall v \in V(G)$, $\vdash_{\Sigma_v} T_v : S_v$.
      \item $\forall (v,w) \in E(G)$,
        $\textcolor{darkgreen}{\mathcal{L}_{(v,w)}(T_v) = T_w}$.
      \end{itemize}
    \end{column}
    \begin{column}{0.3\textwidth}
      \includegraphics[width=\textwidth]{../diagrams/generative-perspective.pdf}
    \end{column}
  \end{columns}
\end{frame}


\begin{frame}
  \frametitle{Algebraic and Generative Relations}

  \note{We have presented two interpretations of G-ACGs based on two different
    intuitive readings of ACG diagrams, one where nodes are thought of as
    languages and one where they stand for individual terms. The result shown
    here validates these two ways of reasoning about diagrams in the context
    of classical ACGs since they lead to the same languages as the canonical
    definition of an object language.

    0:25}

  \begin{itemize}
  \item $\forall \, \mathcal{G}, u$ where $\mathcal{G}$ is arborescent:
    $$\mathcal{E}_{\mathcal{G}}(u) = \mathcal{P}_{\mathcal{G}}(u)
    = \mathcal{O}_{\mathcal{G}}(u)$$
  \pause
  \item However, $\exists \, \mathcal{G}, u$ such that:
    $$\mathcal{E}_{\mathcal{G}}(u) \neq \mathcal{P}_{\mathcal{G}}(u)$$
  \end{itemize}
\end{frame}


\begin{frame}
  \frametitle{Properties of G-ACG Languages}

  $$\forall \, \mathcal{G}, u. \; \mathcal{I}_{\mathcal{G}}(u) \supseteq
  \mathcal{E}_{\mathcal{G}}(u) \supseteq \mathcal{P}_{\mathcal{G}}(u)$$

  $$\mathcal{I} \subseteq \mathcal{E} \subseteq \mathcal{P}$$

  $$\mathcal{I} = \mathcal{A}$$

  $$\mathcal{O} = \mathcal{A}^{\mathcal{L}}$$

  $$\mathcal{E} = \mathcal{A}^{\mathcal{L}{\cap}}$$
\end{frame}

\section{Illustration}

\note{We now illustrate the kind of grammar for which we introduced this
  extension.}


\begin{frame}
  \frametitle{Basic Setup}
  \begin{center}
    \includegraphics[height=0.8\textheight]{../diagrams/illustration0.pdf}
  \end{center}
\end{frame}

\note{We start with the canonical example of a tectogrammatical language that
  is projected into phenogrammatical strings and semantic formulae.

  0:09}


\begin{frame}
  \addtocounter{framenumber}{-1}
  \frametitle{Single Constraint}
  \begin{center}
    \includegraphics[height=0.8\textheight]{../diagrams/illustration1.pdf}
  \end{center}
\end{frame}


\begin{frame}
  \addtocounter{framenumber}{-1}
  \frametitle{Multiple Constraints}
  \begin{center}
    \includegraphics[height=0.8\textheight]{../diagrams/illustration2.pdf}
  \end{center}
\end{frame}


\begin{frame}
  \addtocounter{framenumber}{-1}
  \frametitle{Exposing Another Signature}
  \begin{center}
    \includegraphics[height=0.8\textheight]{../diagrams/illustration3.pdf}
  \end{center}
\end{frame}


\begin{frame}
  \addtocounter{framenumber}{-1}
  \frametitle{Constraints on Multiple Signatures}
  \begin{center}
    \includegraphics[height=0.8\textheight]{../diagrams/illustration4.pdf}
  \end{center}
\end{frame}


\begin{frame}
  \frametitle{Reflection}
  \begin{columns}[c]
    \begin{column}{0.5\textwidth}
      \begin{center}
        \includegraphics[height=0.8\textheight]{../diagrams/illustration5.pdf}
      \end{center}
    \end{column}
    \begin{column}{0.5\textwidth}
      \begin{itemize}
        \item 9 signatures
        \item 8 lexicons
        \item High level of redundancy
        \item Constraints decomplected from each other
        \item Constraints on different signatures
      \end{itemize}
    \end{column}
  \end{columns}
\end{frame}



\section{Conclusion}

\begin{frame}
  \frametitle{Summary of Work Done}

  \begin{itemize}
  \item Formalized graphical abstract categorial grammars.
    \begin{itemize}
      \item ACG diagrams as mathematical objects
      \item Non-arborescence enables intersection
      \item Intersection enables conjunction of independent constraints
      \item Large-scale grammar: collection of fragments
    \end{itemize}
  \vfill
  \item Created a toolkit to serve as basis for more explorations or
    implementations.
  \vfill
  \item Direct embedding of IGs into ACGs based on Perrier's theorem
    conjectured not to be practical.
  \end{itemize}
\end{frame}


\begin{frame}
  \frametitle{Future Directions}

  \begin{itemize}
  \item Are ACG object languages closed on
    intersection? $$\mathcal{O}^{\mathcal{L}{\cap}} \subseteq \mathcal{O}^{\mathcal{L}}$$
  \item Are extrinsic languages as expressive as pangraphical
    ones? $$\mathcal{P} \subseteq \mathcal{E}$$
  \item Express the multiple polarized features of IGs in G-ACGs.
  \vfill
  \item Is there a metagrammatical approach that makes refining the types of a
    signature more concise?
  \vfill
  \item Is this enough to build a satisfactory grammar of a real language?
  \end{itemize}
\end{frame}

\end{document}
