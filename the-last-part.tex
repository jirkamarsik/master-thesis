\section{The Last Part}

In this section, we will attempt to incorporate accounts of several
linguistic phenomena in a single framework, highlight the modularity
issues this brings up and propose some solutions to them.


\subsection{Negation}

We will start with an account of the paired grammatical words mechanism
for negation in French. Recall the EPTDs we have seen
in~\ref{sssec:frigram}, repeated here on Figure~\ref{fig:ig-neg-rep}.

\begin{figure}
  \centering
  \includegraphics[scale=0.25]{ig-neg.pdf}
  \caption{\label{fig:ig-neg-rep} Frigram EPTDs for a negative
    determiner and the paired grammatical word $ne$.}
\end{figure}

At a basic level, $aucun$ functions like any other determiner and $ne$
as any verb modifier. However, we want to encode the constraint that
whenever a verb is modified by $ne$, the modified verb will demand that
one of its arguments contains a negative determiner. Since we need to
make a distinction between phrases that contain negative determiners and
those that do not, we will need to provide different types for both (two
terms with the same type are indistinguishable w.r.t. syntactic
composition in a type-logical grammar such as ACGs). Our type for
$aucun$ will thus end up being $N\_NEG{=}F \limp ((NP\_NEG{=}T \limp S)
\limp S)$ instead of the usual $N \limp ((NP \limp S) \limp
S)$.\footnote{Strictly following the EPTD for $aucun$ would lead us to
  give it a type $DET\_NEG{=}F$ and then give the two types
  $DET\_NEG{=}F \limp ((NP\_NEG{=}F \limp S) \limp S)$ and $DET\_NEG{=}T
  \limp ((NP\_NEG{=}T \limp S) \limp S)$ to nouns. In this
  demonstration, we prefer to align our examples with the treatment
  which is customary to ACG research.} The type for $le$ will need to be
able handle cases where its argument noun either already contains a
negative determiner or not.

$$
N_{aucun} : N\_NEG{=}F \limp ((NP\_NEG{=}T \limp S) \limp S)
$$
$$
N_{le_1} : N\_NEG{=}F \limp ((NP\_NEG{=}F \limp S) \limp S)
$$
$$
N_{le_2} : N\_NEG{=}T \limp ((NP\_NEG{=}T \limp S) \limp S)
$$


The type of $ne$ will be more verbose, as it will consume a verb and
transform its valency so that it demands that exactly one of its $NP$
arguments contains a negative determiner.

$$
N_{ne_{tv_1}} : (NP\_NEG{=}F \limp NP\_NEG{=}F \limp S) \limp (NP\_NEG{=}T \limp NP\_NEG{=}F \limp S)
$$
$$
N_{ne_{tv_2}} : (NP\_NEG{=}F \limp NP\_NEG{=}F \limp S) \limp (NP\_NEG{=}F \limp NP\_NEG{=}T \limp S)
$$

Furthermore, we could add the case when both the subject and the object
contain negative determiners, such as in~(\ref{ex:aucun2}). This case is
currently not covered by Frigram (though it could be), but is considered
grammatical by French speakers. Finally, in a complete grammar, $ne$
would have to provide types capable of transforming all the other
syntactic valencies.

\begin{exe}
  \ex \label{ex:aucun2} Aucune fourmi n'aime aucun tatou.
\end{exe}

$$
N_{ne_{tv_3}} : (NP\_NEG{=}F \limp NP\_NEG{=}F \limp S) \limp (NP\_NEG{=}T \limp NP\_NEG{=}T \limp S)
$$

\begin{figure}
  \centering
  \includegraphics[scale=0.25]{parse-aucun.pdf}
  \caption{\label{fig:parse-aucun} One of the parse trees assigned to
    sentence (\ref{ex:aucun-obj}) by Frigram/Leopar.}
\end{figure}

However, this is not the only change we have to effect on our grammar in
order to properly handle the paired grammatical words for
negation. Consider the example sentence (\ref{ex:aucun-obj}) and the
parse tree assigned to it by Frigram on Figure~\ref{fig:parse-aucun}. It
is not enough that the embedded noun phrase $aucun\ tatou$ has a type
that tells us it contains a negative determiner. We need this bit of
internal information also at the level of the enclosing noun phrase
$l'odeur\ d'aucun\ tatou$ in which the former noun phrase is embedded
since this is a property of the enclosing noun phrase that is crucial to
its syntactic combinatorics. This presupposes the existence of a
mechanism for propagating this kind of information up the parse tree. In
the case of our ACG, this means that all the functions which modify
phrases must be aware whether or not their argument contains a negative
determiner and to convey this information in its result type as
well. Moreover, since the negative determiner can also come from a
complement of a prepositional phrase, the types of prepositions will
need to take into account the presence of negative determiners in both
their complements and the phrases they modify.

$$
N_{que_1} : (NP\_NEG{=}F \limp S) \limp N\_NEG{=}F \limp N\_NEG{=}F
$$
$$
N_{que_2} : (NP\_NEG{=}F \limp S) \limp N\_NEG{=}T \limp N\_NEG{=}T
$$
$$
N_{de_1} : NP\_NEG{=}F \limp N\_NEG{=}F \limp N\_NEG{=}F
$$
$$
N_{de_2} : NP\_NEG{=}F \limp N\_NEG{=}T \limp N\_NEG{=}T
$$
$$
N_{de_3} : NP\_NEG{=}T \limp N\_NEG{=}F \limp N\_NEG{=}T
$$

If we suppose the presence of the appropriate lexical items typed as in
the examples of~\ref{ssec:acg}, we have a type system in which we can
type the syntactic terms corresponding to the sentences
(\ref{ex:good-neg-double}), (\ref{ex:good-neg-embed}) and
(\ref{ex:good-neg-rel}) while making it impossible to type the
structures corresponding to sentences (\ref{ex:bad-neg-noneg}) and
({\ref{ex:bad-neg-rel}}).

In sentence (\ref{ex:bad-neg-noneg}), we have no constant
$N_{ne_{tv_?}}$ for $ne$ which would yield a verb capable of consuming
two noun phrases that contain no negative determiners.

In sentence (\ref{ex:bad-neg-rel}), there are multiple problems with
typing. First, $N_{chasse}$ requires its first argument to be of type
$NP\_NEG{=}F$, a noun phrase with no negative determiner. This means
that $\lambda^{\circ} x.\ N_{chasse}\ x\ y$ has type $NP\_NEG{=}F \limp
S$ which makes it an invalid argument for the function
$N_{aucun}\ N_{loup}$, which has type $(NP\_NEG{=}T \limp S) \limp
S$. Even if we were able to assign it a type compatible with
$N_{que_1}$, the final relative clause would map $N_{tatou}$ to another
term of type $N\_NEG{=}F$, which has no ``free'' negative
determiners. $N_{le_1}$ will map this to a term typed $(NP\_NEG{=}F
\limp S) \limp S$. Finally, this term is not capable of taking
$(\lambda^{\circ} x.\ N_{ne_{iv}}\ N_{court}\ x)$ as its argument, since
its type is $NP\_NEG{=}T \limp S$.

\begin{exe}
  \ex \label{ex:good-neg-double} Aucune fourmi n'aime aucun tatou. \\
      $(N_{aucune}\ N_{fourmi})\ (\lambda^{\circ} x.\ (N_{aucun}\ N_{tatou})\ (\lambda^{\circ} y.\ N_{ne_{tv_3}}\ N_{aime}\ x\ y))$
  \ex \label{ex:good-neg-embed} Jean n'aime l'odeur d'aucun tatou. \\
      $(N_{le_2}\ (N_{de_3}\ (N_{aucun}\ N_{tatou})\ N_{odeur}))\ (\lambda^{\circ} y.\ N_{ne_{tv_2}}\ N_{aime}\ N_{Jean}\ y)$
  \ex \label{ex:good-neg-rel} Le tatou qu'aucun loup ne chasse court. \\
      $(N_{le_1}\ (N_{que_1}\ (\lambda^{\circ} y.\ (N_{aucun}\ N_{loup})\ (\lambda^{\circ} x.\ N_{ne_{tv_1}}\ N_{chasse}\ x\ y))\ N_{tatou}))\ (\lambda^{\circ} x.\ N_{court}\ x)$
  \ex * \label{ex:bad-neg-noneg} Jean n'aime le tatou. \\
      * $(N_{le_1}\ N_{tatou})\ (\lambda^{\circ} y.\ N_{ne_{tv_?}}\ N_{aime}\ N_{Jean}\ y)$
  \ex * \label{ex:bad-neg-rel} Le tatou qu'aucun loup chasse ne court. \\
      * $(N_{le_1}\ (N_{que}\ (\lambda^{\circ} y.\ (N_{aucun}\ N_{loup})\ (\lambda^{\circ} x.\ N_{chasse}\ x\ y))\ N_{tatou}))\ (\lambda^{\circ} x.\ N_{ne_{iv}}\ N_{court}\ x)$
\end{exe}

There are two things that we would like to highlight about this
grammatical treatment. First off, we had to refine our types to make
them convey more than one piece of information in a single type
(e.g. something is a noun phrase and at the same time it is something
that contains a negative determiner). Second, a property that we wish to
express in the type of a term may be due to one of its subconstituents
and it is therefore necessary to encode the propagation of this property
up the parse tree in the types of all the intervening operators. In our
case, it meant that handling negative determiners and the paired
grammatical word for negation required us to not only define and type
constants for these two categories, but to also change the types of noun
modifiers ($que\ ...$) and prepositions ($de$).


\subsection{Extraction}
\label{ssec:extraction}

We will now turn to another phenomenon, extraction in relative clauses,
focus on one of its properties, the fact that subjects can be extracted
using $qui$ only from the relative clause itself whereas objects can be
extracted using $que$ from embedded clauses contained therein, and
present an implementation of this constraint by way of types in an
ACG. The solution that we will show was published in
\cite{pogodalla2012controlling}. Our presentation is slightly adapted to
conform to the patterns seen in previous examples and it does not use
dependent types as they are not essential to this specific constraint.

The key idea in the treatment presented below is to give a special type
to extracted constituents. We will thus have two types for noun phrases:
$NP\_VAR{=}T$, which stands for ``empty'' noun phrases introduced by
extraction, and $NP\_VAR{=}F$, which stands for the other noun phrases,
i.e.  proper names and nouns with determiners. The grammar gives no
constant capable of constructing a value of type $NP\_VAR{=}T$. The only
way to obtain a term with this type is to abstract over a variable
having the type. Then, the types of relative pronouns end up being
$(NP\_VAR{=}T \limp S) \limp N \limp N$, ensuring that the incomplete
clause given as the first argument is a $\lambda$-abstraction whose
argument has type $NP\_VAR{=}T$.

Once we have this kind of information in the type system, we can start
distinguishing clauses containing rooted extraction, where a subject or
object is extracted directly from the relative clause, embedded
extraction, where an object is extracted from a clause embedded inside
the relative clause, or no extraction. Types of functions which produce
values of type $S$ will be sensitive to the presence of extracted
variables in their noun phrase and clausal arguments.

$$
E_{dort_1} : NP\_VAR{=}F \limp S\_EXT{=}NO
$$
$$
E_{dort_2} : NP\_VAR{=}T \limp S\_EXT{=}ROOT
$$
$$
E_{aime_1} : NP\_VAR{=}F \limp NP\_VAR{=}F \limp S\_EXT{=}NO
$$
$$
E_{aime_2} : NP\_VAR{=}T \limp NP\_VAR{=}F \limp S\_EXT{=}ROOT
$$
$$
E_{aime_3} : NP\_VAR{=}F \limp NP\_VAR{=}T \limp S\_EXT{=}ROOT
$$
$$
E_{dit\ que_1} : NP\_VAR{=}F \limp S\_EXT{=}NO \limp S\_EXT{=}NO
$$
$$
E_{dit\ que_2} : NP\_VAR{=}T \limp S\_EXT{=}NO \limp S\_EXT{=}ROOT
$$
$$
E_{dit\ que_3} : NP\_VAR{=}F \limp S\_EXT{=}ROOT \limp S\_EXT{=}EMB
$$
$$
E_{dit\ que_4} : NP\_VAR{=}F \limp S\_EXT{=}EMB \limp S\_EXT{=}EMB
$$

Now all of the infrastructure is in place for us to express the type of
the relative pronouns $qui$ and $que$.

$$
E_{qui} : (NP\_VAR{=}T \limp S\_EXT{=}ROOT) \limp N \limp N
$$
$$
E_{que_1} : (NP\_VAR{=}T \limp S\_EXT{=}ROOT) \limp N \limp N
$$
$$
E_{que_2} : (NP\_VAR{=}T \limp S\_EXT{=}EMB) \limp N \limp N
$$

The above types ensure that the abstracted variables corresponding to
traces of extracted constituents will be given the type $NP\_VAR{=}T$,
that $qui$ can only be used to extract constituents from root positions
while $que$ allows for extraction from both root and embedded
positions.\footnote{NB: The constraint that $qui$ can only extract
  subjects and that $que$ can only extract objects is not enforced
  here. Likewise, this fragment does not handle multiple extraction.}

We also note that the splitting of the clause type $S$ into three finer
types $S\_EXT={NO}$, $S\_EXT{=}ROOT$ and $S\_EXT{=}EMB$ means that types
that work with $S$ regardless of its extraction status will have to
provide alternatives for all the possible variants.

$$
E_{le_1} : N \limp ((NP\_VAR{=}F \limp S\_EXT{=}NO) \limp S\_EXT{=}NO)
$$
$$
E_{le_2} : N \limp ((NP\_VAR{=}F \limp S\_EXT{=}ROOT) \limp S\_EXT{=}ROOT)
$$
$$
E_{le_3} : N \limp ((NP\_VAR{=}F \limp S\_EXT{=}EMB) \limp S\_EXT{=}EMB)
$$

While we have introduced the notion of an $NP\_VAR{=}T$ for describing
traces, we can cover another element of our fragment, the preposition
$de$, and enforce the constraint that $qui$/$que$ cannot extract
prepositional complements.

$$
E_{de} : NP\_VAR{=}F \limp N \limp N
$$

Given this type system, we can assign the type $S\_EXT{=}NO$ to the term
expressing the syntactic structure of (\ref{ex:good-ext}) but we cannot
a find a typable term for sentence (\ref{ex:bad-ext}), where $qui$ is
used to extract a subject from a clause embedded in a relative clause.
The problem with sentence (\ref{ex:bad-ext}) is that the type of the
relative clause is $NP\_VAR{=}T \limp S\_EXT{=}EMB$ which is not a valid
argument for any constant representing the relative pronoun
$qui$. Typing the variable $t$ with $NP\_VAR{=}F$ instead of
$NP\_VAR{=}T$ and using $E_{aime_1}$ instead of $E_{aime_2}$ would not
help, since the resulting relative clause would then have type
$NP\_VAR{=}F \limp S\_EXT{=}NO$, which would not be admitted by any
relative pronoun.

\begin{exe}
  \ex * \label{ex:bad-ext} Le tatou qui Jean dit que \_ aime Marie dort. \\
      * $(E_{le_1}\ (E_{qui_?}\ (\lambda^{\circ} t.\ E_{dit\ que_3}\ E_{Jean}\ (E_{aime_2}\ t\ E_{Marie}))\ E_{tatou}))\ (\lambda^{\circ} x.\ E_{dort_1}\ x)$
  \ex \label{ex:good-ext} Le tatou que Jean dit que Marie aime \_ dort. \\
      $(E_{le_1}\ (E_{que_2}\ (\lambda^{\circ} t.\ E_{dit\ que_3}\ E_{Jean}\ (E_{aime_3}\ E_{Marie}\ t))\ E_{tatou}))\ (\lambda^{\circ} x.\ E_{dort_1}\ x)$
\end{exe}

As was the case with our treatment of negation, we refined our familiar
types by tacking on new kinds of information and defined rules for how
these pieces of information percolate up the parse tree all the way to
the topmost level still pertinent for ensuring grammaticality. We will
now look at another similar example.


\subsection{Agreement}

Finally, we will cover the phenomenon of number agreement. This
motivates a straightforward refinement of our types, where all noun and
noun phrase types carry number information.

$$
A_{Marie} : NP\_NUM{=}SG
$$
$$
A_{tatou} : N\_NUM{=}SG
$$
$$
A_{tatous} : N\_NUM{=}PL
$$
$$
A_{le} : N\_NUM{=}SG \limp ((NP\_NUM{=}SG \limp S) \limp S)
$$
$$
A_{les} : N\_NUM{=}PL \limp ((NP\_NUM{=}PL \limp S) \limp S)
$$
$$
A_{dort} : NP\_NUM{=}SG \limp S
$$
$$
A_{dorment} : NP\_NUM{=}PL \limp S
$$
$$
A_{aime_1} : NP\_NUM{=}SG \limp NP\_NUM{=}SG \limp S
$$
$$
A_{aime_2} : NP\_NUM{=}SG \limp NP\_NUM{=}PL \limp S
$$
$$
A_{aiment_1} : NP\_NUM{=}PL \limp NP\_NUM{=}SG \limp S
$$
$$
A_{aiment_2} : NP\_NUM{=}PL \limp NP\_NUM{=}PL \limp S
$$
$$
A_{qui_1} : (NP\_NUM{=}SG \limp S) \limp N\_NUM{=}SG \limp N\_NUM{=}SG
$$
$$
A_{qui_2} : (NP\_NUM{=}PL \limp S) \limp N\_NUM{=}PL \limp N\_NUM{=}PL
$$
$$
A_{de_1} : NP\_NUM{=}SG \limp N\_NUM{=}SG \limp N\_NUM{=}SG
$$
$$
A_{de_2} : NP\_NUM{=}PL \limp N\_NUM{=}SG \limp N\_NUM{=}SG
$$
$$
A_{de_3} : NP\_NUM{=}SG \limp N\_NUM{=}PL \limp N\_NUM{=}PL
$$
$$
A_{de_4} : NP\_NUM{=}PL \limp N\_Pl \limp N\_NUM{=}PL
$$


Given the above signature, we can assign the type $S$ to the syntactic
term of sentence (\ref{ex:good-agr}) but not to the one of sentence
(\ref{ex:bad-agr}). In the latter sentence, the type of the variable $t$
has to be $NP\_NUM{=}SG$ since it is used as an argument to $A_{dort}$. This
means that the relative clause $dort$ has type $NP\_NUM{=}SG \limp
S$. However, there is no type for $qui$ which would allow us to modify
the plural noun $A_{tatous}$ of type $N\_NUM{=}PL$ by a relative clause with a
missing singular noun phrase, which has the type $NP\_NUM{=}SG \limp S$.

\begin{exe}
  \ex \label{ex:good-agr} Les tatous de Marie dorment. \\
      $(A_{les}\ (A_{de}\ A_{Marie}\ A_{tatous}))\ (\lambda^{\circ} x.\ A_{dorment}\ x)$
  \ex * \label{ex:bad-agr} Marie aime les tatous qui dort. \\
      * $(A_{les}\ (A_{qui_?}\ (\lambda^{\circ} t.\ A_{dort}\ t)\ A_{tatous}))\ (\lambda^{\circ} y.\ A_{aime_2}\ A_{Marie}\ y)$
\end{exe}

Once again, we have expressed a linguistic constraint by refining our
types, augmenting them with a new bit of information, we constrained the
arguments of functions using these new types and we have described how
this new type information propagates from the given arguments to the
produced values.

In the next subsection, we will see what it takes to enforce all of the
constraints above (negation, extraction and agreement) at the same time,
which will give us some idea on how to go about building a grammar which
handles a wide variety of such phenomena.


\subsection{Putting It All Together}

In a grammar that handles negation, extraction and agreement, our types
will need to carry all of the refinements we introduced before. To
recapitulate, $NP$ will be parameterized by whether or not it contains a
negative determiner ($\_\textcolor{red}{NEG{=}T}$ and
$\_\textcolor{red}{NEG{=}F}$ respectively), whether or not it is a trace
from an extraction ($\_\textcolor{green}{VAR{=}T}$ and
$\_\textcolor{green}{VAR{=}F}$) and its number
($\_\textcolor{blue}{NUM{=}SG}$ or $\_\textcolor{blue}{NUM{=}PL}$). For
$N$, we have the same refinements except for $\textcolor{green}{VAR}$
since extraction only moves $NP$s. Clauses, of type $S$, will be
discriminated based on the level of extraction within them
($\_\textcolor{green}{EXT{=}NO}$, $\_\textcolor{green}{EXT{=}ROOT}$ and
$\_\textcolor{green}{EXT{=}EMB}$).

The individual types will have to respect all of the constraints at the
same time, each one being a complete specification of a possible
situation (complete w.r.t. the features listed above). To see and
appreciate what this means in practice, we give the types for the
preposition $de$:

$$
C_{de_1} : (NP\_\textcolor{red}{NEG{=}F}\_\textcolor{green}{VAR{=}F}\_\textcolor{blue}{NUM{=}SG}) \limp (N\_\textcolor{red}{NEG{=}F}\_\textcolor{blue}{NUM{=}SG}) \limp (N\_\textcolor{red}{NEG{=}F}\_\textcolor{blue}{NUM{=}SG})
$$
$$
C_{de_2} : (NP\_\textcolor{red}{NEG{=}F}\_\textcolor{green}{VAR{=}F}\_\textcolor{blue}{NUM{=}SG}) \limp (N\_\textcolor{red}{NEG{=}F}\_\textcolor{blue}{NUM{=}PL}) \limp (N\_\textcolor{red}{NEG{=}F}\_\textcolor{blue}{NUM{=}PL})
$$
$$
C_{de_3} : (NP\_\textcolor{red}{NEG{=}F}\_\textcolor{green}{VAR{=}F}\_\textcolor{blue}{NUM{=}SG}) \limp (N\_\textcolor{red}{NEG{=}T}\_\textcolor{blue}{NUM{=}SG}) \limp (N\_\textcolor{red}{NEG{=}T}\_\textcolor{blue}{NUM{=}SG})
$$
$$
C_{de_4} : (NP\_\textcolor{red}{NEG{=}F}\_\textcolor{green}{VAR{=}F}\_\textcolor{blue}{NUM{=}SG}) \limp (N\_\textcolor{red}{NEG{=}T}\_\textcolor{blue}{NUM{=}PL}) \limp (N\_\textcolor{red}{NEG{=}T}\_\textcolor{blue}{NUM{=}PL})
$$
$$
C_{de_5} : (NP\_\textcolor{red}{NEG{=}T}\_\textcolor{green}{VAR{=}F}\_\textcolor{blue}{NUM{=}SG}) \limp (N\_\textcolor{red}{NEG{=}F}\_\textcolor{blue}{NUM{=}SG}) \limp (N\_\textcolor{red}{NEG{=}T}\_\textcolor{blue}{NUM{=}SG})
$$
$$
C_{de_6} : (NP\_\textcolor{red}{NEG{=}T}\_\textcolor{green}{VAR{=}F}\_\textcolor{blue}{NUM{=}SG}) \limp (N\_\textcolor{red}{NEG{=}F}\_\textcolor{blue}{NUM{=}PL}) \limp (N\_\textcolor{red}{NEG{=}T}\_\textcolor{blue}{NUM{=}PL})
$$
$$
C_{de_7} : (NP\_\textcolor{red}{NEG{=}F}\_\textcolor{green}{VAR{=}F}\_\textcolor{blue}{NUM{=}PL}) \limp (N\_\textcolor{red}{NEG{=}F}\_\textcolor{blue}{NUM{=}SG}) \limp (N\_\textcolor{red}{NEG{=}F}\_\textcolor{blue}{NUM{=}SG})
$$
$$
C_{de_8} : (NP\_\textcolor{red}{NEG{=}F}\_\textcolor{green}{VAR{=}F}\_\textcolor{blue}{NUM{=}PL}) \limp (N\_\textcolor{red}{NEG{=}F}\_\textcolor{blue}{NUM{=}PL}) \limp (N\_\textcolor{red}{NEG{=}F}\_\textcolor{blue}{NUM{=}PL})
$$
$$
C_{de_9} : (NP\_\textcolor{red}{NEG{=}F}\_\textcolor{green}{VAR{=}F}\_\textcolor{blue}{NUM{=}PL}) \limp (N\_\textcolor{red}{NEG{=}T}\_\textcolor{blue}{NUM{=}SG}) \limp (N\_\textcolor{red}{NEG{=}T}\_\textcolor{blue}{NUM{=}SG})
$$
$$
C_{de_{10}} : (NP\_\textcolor{red}{NEG{=}F}\_\textcolor{green}{VAR{=}F}\_\textcolor{blue}{NUM{=}PL}) \limp (N\_\textcolor{red}{NEG{=}T}\_\textcolor{blue}{NUM{=}PL}) \limp (N\_\textcolor{red}{NEG{=}T}\_\textcolor{blue}{NUM{=}PL})
$$
$$
C_{de_{11}} : (NP\_\textcolor{red}{NEG{=}T}\_\textcolor{green}{VAR{=}F}\_\textcolor{blue}{NUM{=}PL}) \limp (N\_\textcolor{red}{NEG{=}F}\_\textcolor{blue}{NUM{=}SG}) \limp (N\_\textcolor{red}{NEG{=}T}\_\textcolor{blue}{NUM{=}SG})
$$
$$
C_{de_{12}} : (NP\_\textcolor{red}{NEG{=}T}\_\textcolor{green}{VAR{=}F}\_\textcolor{blue}{NUM{=}PL}) \limp (N\_\textcolor{red}{NEG{=}F}\_\textcolor{blue}{NUM{=}PL}) \limp (N\_\textcolor{red}{NEG{=}T}\_\textcolor{blue}{NUM{=}PL})
$$

This highlights two important problems with the current approach. One is
the untenable complexity. Even though the three mechanisms are
completely independent of each other, here we are forced to refer to all
of them in every type. It is no longer easy to glance at the types
assigned to $de$ and see how it behaves with respect to, for example,
the mechanism of negation (compare with the type assignments we gave to
$de$ in previous subsections). Another cause of unreadability of this
signature is due to the sheer number of types on display, which leads us
to our second issue.

The signature demonstrates an exponential growth in the number of typed
constants included. In cases where the behaviors of a wordform across
the different mechanisms are independent, the complete type signature
will need to provide a type for every combination of situations in all
of the mechanisms. Assuming that every mechanism will assign on average
more than one type to every wordform, the number of the wordform's types
in the final signature will be exponential w.r.t. the number of
different mechanisms handled.

We can try making the enumeration of these types easier for the grammar
author by introducing metavariables for the values of the features. This
means that we could write the following 3 templates instead of the 12
types above:

$$
\forall \textcolor{blue}{x}, \textcolor{blue}{y} \in \{\textcolor{blue}{SG}, \textcolor{blue}{PL}\}
$$
$$
C_{de_{1(\textcolor{blue}{x}, \textcolor{blue}{y})}} : (NP\_\textcolor{red}{NEG{=}F}\_\textcolor{green}{VAR{=}F}\_\textcolor{blue}{NUM{=}x}) \limp (N\_\textcolor{red}{NEG{=}F}\_\textcolor{blue}{NUM{=}y}) \limp (N\_\textcolor{red}{NEG{=}F}\_\textcolor{blue}{NUM{=}y})
$$
$$
C_{de_{2(\textcolor{blue}{x}, \textcolor{blue}{y})}} : (NP\_\textcolor{red}{NEG{=}F}\_\textcolor{green}{VAR{=}F}\_\textcolor{blue}{NUM{=}x}) \limp (N\_\textcolor{red}{NEG{=}T}\_\textcolor{blue}{NUM{=}y}) \limp (N\_\textcolor{red}{NEG{=}T}\_\textcolor{blue}{NUM{=}y})
$$
$$
C_{de_{3(\textcolor{blue}{x}, \textcolor{blue}{y})}} : (NP\_\textcolor{red}{NEG{=}T}\_\textcolor{green}{VAR{=}F}\_\textcolor{blue}{NUM{=}x}) \limp (N\_\textcolor{red}{NEG{=}F}\_\textcolor{blue}{NUM{=}y}) \limp (N\_\textcolor{red}{NEG{=}T}\_\textcolor{blue}{NUM{=}y})
$$

This is getting more concise and easier to comprehend. We can take this
approach further by not only quantifying over variables but also
computing some of the values using functions:

$$
\forall \textcolor{blue}{x}, \textcolor{blue}{y} \in\{\textcolor{blue}{SG}, \textcolor{blue}{PL}\}, \forall \textcolor{red}{p}, \textcolor{red}{q} \in \{\textcolor{red}{F}, \textcolor{red}{T}\}, \textcolor{red}{p \land q} \neq \textcolor{red}{T}
$$
$$
C_{de(\textcolor{blue}{x}, \textcolor{blue}{y}, \textcolor{red}{p}, \textcolor{red}{q})} : (NP\_\textcolor{red}{NEG{=}p}\_\textcolor{green}{VAR{=}F}\_\textcolor{blue}{NUM{=}x}) \limp (N\_\textcolor{red}{NEG{=}q}\_\textcolor{blue}{NUM{=}y}) \limp (N\_\textcolor{red}{NEG{=}(p \lor q)}\_\textcolor{blue}{NUM{=}y})
$$

The above is vastly more concise and manages to convey the intent more
clearly (the intent being that it cannot be the case that there are
negative determiners in both the complement noun phrase and the modified
noun and that the value of the result's \textcolor{red}{NEG} feature is
always the disjunction of the \textcolor{red}{NEG} features of the two
arguments).

As of right now, we have introduced this kind of rule only as
meta-notation. This means that the real underlying grammar that will end
up being used by some algorithm still suffers from the same exponential
cardinality of the signature. Fortunately, there exists a well-studied
construction in type theory that lets us assign to terms generic types
such as the one defined in the template above. This construction is the
\emph{dependent product} and its utility in ACGs has already been
noted.\footnote{See \cite{de2007two} for its introduction into the
  domain of ACGs, \cite{de2007type} for a small example grammar, which
  handles the interaction between number and gender agreement and
  coordination, and finally see \cite{pogodalla2012controlling} for
  using dependent types to handle several constraints regarding
  extraction (including the one we covered in~\ref{ssec:extraction}).}

While extending the type system by including dependent types is
definitely useful and it helps to solve the problem of having our
signatures grow exponentially in size, it does not solve our problem
with the complexity of the new grammar. Granted, the new meta-type
written in the dependent style is more readable and easier to reason
about than the enumeration of 12 basic types we presented at the
beginning of this subsection, however, the result is still complex in
that it forces us to describe the behavior of three independent
mechanisms in one place. This one place happens to be a wordform which
has seemingly little connection to these mechanisms, the preposition
$de$. We can therefore imagine that it will participate in many other
phenomena and that the type we assign to it will grow beyond our
abilities to reason about it effectively.

Furthermore, the current notation does not even make it obvious that
these are independent mechanisms. One has to examine the type to make
sure that there is no interaction. In the next two subsections, we will
devote ourselves to investigating ways of disentangling these.


\subsection{Intersections of Types}
